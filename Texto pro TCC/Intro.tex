\chapter {Introdu\c{c}\~{a}o}
\label{Intro}

\subparagraph{
O sigilo sempre foi uma arma explorada pelos seres humanos para vencer certas batalhas, mesmo que na cotidiana miss\~{a}o de se 	comunicar. Foi a partir dessa necessidade que se criou o que chamamos de \textit{criptografia}, nome dado ao conjunto de t�cnicas usadas para se falar e escrever em c\'odigos. Seu objetivo \'{e} garantir que apenas as pessoas  envolvidas na comunica\c{c}\~ao possam compreender a mensagem codificada (ou criptogtafada), garantindo que terceiros n\~ao saibam o que foi conversado.
}
\subparagraph{
Para compreender como funciona o processo de codifica\c{c}\~ao e decodifica\c{c}\~ao faz-se necess\'ario o uso de uma s�rie de termos t\'ecnicos, para fins pedag�gicos iremos introduzir tais conceitos apresentando um dos primeiros algor\'itmos criptogr\'aficos que se tem conhecimento, a criptografia de C\'esar, al\'em de seus sucessores. Caso queira se aprofudar sobre criptografia recomendamos a leitura de ``Criptografia" , por Coutinho \cite{coutinho}
}
\subparagraph{
A chamada \textit{criptografia de C\'esar}, criada pelo imperador romano C\'esar Augusto, consistia em substituir cada letra da mensagem
por outra que estivesse a tr\^es posi\c{c}\~oes a frente, como, por exemplo, a letra \textbf{A} era substitu\'ida pela letra \textbf{D}.  
}
\subparagraph{
Uma forma muito natural de se generalizar o algoritmo de C\'esar \'e fazer a troca de cada letra da mensagem por outra em uma posi\c{c}\~ao qualquer fixada. A chamada \textit{criptografia de substitui\c{c}\~ao monoalfab\'etica} consite em substituir cada letra por outra que ocupe $n$ posi\c{c}\~oes a sua frente, sendo que o n\'umero $n$ \'e conhecido apenas pelo emissor e pelo receptor da mensagem. Chamamos este n\'umero $n$ de \textit{chave criptogr\'afica}. Para podermos compreender a mensagem, precisamos substituir as letras que formam a mensagem criptografada pelas as que est\~ao $n$ posi\c{c}\~oes antes.
}
\subparagraph{
O algoritmo monoalfab\'etico tem a caracter\'istica indesejada de ser de f\'acil decodifica\c{c}\~ao, pois possui apenas {26} chaves poss\'iveis, e isso faz com que no m\'aximo em {26} tentativas o c\'odigo seja decifrado. Com o intuito de dificultar a quebra do c\'odigo monoalfab\'etico foram propostas as \textit{cifras de substitui\c{c}\~ao polialfab\'eticas} em que a chave criptogr\'afica passa a ser uma \textit{palavra} ao inv\'es de um n\'umero. A ideia \'e usar as posi\c{c}\~oes ocupadas pelas letras da chave para determinar o n\'umero de posi\c{c}\~oes que devemos avan\c{c}ar para obter a posi\c{c}\~ao da letra encriptada. Vejamos, por meio de um exemplo, como funciona esse sistema criptogr\'afico.
}
\subparagraph{
Sejam ``SENHA'' a nossa chave criptogr\'afica e ``ABOBORA'' a mensagem a ser encriptada. Abaixo colocamos as letras do alfabeto com suas respectivas posi\c{c}\~oes. Observe que repetimos a primeira linha de letras para facilitar a localiza\c{c}\~ao da posi\c{c}\~ao da letra encriptada e usamos a barra para indicar que estamos no segundo ciclo. 
}
\[
\begin{array}{ccccccccccccc}
    1      & 2 & 3 & 4 & 5          & 6 & 7 & 8          & 9 & 10 &  11 & 12 & 13 \\  
\textbf{A} & B & C & D & \textbf{E} & F & G & \textbf{H} & I & J  &  K  & L  & M  \\ 
  &   &   &   &   &   &   &   &   &    &     &    &    \\ 
    14      & 15 & 16 & 17 & 18 & 19          & 20 & 21 & 22 & 23 & 24 & 25 & 26 \\
\textbf{N}  & O  & P  & Q  & R  & \textbf{S}  & T  & U  & V  & X  & Y  & W  & Z \\
&   &   &   &   &   &   &   &   &    &     &    &    \\ 
    27      & 28 & 29 & 30 & 31          & 32 & 33 & 34          & 35 & 36 &  37 & 38 & 39 \\  
\overline{A} & \overline{B} & \overline{C} & \overline{D} & \overline{E} & \overline{F} & \overline{G} & \overline{H} & \overline{I} & \overline{J}  &  \overline{K}  & \overline{L}  & \overline{M}  \\
\end{array}
\]
\subparagraph{
Vejamos como encriptar a palavra ``ABOBORA''. Iniciamos o processo escrevendo a mensagem. Ao lado de cada letra da mensagem aparece entre par\^enteses o n\'umero que indica a sua posi\c{c}\~ao. Abaixo da mensagem escrevemos as letras da chave criptogr\'afica, repetindo-as de forma c\'iclica quando necess\'ario. Analogamente, ao lado de cada letra da chave aparece entre par\^enteses o n\'umero da posi\c{c}\~ao ocupada de cada letra, e o sinal de soma indica que devemos avan\c{c}ar aquele n�mero de posi��es. Ao final do processo aparecem as letras encriptadas. Entre par�nteses est� a posi\c{c}\~ao resultante da combina\c{c}\~ao das posi\c{c}\~oes da mensagem e da chave.   
}
\[
\begin{array}{lllllll||l}
     A (1)  &      B (2)  &      O (15) &      B (2)  &      O (15) &     R (18)  &    A (1)	 & \textrm{Mensagem}  \\
\downarrow  & \downarrow  & \downarrow  & \downarrow  & \downarrow  & \downarrow  & \downarrow &\\ 
    S (+19) &     E (+5)  &     N (+14) &     H  (+8) &     A (+1)  &    S (+19)  &   E (+5)   &\textrm{Chave}  \\
\downarrow  & \downarrow  & \downarrow  & \downarrow  & \downarrow  & \downarrow  & \downarrow & \\
		 T (20) &      G (7)  &      C (29) &      J (10) &     P (16)  &    K (37)   &    F (6)   & \textrm{Mensagem encriptada}  \\
\end{array}
\]
  
\subparagraph{
Observe que a encripta\c{c}\~ao polialfab\'etica \'e mais dif\'icil de ser quebrada que a monoalfab\'etica uma vez que letras iguais n\~ao t\^em, necessariamente, a mesma encripta\c{c}\~ao. Observe que neste tipo de criptografia o emissor precisa passar a chave para o receptor da mensagem de forma segura para que o receptor possa decifrar a mensagem, isto \'e, a chave usada para encriptar a mensagem \'e a mesma que deve ser usada para decifrar a mensagem. Veremos que esse \'e justamente o ponto fraco nesse tipo de encripta\c{c}\~ao pois usa a chamada \textit{chave sim\'etrica}, ou seja, a chave usada pelo emissor para codificar a mensagem \'e a mesma usada pelo receptor para decodificar a mensagem. Nesse processo, a chave deve ser mantida em segredo e bem guardada para garantir que o c\'odigo n\~ao seja quebrado e isso requer algum tipo de contato f\'isico entre emissor e receptor. 
}
\subparagraph{
Durante a  Primeira Guerra Mundial o contato f\'isico para a troca de chaves era complicado, e isso estimulou a cria\c{c}\~ao de m\'aquinas autom\'aticas de criptografia. O \textit{Enigma} foi uma destas m\'aquinas e era utilizada pelos alem\~aes tanto para criptografar como para descriptografar c\'odigos de guerra. Semelhante a uma m\'aquina de escrever, os primeiros modelos foram patenteados por Arthur Scherbius em 1918. Essas m\'aquinas ganharam popularidade entre as for\c{c}as militares alem\~aes devido a facilidade de uso e sua suposta indecifrabilidade do c\'odigo. 
}
\subparagraph{
O matem\'atico Alan Turing foi o respons\'avel por quebrar o c\'odigo dos alem\~aes durante a Segunda Guerra Mundial. A descoberta de Turing mostrou a fragilidade da criptografia baseada em chave sim\'etrica e colocou novos desafios \`a criptografia. O grande problema passou a ser a quest\~ao dos protocolos, isto \'e, como transmitir a chave para o receptor de forma segura sem que seja necess\'ario o contato f\'isico entre as partes? 
}
\subparagraph{%reescrever este par�grafo
Em 1949, com a publica\c{c}\~ao do artigo \textit{Communication Theory of Secrecy Systems} \cite{shannon} de Shannon, temos a inaugura\c{c}\~ao da criptografia moderna. Neste artigo ele escreve matematicamente que cifras teoricamente inquebr\'aveis s\~ao semelhantes as cifras polialfab\'eticas. Com isso ele transformou a criptografia que at\'e ent\~ao era uma arte em uma ci\^encia.
}
\subparagraph{
Em 1976 Diffie e Hellman publicaram \textit{New Directions in Cryptography} \cite{newdirections}. Neste artigo h\'a a introdu\c{c}\~ao ao conceito de \textit{chave assim\'etrica}, onde h\'a chaves diferentes entre o emissor da mensagem e seu receptor. Com a assimetria de chaves n\~ao era mais necess\'ario um contato t\~ao pr\'oximo entre emissor e receptor. Neste mesmo artigo \'e apresentado o primeiro algoritmo de criptografia de chave assim\'etrica ou como \'e mais conhecido nos dias atuais \textit{Algoritmo de Criptografia de Chave P\'ublica}, o protocolo de Diffie-Hellman.
}
\subparagraph{
Um dos algoritmos mais famosos da criptografia assim\'etrica \'e o \textit{RSA}(RIVEST et al, 1983) \cite{rivest}, algoritmo desenvolvido por Rivest, Shamir e Adleman. Este algoritmo est\'a presente em muitas aplica\c{c}\~oes de alta seguran\c{c}a, como bancos, sistemas militares e servidores de internet, e ele utiliza para a gera\c{c}\~ao de chaves dois n\'umeros primos de grandeza superior a $2^{512}$ multiplicados entre si.
}
\subparagraph{
Neste trabalho ser\'a feita a exposi\c{c}\~ao detalhada da chamada criptografia RSA cl\'assica, enfatizando a parte matem\'atica relacionada \`{a} Teoria dos n\'umeros, necess\'aria para a constru\c{c}\~ao do algoritmo.
}
\subparagraph{
O maior objetivo deste artigo \'e analisar a viabilidade de uma criptografia inspirada pelo algoritmo RSA cl\'assico, a qual substitui o s n\'umeros primos pelo conjunto denominado de \textit{primos de Gauss}, resultando, assim, no que chamamos por \textit{criptografia RSA gaussiana}. Para que tal algoritmo seja vi\'avel \'e necess\'ario se adaptar uma s\'erie de propiedades relativas aos n\'umeros primos aos n\'umero primos de Gauss. Dessa forma, nossa tarefa ser\' adaptar tanto quanto o poss\'ivel os primos de Gauss \`as dmosnta\c{c}\~oes desses teoremas.
}
\subparagraph{
Como se trata de uma proposta inovadora, deixamos para trabalhos futuros uma an\'alise comparativa entre as criptografias RSA cl\'assica e a RSA gaussiana.
}