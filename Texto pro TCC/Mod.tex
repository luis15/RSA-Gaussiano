\chapter {Opera\c{c}\~oes Modulares}
\label{Mod}

\section{Defini\c{c}\~ao de m\'odulo}	
\subparagraph{
J\'a lhe foi apresentado anteriormente o conceito da ciclicidade para a defini\c{c}\~ao de restos, neste cap\'itulo iremos nos aprofundar mais sobre esse conceito, estudando as propiedades necess\'arias da aritm\'etica modular para a elabora\c{c}\~ao da cripotgrafia RSA.
}
\subparagraph{
Um dos conceitos mais importantes da aritm\'etica modular \'e o de congru\^encia, representado pelo s\'imbolo $\equiv$. Talvez o exemplo mais comum de congru�ncia em nossas vidas sejam os dias da semana, embora o n\'umero do dia venha a variar ao longo do m\^es, sempre ap\'os {7} dias voltar\'a a ser domingo, por exemplo, logo a semana \'e uma congru\^encia de m\'odulo 7.
}
\subparagraph{
Para exemplificar vamos supor que primeiro domingo deste m\^es foi dia 4, e o \'ultimo ser\'a dia 25, logo temos que
}
\[	
	\begin{array}{c}
		\textit{$ 4 \equiv 25 (mod 7)$}
	\end{array}
\]
\subparagraph{
Fique claro que tornou $25$ congruente a $4$ no m\'odulo $7$ n�o foi o fato de ca\'irem no mesmo dia, isso \'e apenas um consequ\^encia, o que os torna congruentes \'e o fato de que divididos pelo m\'odulo, no caso $7$, eles apresentam o mesmo resto. Esse fato n�o se repete, por exemplo, se o m\'odulo for 5, neste caso $ 4 \equiv 4 (mod 5)$ e $ 25 \equiv 0 (mod 5)$.
}