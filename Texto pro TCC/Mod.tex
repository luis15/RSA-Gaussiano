\chapter {Opera\c{c}\~oes Modulares}
\label{Mod}
\subparagraph{
J\'a lhe foi apresentado anteriormente o conceito da ciclicidade para a defini\c{c}\~ao de restos, neste cap\'itulo iremos nos aprofundar mais sobre esse conceito, estudando as propiedades necess\'arias da aritm\'etica modular para a elabora\c{c}\~ao da cripotgrafia RSA.
}

\section{Defini\c{c}\~ao de m\'odulo}	
\subparagraph{
Um dos conceitos mais importantes da aritm\'etica modular \'e o de congru\^encia, representado pelo s\'imbolo $\equiv$. Talvez o exemplo mais comum de congru�ncia em nossas vidas sejam os dias da semana, embora o n\'umero do dia venha a variar ao longo do m\^es, sempre ap\'os {7} dias voltar\'a a ser domingo, por exemplo, logo a semana \'e uma congru\^encia de m\'odulo 7.
}
\subparagraph{
Para exemplificar vamos supor que primeiro domingo deste m\^es foi dia 4, e o \'ultimo ser\'a dia 25, logo temos que
}
\[	
	\begin{array}{c}
		\textit{$ 4 \equiv 25 (mod 7)$}
	\end{array}
\]
\subparagraph{
Fique claro que o que tornou $25$ congruente a $4$ no m\'odulo $7$ n�o foi o fato de ca\'irem no mesmo dia, isso \'e apenas um consequ\^encia, o que os torna congruentes \'e o fato de que divididos pelo m\'odulo, no caso $7$, eles apresentam o mesmo resto. Esse fato n�o se repete, por exemplo, se o m\'odulo for 5, neste caso $ 4 \equiv 4 (mod 5)$ e $ 25 \equiv 0 (mod 5)$.
}

\section{Propiedades das congru\^encias}	
\subparagraph{
Assim como as igualdades e desigualdades, as congru\^encias tamb\'em possuem uma listagem de propiedades em suas opera\c{c}\~oes. Ao longo desta se\c{c}\~ao lhe ser\~ao demonstradas essas propiedades. Fique atento pois as propiedades das congru\^encias nos facilitar\~ao a compreens\~ao de alguns conceitos importantes do algoritmo RSA mais a frente.
}
\subparagraph{
A primeira propiedade das congru\^encias, e a mais simples dela, \'e a \textit{reflexiva}, onde se diz que um n\'umero sempre \'e congruente a si mesmo. Para termos certeza vamos tomar um n\'umero qualquer $a$, sendo $a \equiv a (mod n)$, \'e equivalente dizermos que $a-a \equiv 0 (mod n)$. Por $0$ ser m\'ultiplo de qualquer n\'umero podemos confirmar que $a \equiv a (mod n)$.
}
\subparagraph{
A propiedade \textit{sim\'etrica} nos diz que se $a \equiv b (mod n)$, $b \equiv a (mod n)$. A afirma\c{c}~ao anterior ode ser escrito como se $a-b$ \'e m\'ultiplo de $n$, mas para isso deve ocorrer algum n�mero $k$ que equivala �: 
}
\[	
	\begin{array}{c}
		\textit{$a - b = k \cdot n $}
	\end{array}
\]
\paragraph{
Caso multipliquemos esta equa\c{c}\~ao por $-1$, vamos obter:
}
\[	
	\begin{array}{c}
		\textit{$b - a = (-k) \cdot n $}
	\end{array}
\]
\paragraph{
Que nos prova que $b-a$ \'e m\'ultiplo de $n$, logo $b \equiv a (mod n)$.
}
\subparagraph{
A terceira propiedade das congru\^encias \'e a \textit{transitiva}, onde se diz que se $a \equiv b (mod n)$ e $b \equiv c (mod n)$, $a \equiv c (mod n)$. Para prov�-la vamos observar as equa\c{c}\~oes
}
\[	
	\begin{array}{ccc}
		\textit{$ a - b = k \cdot n  $} & e & \textit{$ b - c = l \cdot m  $}
	\end{array}
\]
\paragraph{
Sabendo que $k$ e $l$ s\~ao inteiros escolhidas de forma adequada as equa\c{c}\~oes, podemos somar as equa\c{c}\~oes, resultando em:
}
\[	
	\begin{array}{c}
		\textit{$ (a - b) + (b - c) = k \cdot n + l \cdot m  $}
	\end{array}
\]
\paragraph{
Que pode ser simplificada em:
}
\[	
	\begin{array}{c}
		\textit{$ a - c = (k + l) \cdot m  $}
	\end{array}
\]
\paragraph{
Essa equa\c{c}\~ao equivale em valor a $a \equiv c (mod n)$, logo a propriedade transitiva \'e v\'alida.
}