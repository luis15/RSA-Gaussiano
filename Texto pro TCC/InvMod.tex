\chapter {Inversos Modulares}
\label{InvMod}

\section{Crit\'erios de multiplicidade e inversos modulares}	
\paragraph{
Nosso objetivo com o decorrer deste cap\'itulo \'e o de introduzir a opera\c{c}\~ao matem\'atica mais importante para para o algoritmo RSA. Para podermos compr\^ende-la antes precisamos ver com aten\c{c}\~aos quais s\~ao os crit\'erios para ser m\'utliplo de um determinado n\'umero.
}
\paragraph{
\'E ensinado desde o ensino fundamental uma s\'erie de t'ecnicas para acharmos os m\'ultiplos dos n\'umeros mais baixos. Por exemplo, para saber se o n\'umero \'e m\'ultiplo de $2$ checamos se ele \'e par. Um m\'ultiplo de $3$ tem a soma de seus algarismos igual a um m\'ultiplo de $3$. Para ser m\'ultiplo de $4$, seus dois \'ultimos algarismos devem ser m\'uliplos, j\'a um m\'ultiplo de $5$ sempre tem por \'ultimo algarismo $5$ ou $0$ e um m\'ultiplo de $6$ \'e alguem que seja m\'ultiplo de $2$ e $3$ ao mesmo tempo.
}
\paragraph{
Quando tentamos descobrir um m\'ultiplo de $7$ vemos que n\~ao conhecemos um m\'etodo t\~ao simples e pendemos para a for�a bruta.
}