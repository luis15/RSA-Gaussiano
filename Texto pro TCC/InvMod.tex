\chapter {Inversos Modulares}
\label{InvMod}

\section{Inversos modulares}	
\subparagraph{
Nosso objetivo com o decorrer deste cap\'itulo \'e o de explicar a opera\c{c}\~ao matem\'atica mais importante para para o algoritmo RSA. Para podermos comprend\^e-la vamos relembrar do cenceito ensinado no col\'egio de inverso multiplicativo, que consiste em obter o n\'umero que multiplicado a um n\'umero $n$ qualquer resulte em $1$. A opera\c{c}\~ao do inverso modular parte do mesmo princ\'ipio.
}
\subparagraph{
Vamos supor que queremos obter o inverso modular de $6$ para o m\'odulo $7$, o que n\'os teremos que fazer ent\~ao \'e encontrar qual o n\'umero que multiplicado por $6$ tem resto $1$ quando dividido por $7$. Come\c{c}amos pelo $1$, teremos que $6 \cdot 1 = 6$, $6 \equiv 6 (mod7)$. Com $2$ o resultado ser� $12$, logo $12 \equiv 5 (mod7)$, que para n\'os tamb\'em n\~ao serve. Tentando o $3$ obtemos $4$ e com $4$ obtemos $3$. Com o $5$ nosso retorno ser\'a $2$. Finalmente quando chegamos ao $6$ n\'os temos que $6 \cdot 6 = 36$, $36 \equiv 1 (mod 7)$. Com isso podemos concluir que o inverso multiplicativo de $6$ no m\'odulo $7$ \'e o pr\'opio $6$.
}
\subparagraph{
Para simplificar o que foi dito acima, podemos dizer a opera\c{c}\~ao de inverso multiplicativo no m\'odulo $n$ para $a$ consiste em encontar um n\'umero $a'$ tal que:
}
\[	
	\begin{array}{c}
		\textit{$a \cdot a' \equiv 1 (mod n)$}
	\end{array}
\]

\section{Inexist\^encia e exist\^encia de inversos}	

\subparagraph{
Antes de come\c{c}armos vamos tentar calcular o inverso multiplicativo de $2$ no m\'odulo $8$, vamos l\'a: 
}
\[	
	\begin{array}{c}
		\textit{$2 \cdot 0 \equiv 0 \not\equiv 1(mod 8)$}\\
		\textit{$2 \cdot 1 \equiv 2 \not\equiv 1(mod 8)$}\\
		\textit{$2 \cdot 2 \equiv 4 \not\equiv 1(mod 8)$}\\
		\textit{$2 \cdot 3 \equiv 6 \not\equiv 1(mod 8)$}\\
		\textit{$2 \cdot 4 \equiv 8 \not\equiv 1(mod 8)$}\\
		\textit{$2 \cdot 5 \equiv 0 \not\equiv 1(mod 8)$}\\
		\textit{$2 \cdot 6 \equiv 2 \not\equiv 1(mod 8)$}\\
		\textit{$2 \cdot 7 \equiv 4 \not\equiv 1(mod 8)$}\\
	\end{array}
\]

\subparagraph{
N\~ao encontramos nenhuma resposta pois, simplesmente, n\~ao h\'a. Antes que se pergunte o motivo de n\~ao tentarmos com n\'umeros maiores que $7$, \'e v\'alido lembrar que a partir do $8$ ter\'iamos a repeti\c{c}\~ao de resultados por conta das congru\^encias.
}
\subparagraph{
A opera\c{c}\~ao de inverso multiplicativo s\'o possui resultado em casos onde o n\'umero $a$ ao qual queremos calcular o inverso e o m'odulo s\~ao \textit{primos entre si}, ou seja, n\~ao possuam nenhum fator em comum. Por conta disso usamos os n\'umeros primos no algoritmo RSA.
}
\subparagraph{
Para comprovar o que foi dito acima, vamos tomar um n\'umero $a$, tal que
}
\[	
	\begin{array}{c}
		\textit{$a \cdot a' \equiv 1(mod n)$}\\
	\end{array}
\]
\paragraph{
isso pode ser traduzido em linguagem humana como $n$ divide $ a \cdot a' - 1$. Isso em linguajar matem\'atico pode ser escrito como:
}
\[	
	\begin{array}{c}
		\textit{$a \cdot a' - 1 = n \cdot k$}\\
	\end{array}
\]
\paragraph{
como estamos atr\'as de saber se $a$ e $n$ n\~ao possuem fator comum, ent�o h\'a de haver um $k$ inteiro para a equa\c{c}\~ao acima. Nosso primeiro passo para provar isso ser\'a de se criar o conjunto $V(a,n)$, esse conjunto \'e formado por inteiros positivos e pode ser escrito como
}
\[	
	\begin{array}{c}
		\textit{$x \cdot a + y \cdot n$}\\
	\end{array}
\]
\subparagraph{
Em um primeiro momento este conjunto e esta nova f\'ormula podem parecer estranhos ao que se via antes, mas 
}