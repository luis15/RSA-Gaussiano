\chapter {Inversos Modulares}
\label{InvMod}

\section{Inversos modulares}	
\subparagraph{
Nosso objetivo com o decorrer deste cap\'itulo \'e o de explicar a opera\c{c}\~ao matem\'atica mais importante para para o algoritmo RSA. Para podermos comprend\^e-la vamos relembrar do cenceito ensinado no col\'egio de inverso multiplicativo, que consiste em obter o n\'umero que multiplicado a um n\'umero $n$ qualquer resulte em $1$. A opera\c{c}\~ao do inverso modular parte do mesmo princ\'ipio.
}
\subparagraph{
Vamos supor que queremos obter o inverso modular de $6$ para o m\'odulo $7$, o que n\'os teremos que fazer ent\~ao \'e encontrar qual o n\'umero que multiplicado por $6$ tem resto $1$ quando dividido por $7$. Come\c{c}amos pelo $1$, teremos que $6 \cdot 1 = 6$, $6 \equiv 6 (mod7)$. Com $2$ o resultado ser� $12$, logo $12 \equiv 5 (mod7)$, que para n\'os tamb\'em n\~ao serve. Tentando o $3$ obtemos $4$ e com $4$ obtemos $3$. Com o $5$ nosso retorno ser\'a $2$. Finalmente quando chegamos ao $6$ n\'os temos que $6 \cdot 6 = 36$, $36 \equiv 1 (mod 7)$. Com isso podemos concluir que o inverso multiplicativo de $6$ no m\'odulo $7$ \'e o pr\'opio $6$.
}
\subparagraph{
Para simplificar o que foi dito acima, podemos dizer a opera\c{c}\~ao de inverso multiplicativo no m\'odulo $n$ para $a$ consiste em encontar um n\'umero $a'$ tal que:
}
\[	
	\begin{array}{c}
		\textit{$a \cdot a' \equiv 1 (mod n)$}
	\end{array}
\]
