\chapter {Inteiros e Primos de Gauss}
\label{IG}
At\'e o momento apenas os n\'umeros inteiros foram abordados neste projeto, mas para podermos entender a RSA Gaussiana \'e necess\'ario conhecer o conjunto dos n\'umeros inteiros gaussianos. Ao longo deste cap\'itulo vamos conhecer os inteiros e os primos gaussianos e suas propiedades aritm\'eticas b\'asicas. 

\section{Inteiros de Gauss e suas propiedades}

Os inteiros gaussianos, conjunto que a partir de agora iremos nos referenciar por $\mathbb{Z}[i]$, s\~ao um subconjuntos dos n\'umeros complexos, relembrando que os n\'umeros complexos s\~ao os n\'umeros de forma $a+b\textbf{i}$, onde $a$ e $b$ s\~ao reais e $\textbf{i}$ \'e a $\sqrt{-1}$. A diferen\c{c}a entre o conjunto $\mathbb{Z}[i]$ e o conjunto $C$ reside no fato de em $\mathbb{Z}[i]$ $a$ e $b$ serem n\'umeros inteiros. Formalmente dizemos que os inteiros gaussianos s\~ao:

$$\mathbb{Z}[i]= \left\{a+b\textbf{i} | a,b \in \mathbb{Z}  \right\}, \textrm{ onde } \textbf{i}^2 = -1$$

Por $\mathbb{Z}[i]$ estar contido em $\mathbb{C}$, as opera\c{c}\~oes deste conjunto podem ser realizadas, por exemplo, se tomarmos $z_1= a + b\textbf{i}$ e $z_2= c + d\textbf{i}$ n\'os iremos obter:

$$z_1   +   z_2 = (a + c) + (b + d)\textbf{i}$$
$$z_1 \cdot z_2 = (ac - bd) + (ad + bc)\textbf{i}$$

Outra propiedade herdada \'e a dos elementos neutros, o $0 = 0 + 0\textbf{i}$ continua sendo o elemento neutro da adi\c{c}\~ao, enquanto o $1 = 1 + 0\textbf{i}$ tamb\'em continua sendo o elemento neutro da multiplica\c{c}\~ao. As propiedades associativa da adi\c{c}\~ao e da multiplica\c{c}\~ao, comutativa da adi\c{c}\~ao e multiplica\c{c}\~ao e distributiva tamb\'em s\~ao herdadas do conjunto complexo.

Repare que se considermos o plano complexo, os inteiros gaussianos ter\~ao uma marca\c{c}\~ao reticulada. Outro conceito importante para os inteiros gaussianos \'e a norma do n\'umero, ela \'e importante para auxiliar na defini\c{c}\~ao de um primo gaussiano, assim como s\~ao importantes os conceitos de n\'umero conjugado e n\'umero associado. Caso venhamos a tomar um n\'umero inteiro gaussiano de forma $a+b\textbf{i}$, sua norma ser\'a $a^2 +b^2$.

\begin{Df}
A norma de um n\'umero gaussiano \'e a soma dos quadrados de seus valores absolutos como n\'umero complexo. Ela \'e o resultado de:

$$N(a+b\textbf{i}) = a^2 + b^2 = (a+b\textbf{i})(a-b\textbf{i}),$$

onde o $(a-b\textbf{i})$ \'e a conjugado de $(a+b\textbf{i})$, tamb\'em denotado por $\overline{(a+b\textbf{i})}$.
\end{Df}

Uma das propiedades da norma \'e ser multiplicativa, ou seja, a norma de $N(zw)$ \'e igual a $N(z) \cdot N(w)$.

Os inteiros gaussianos possuem como unidades b\'asicas $\pm 1$ e $\pm \textbf{i}$. Caso venhamos a multiplicar um inteiro gaussiano x, teremos que $\pm x$ e $\pm x\textbf{i}$ sendo seus elementos associados.

\begin{Df}

Os elementos associados de um n\'umero $x$, tal que $x \in \mathbb{Z}[i]$, s\~ao $\pm x$ e $\pm x\textbf{i}$.

\end{Df}

Para podermos prosseguir at\'e chegarmos aos primos precisaremos demonstrar para o conjunto $\mathbb{Z}[i]$ uma s\'erie de resultados que j\'a \'e conhecida do conjunto dos n\'umeros inteiros, como o funcionamento da divis\~ao e o teorema da fatora\c{c}\~ao \'unica.

Podemos definir a divisibilidade gaussiana por quando dizemos que $\beta$ divide $\alfa$, representado por $\beta | \alpha$ se $\alpha = \beta \gamma$, para qualquer $\gamma \in \mathbb{Z}[i] $. Nesse caso, $\beta$ \'e um fator de $\alpha$.

\begin{Th}\label{div_gaussiana1}

Um inteiro Gaussiano $\alpha = a+b\textbf{i}$ \'e dividido por um primo $c$ se e somente se $c|a$ e $c|b$ em $\mathbb{Z}$.

\end{Th}

\noindent{\textbf{\textit{Demonstra\c{c}\~ao}}}\\

Dizer que $c|(a+b\textbf{i})$ em $\mathbb{Z}$ \'e o mesmo que dizer que $a+b\textbf{i} = c(m +  n\textbf{i})$, para algum $m, n \in \mathbb{Z}$, que equivale a $a=cm$ e $b=cn$.

\hfill\newline

Tomemos uma divis\~ao entre inteiros gaussianos, onde $\alpha$ \'e o dividendo, $\beta$ o divisor, $\gamma$ o quociente e $\rho$ o dividendo

\begin{Th}[Teorema da divis\~ao conjunto gaussiano  ]  \label{divgauss}

Para $\alpha, \beta \in \mathbb{Z}$ com $\beta \neq 0$ existe um $\gama, \rho \in \mathbb{Z}$ tal qual $\alpha = \beta \gamma + \rho$ e $N(\rho) < N(\beta)$. De fato, podemos escolher $\rho$  de forma que $N(\rho) \leq (1/2)N(\beta)$

\end{Th}

Agora que j\'a entendemos a divis\~ao, vamos definir o m\'aximo divisor comum no conjunto $\mathbb{Z}[i]$.

\begin{Th}[Algoritmo Euclidiano no conjunto gaussiano ]
\label{euclideszi}

Tomemos $\alpha , \beta \in \mathbb{Z}[i]$ e diferentes de $0$. Aplicamos recursivamente o teorema da divis\~ao em $\mathbb{Z}[i]$ (\ref{divgauss}), come\c{c}ando com esse par e fazendo com o resto uma  equa\c{c}\~ao com um novo dividendo e divisor no pr\'oximo caso, enquanto o resto for diferente de zero:

\[
\begin{array}{lcll}
\alpha & = & \beta \gama_1 + \rho_1,  & N(\rho_1) < N(\beta)  \\
\beta  & = & \rho_1 \gama_2 + \rho_2, & N(\rho_2) < N(\rho_1) \\
\rho_1 & = & \rho_2 \gama_3 + \rho_3, & N(\rho_3) < N(\rho_2) \\
& \vdots &  &\\
\end{array}
\]

O \'ultimo elemento que n\~ao possua resto $0$ \'e divis\'ivel por todos os divisores comuns de $\alpha$ e $\beta$, sendo esse o maior divisor comum de $\alpha$ e $\beta$.

\end{Th}

Podemos dizer que se $\alpha$ e $\beta$ possuem apenas as unidades como fatores em comum eles s\~ao primos entre si. 


Os elementos primos do conjunto $\mathbb{Z}[i]$ tamb\'em denomidados primos gaussianos. Eles s\~ao sim\'etricos tanto no eixo real quanto no imagin\'ario. 

\begin{Df}

Um inteiro gaussiano \'e um primo se e somente se:

\begin{enumerate}
	
	\item um dos elementos, $a$ ou $b$, \'e igual a $0$ e o outro elemento \'e um primo de forma $(4n + 3)$ ou $-(4n + 3)$, com $n$ sendo um n\'umero inteiro positivo
	
	\item ambos os n\'umeros s\~ao diferentes de $0$ e $(a^2 + b^2)$ \'e um n\'umero primo
	
\end{enumerate}

\end{Df}

Para resurmimos, podemos dizer que um primo gaussiano \'e um inteiro gaussiano cuja a norma seja um n\'umero primo ou os elemetos associados de primo de forma $4n + 3$. H\'a tr\^es casos em que um n\'umero primo $p$ pode ser fatorado no conjunto gaussiano:

\begin{enumerate}
	
	\item Se $p$ \'e congruente a $3$ no m\'odulo $4$ ele \'e primo gaussiano
	
	\item Se $p$ \'e congruente a $1$ no m\'odulo $4$ ele \'e fatorado pela multiplica\c{c}\~ao de um primo gaussiano por seu conjugado. Por exemplo: $5 = (2 + \textbf{i})(2 - \textbf{i})$ e $13 = (3+2\textbf{i})(3-2\textbf{i})$
	
	\item Se $p$ \'e igual a $2$, temos que $2 = \textbf{i}{(1-i)}^2$ 
	
\end{enumerate}