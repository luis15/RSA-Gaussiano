\chapter {Criptografia RSA}
\label{RSA}

\subparagraph{
A criptografia RSA tem suma import\^ancia para toda a comunica\c{c}\~ao moderna. Ela \'e t\~ao importante que a descoberta de uma forma de se desencript\'a-la colocaria em risco a sociedade como a conhecemos.
}
\subparagraph{
Ao longo deste cap\'itulo n\'os vamos ver como \'e seu funcionamento, unindo todos os conte\'udos j\'a vistos anteriormente. Al\'em disso
}

\section{Pr\'e-requisitos}
\subparagraph{
Para que o algoritmo RSA possa encriptar de forma eficiente, primeiro n\'os precisaremos seguir uma s\'erie de passos. Estes s\~ao necess\'arios para que o RSA funcione, mas ainda n\~ao s\~ao parte do algoritmo.
}
\subparagraph{
O primeiro passo \'e a convers\~ao das letras da mensagem em n\'umeros. A essa etapa n\'os chamaremos de pr\'e-codifica\c{c}\~ao. Para que o RSA venha a funcionar precisamos seguir uma tabela como a abaixo:
}
\[
\begin{array}{ccccccccccccc}
A & B & C & D & E & F & G & H & I & J  &  K  & L  & M  \\ 
10 & 11 & 12 & 13 & 14 & 15 & 16 & 17 & 18 & 19 &  20 & 21 & 22 \\ 
\\
N & O  & P  & Q  & R  & S & T  & U  & V  & X  & Y  & W  & Z \\
23 & 24 & 25 & 26 & 27 & 28 & 29 & 30 & 31 & 32 & 33 & 34 & 35 \\
\end{array}
\]
\subparagraph{
Para o espa\c{c}o vamos usar o 99. Avisamos que esta \'e uma tabela apenas com fim did\'atico, e, por isso h\'a v\'arios caracteres desconsiderados. Como exemplo vamos pr\'e-encriptar o poema Amor, de Oswald de Andrade. O texto do poema a ser encriptado � o  pr\'e-seguinte:
}
\[
\begin{array}{ccccccccccccc}
Amor  \\ 
Humor. \\ 
\end{array}
\]
\subparagraph{
Como primeiro passo substituiremos cada letra da poesia por um n\'umero correspondente. Feito isso juntaremos todos os n\'umeros Ao terminarmos vamos obt\^e-lo convertido como: 
}
\[
\begin{array}{c}
 10222427991730222427
\end{array}
\]
\subparagraph{
Preste aten\c{c}\~ao ao fato de todo o caracter convertido ter sempre o mesmo n\'umero de algarismos. Isso \'e \'util para evitar ambiguidades na desencripta\c{c}\~ao.
}
\subparagraph{
Nosso pr'oximo passo nesta fase que antecede a encripta\c{c}\~ao consiste em definir que s\~ao os primos $p$ e $q$. Para nosso exemplo vamos usar $p=17$ e $q=23$, como $n = pq$. Temos que $n=391$
}
\subparagraph{
O \'ultimo passo da pr\'e-encripta\c{c}\~ao consiste em quebrar o n\'umero qu obtemos acima em blocos menores. Essas blocos devem obedecer a duas regras b\'asicas, devem ser menores que $n$, ou no nosso exemplo $391$ e n\~ao podem se iniciar por 0, Vejamos como fica a nossa mensagem escrita nesta forma.
}
\[
\begin{array}{c}
 102 | 224 | 279 | 91 | 7 | 30 | 222 | 42 | 7
\end{array}
\]
\subparagraph{
Perceba que n\~ao h\'a rela\c{c}\~ao entre nenhum dos n\'umeros obtidos com um caractere espec\'ifico, o que torna imposs\'ivel a associa\c{c}\~ao de um n\'umero a uma letra por frequ\^encia de aparecimento. 
}
\section{Codificando e decodificando mensagens}
\subparagraph{
Encerrada a fase de pr\'e-codificando vamos agora codificar nossas mensagens. Manteremos os valores e exemplos da se��o anterior a fim de facilitar a compreens\~ao.
}
\subsection{Codificando uma mensagem}
\subparagraph{
A esta altura n\'os j\'a conhecemos a $n$. Vamos chamar o bloco codificado que iremos encriptar de $b$, lembrando que $b$ \'e um n\'umero inteiro menor que $n$. Vamos chamao o bloco coficado de C(b). Para obt\^e-lo deveos aplicar a seguinte f\'ormula:
}
\[
\begin{array}{c}
$C(b)$ = resto da divis\~ao de $b^3$ por $n$
\end{array}
\]
\subparagraph{
Podemos para facilitar dizer que $C(b)$ � o res�duo de $b^3$ pelo m\'odulo $n$. Vamo a uma demonstra\c{c}\~ao pr\'atica com o primiro bloco de nossa mensagem, com o n\'umero $102$. Para simplificar o nosso trabalho vamos utilizar as opera\c{c}\~oes modulares.
}
\[
\begin{array}{c}
$102^3 \equiv 24276 \equiv 34 (mod 391) $
\end{array}
\]
\subparagraph{
Faremos o mesmo procedimento para todos nossos blocos
}
\[
\begin{array}{c}
$224^3 \equiv 11239424 \equiv 129 (mod 391) $
$279^3 \equiv 21717639 \equiv 326 (mod 391) $
$91^3  \equiv 753571   \equiv 114 (mod 391) $
$7^3   \equiv 343      \equiv 343 (mod 391) $
$30^3  \equiv 27000    \equiv 21  (mod 391) $
$222^3 \equiv 10941048 \equiv 86  (mod 391) $
$42^3  \equiv 74088    \equiv 189 (mod 391) $
$7^3   \equiv 343      \equiv 343 (mod 391) $
\end{array}
\]
\subparagraph{
Portanto, ``Amor/Humor.'', encriptado pelo RSA com as chaves privadas $17$ e $23$ e a p\'ublica $391$ \'e: 
}
\[
\begin{array}{c}
 34 | 129 | 326 | 114 | 343 | 21 | 86 | 189 | 343
\end{array}
\]

\subsection{Decodificando uma mensagem}
\subparagraph{
Para podermos desencriptar uma mensagem n\'os precisamos de dois n�meros. O primeiro \'e a nossa chave p\'ublica $n$. O segundo n\'umero \'e $d$, o $d$ � o inverso de $3$ para o m\'odulo $(p-1)(q-1)$. Mais a frente explicaremos o motivo de se adicionar este n\'umero. Pela defini\c{c}\~ao do inverso temos:
}
\[
\begin{array}{c}
$ 3d \equiv 1(mod((p-1)(q-1)))$
\end{array}
\]
