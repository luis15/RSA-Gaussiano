\chapter {Criptografia RSA}
\label{RSA}

\subparagraph{
A criptografia RSA tem suma import\^ancia para toda a comunica\c{c}\~ao moderna. Ela \'e t\~ao importante que a descoberta de uma forma de se desencript\'a-la colocaria em risco a sociedade como a conhecemos.
}
\subparagraph{
Ao longo deste cap\'itulo n\'os vamos ver como \'e seu funcionamento, unindo todos os conte\'udos j\'a vistos anteriormente. Al\'em disso
}

\section{Pr\'e-requisitos}
\subparagraph{
Para que o algoritmo RSA possa encriptar de forma eficiente, primeiro n\'os precisaremos seguir uma s\'erie de passos. Estes s\~ao necess\'arios para que o RSA funcione, mas ainda n\~ao s\~ao parte do algoritmo.
}
\subparagraph{
O primeiro passo \'e a convers\~ao das letras da mensagem em n\'umeros. A essa etapa n\'os chamaremos de pr\'e-codifica\c{c}\~ao. Para que o RSA venha a funcionar precisamos seguir uma tabela como a abaixo:
}
\[
\begin{array}{ccccccccccccc}
A & B & C & D & E & F & G & H & I & J  &  K  & L  & M  \\ 
10 & 11 & 12 & 13 & 14 & 15 & 16 & 17 & 18 & 19 &  20 & 21 & 22 \\ 
\\
N & O  & P  & Q  & R  & S & T  & U  & V  & X  & Y  & W  & Z \\
23 & 24 & 25 & 26 & 27 & 28 & 29 & 30 & 31 & 32 & 33 & 34 & 35 \\
\end{array}
\]
\subparagraph{
Para o espa\c{c}o vamos usar o 99. Avisamos que esta \'e uma tabela apenas com fim did\'atico, e, por isso h\'a v\'arios caracteres desconsiderados. Como exemplo vamos pr\'e-encriptar o poema Amor, de Oswald de Andrade. O texto do poema a ser encriptado � o  pr\'e-seguinte:
}
\[
\begin{array}{ccccccccccccc}
Amor  \\ 
Humor. \\ 
\end{array}
\]
\subparagraph{
Como primeiro passo substituiremos cada letra da poesia por um n\'umero correspondente. Feito isso juntaremos todos os n\'umeros Ao terminarmos vamos obt\^e-lo convertido como: 
}
\[
\begin{array}{c}
 10222427991730222427
\end{array}
\]
\subparagraph{
Preste aten\c{c}\~ao ao fato de todo o caracter convertido ter sempre o mesmo n\'umero de algarismos. Isso \'e \'util para evitar ambiguidades na desencripta\c{c}\~ao.
}
\subparagraph{
Nosso pr'oximo passo nesta fase que antecede a encripta\c{c}\~ao consiste em definir que s\~ao os primos $p$ e $q$. Para nosso exemplo vamos usar $p=17$ e $q=23$, como $n = pq$. Temos que $n=391$
}
\subparagraph{
O \'ultimo passo da pr\'e-encripta\c{c}\~ao consiste em quebrar o n\'umero qu obtemos acima em blocos menores. Essas blocos devem obedecer a duas regras b\'asicas, devem ser menores que $n$, ou no nosso exemplo $391$ e n\~ao podem se iniciar por 0, Vejamos como fica a nossa mensagem escrita nesta forma.
}
\[
\begin{array}{c}
 102 | 224 | 279 | 91 | 7 | 30 | 222 | 42 | 7
\end{array}
\]
\subparagraph{
Perceba que n\~ao h\'a rela\c{c}\~ao entre nenhum dos n\'umeros obtidos com um caractere espec\'ifico, o que torna imposs\'ivel a associa\c{c}\~ao de um n\'umero a uma letra por frequ\^encia de aparecimento. 
}
\section{Codificando e decodificando mensagens}
\subparagraph{
Encerrada a fase de pr\'e-codificando vamos agora codificar nossas mensagens. Manteremos os valores e exemplos da se��o anterior a fim de facilitar a compreens\~ao.
}
\subsection{Codificando uma mensagem}
\subparagraph{
A esta altura n\'os j\'a conhecemos a $n$. Vamos chamar o bloco codificado que iremos encriptar de $b$, lembrando que $b$ \'e um n\'umero inteiro menor que $n$. Vamos chamar o bloco codificado de C(b). Para obt\^e-lo deveos aplicar a seguinte f\'ormula:
}
\[
\begin{array}{c}
$C(b) \equiv b^3 (mod n)$
\end{array}
\]
\subparagraph{
Podemos para facilitar dizer que $C(b)$ � o res�duo de $b^3$ pelo m\'odulo $n$. Vamo a uma demonstra\c{c}\~ao pr\'atica com o primiro bloco de nossa mensagem, com o n\'umero $102$. Para simplificar o nosso trabalho vamos utilizar as opera\c{c}\~oes modulares.
}
\[
\begin{array}{c}
$102^3 \equiv 24276 \equiv 34 (mod 391) $
\end{array}
\]
\subparagraph{
Faremos o mesmo procedimento para todos nossos blocos
}
\[
\begin{array}{c}
$224^3 \equiv 11239424 \equiv 129 (mod 391) $\\
$279^3 \equiv 21717639 \equiv 326 (mod 391) $\\
$91^3  \equiv 753571   \equiv 114 (mod 391) $\\
$7^3   \equiv 343      \equiv 343 (mod 391) $\\
$30^3  \equiv 27000    \equiv 21  (mod 391) $\\
$222^3 \equiv 10941048 \equiv 86  (mod 391) $\\
$42^3  \equiv 74088    \equiv 189 (mod 391) $\\
$7^3   \equiv 343      \equiv 343 (mod 391) $\\
\end{array}
\]
\subparagraph{
Portanto, ``Amor/Humor.'', encriptado pelo RSA com as chaves privadas $17$ e $23$ e a p\'ublica $391$ \'e: 
}
\[
\begin{array}{c}
 34 | 129 | 326 | 114 | 343 | 21 | 86 | 189 | 343
\end{array}
\]

\subsection{Decodificando uma mensagem}
\subparagraph{
Para podermos desencriptar uma mensagem n\'os precisamos de dois n�meros. O primeiro \'e a nossa chave p\'ublica $n$. O segundo n\'umero \'e $d$, o $d$ � o inverso de $3$ para o m\'odulo $(p-1)(q-1)$. Mais a frente explicaremos o motivo de se adicionar este n\'umero. Pela defini\c{c}\~ao do inverso temos:
}
\[
\begin{array}{c}
$ 3d \equiv 1(mod((p-1)(q-1)))$
\end{array}
\]
\subparagraph{
A partir do momento em que estamos de posse de $d$, podemos usar o par $(n,d)$ para desencriptar a mensagem, onde $a$ \'e o bloco encriptado e $D(a)$ a mensagem desencriptada, usando a f\'ormula:
}
\[
\begin{array}{c}
$ D(a) \equiv a^d (mod n) $
\end{array}
\]
\subparagraph{
Note que na fun\c{c}\~ao acima n\'os assumimos o compromisso de que $D(C(b)) = b$. Para comprov\'a-la vamos mais adiante fazer sua demonstra\c{c}\~ao. Neste momento vamos apenas aplic\'a-la ao nosso exemplo. O primeiro passo consiste em calcular $d$. Vamos supor que $p$ e $q$ deixam resto $5$ na divis\~ao por $6$. Com isso podemos afirmar que:
}
\[
\begin{array}{c}
$(p-1)(q-1) \equiv 4 \cdot 4 \equiv 16 \equiv 4 \equiv -2(mod 6)$ 
\end{array}
\]
\paragraph{
ou seja:
}
\[
\begin{array}{c}
$(p-1)(q-1) = 6 \cdot k -2$
\end{array}
\]
\subparagraph{
No entanto, podemos dizer que $6 \cdot k - 2 \equiv 4 \cdot k - 1 (mod 3)$.Podendo dizer que assim que $d$ \'e igual a $4 \cdot k -1$. Feito isso vamos aos n\'umeros de nosso exemplo: $17$ e $23$, com isso vamos obter:
}
\[
\begin{array}{c}
$(p-1)(q-1) = 16 \cdot 22 = 352 = 6 \cdot 58 + 4 = 6\cdot 59 -2$
\end{array}
\]
\paragraph{
Com isso obtemos que $k=59$. Aplicando k teremos que:
}
\[
\begin{array}{c}
$d = 4 \cdot 59 - 1 = 235.$
\end{array}
\]
\subparagraph{
Agora que j\'a conhecemos a $d$ podemos decodificar a mensagem. Vamos fazer isso em nosso primeiro bloco codificado, que possui o valor $34$. Para achar a resposta precisaremos calcular $D(34) \equiv 34^{235} (mod 391)$. Esse c\'alculo seria praticamente imposs\'ivel sem o uso dos Teoremas chin\^es do resto e de Fermat.
}
\subparagraph{
Nosso primeiro passo ser\'a o de calcular $34^{235}$ nos m\'odulos $17$ e $23$, que s\~ao os primos em que $n$ fatora. Neste caso, come\c{c}amos com:
}
\[
\begin{array}{c}
$34 \equiv 0 (mod 17)$\\
$34 \equiv 11(mod 23)$\\
\end{array}
\]
\subparagraph{
Assim teremos que $34^{235} \equiv 0^{235} \equiv 0 (mod 17)$. Aplicando o teorema de Fermat a outra congr\^encia teremos:
}
\[
\begin{array}{c}
$11^{235} \equiv (11^{22})^{10} \cdot 11^{15} \equiv 11^{15}(mod 23)$
\end{array}
\]
\subparagraph{
Como $ 11 \equiv -12 \equiv -4 \cdot 3(mod 23)$. Desse modo podemos afirmar que:
}
\[
\begin{array}{c}
$11^{235} \equiv 11^{15} \equiv -4^{15} \cdot 3^{15}(mod 23)$
\end{array}
\]
\paragraph{
Com isso, teremos:
}
\[
\begin{array}{c}
$415 \equiv 230 \equiv (2^{11})^2 \cdot 2^8 \equiv 2^8 \equiv 3 (mod 23)$\\
$315 \equiv 3^{11} \cdot 3^4 \equiv 3^4 \equiv 12 (mod 23)$
\end{array}
\]
\paragraph{
Concluindo assim:
}
\[
\begin{array}{c}
$11235 \equiv -415 \cdot 315 \equiv -3 \cdot 12 \equiv 10 (mod 23)$
\end{array}
\]
\subparagraph{
Temos assim as congru\^encias $34^{235} \equiv 0 (mod 17)$ e $34^{235} \equiv 10 (mod 23)$. Com isso podemoa aplicaro o teorema chin\^es do resto no sistema:
}
\[
\begin{array}{c}
$x \equiv 0 (mod 17)$\\
$x \equiv 10 (mod 23)$\\
\end{array}
\]
\subparagraph{
Com ele n\'os iremos obter que 
}
\[
\begin{array}{c}
$10 + 23y \equiv 0 (mod 17)$\\
\end{array}
\]
\paragraph{
Obtendo assim:
}
\[
\begin{array}{c}
$6y \equiv 7 (mod 17)$\\
\end{array}
\]
\paragraph{
Por�m, $3$ \'e o inverso de $6$ no m\'odulo $17$, e por isso teremos
}
\[
\begin{array}{c}
$y \equiv 3 \cdot 7 \equiv 4 (mod 17)$\\
\end{array}
\]
\subparagraph{
Com isso iremos chegar at\'e $x = 10 + 23y = 10 + 23 \codt 4 = 102$. Caso voc� venha a conferir na se\c{c}\~ao sobre codifica\c{c}\~ao de mensagens poder\'a confirmar o resultado. Os demais blocos podem ser decodificados da mesma forma, apenas n\~ao ser\~ao mostradas nesta obra por necessitar de muitos passos, o que tornaria este cap\'itulo inutilmente mais longo.
}


\section{Provando a funcionalidade do RSA}
\subparagraph{
Ao longo dessa se\c{c}\~ao vamos provar que o RSA funciona no processo de decodifica\c{c}\~ao, para podermos fazer isso tudo o que teremos que fazer \'e provar que:
}
\[
\begin{array}{c}
$b \equiv DC(b) (mod n) $
\end{array}
\]
\subparagraph{
N\'os j\' vimos que $C(b) \equiv b^3 (mod n)$ e $D(a) \equiv a^d(mod n)$. Se combinarmos ambos as congru\^encia iremos obter
}
\[
\begin{array}{c}
$DC(b) \equiv D(b^3) \equiv b^{3d}(mod n) $
\end{array}
\]
\subparagraph{
Logo o que n\'os temos de provar \'e que $b^{3d} \equiv b (mod n)$. Como por defini\c{c}\~ao $3d \equiv 1 (mod(p-1)(q-1))$, o que nos leva at\'e:
}
\[
\begin{array}{c}
$3d = 1+k(p-1)(q-1) $
\end{array}
\]





\section{Discutindo a seguran\c{c}a do RSA}
\subparagraph{
Antes de mudarmos completamente o foco deste trabalho, vaos nos focar no que tange a seguran\c{c}a do RSA. Vamos supor que algu\'em, que vamos denominar Irineu, esteja com uma escuta em nossas mensagens, tendo assim acesso tanto a mensagem codificada quanto a chave p\'ublica $n$. Vamos lembrar que $n$ \'e a multiplica\c{c}\~ ao dos primos $p$ e $q$. Sabendo disso bastaria para Irineu fatorar $n$ para obter $p$ e $q$ e depois descobrir $d$ para poder decodificar a mensagem, como j\'a foi explicado neste cap\'itulo.
}
\subparagraph{
Isto pode parecer muito simples, mas cmo j\'a mostramos na se��o sobre fatora\c{c}\~ao n\~ao h\'a um algoritmo conhecido que possa fazer isso de forma eficiente. O que ocorre \'e que um algoritmo que fa\c{c}a a fatora\c{c}\~ao de forma eficinte pode vir a surgir a qualquer momento, do ponto que n\~ao h\'a nenhuma prova matem\'atica de que esse algoritmo n\~ao exista.
}
\subparagraph{
O que iremos fazer nos pr\'oximos cap\'itulos consiste em propor uma varia\c{c}\~ao da criptografia RSA que n\~ao seja completamente vulner\'avel caso tal algoritmo sejja descoberto, a \textit{Criptografia RSA Gaussiana}.
}
