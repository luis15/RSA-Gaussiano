
%%%%%%%%%%%%%%%%%%%%%%%%%%%%%%%%%% DISSERTA��O OU TESE %%%%%%%%%%%%%%%%%%%%%%%%%%%%%%%%%%%%%%%%%
\documentclass[11pt]{report}
\usepackage{graphicx}
\usepackage[brazil]{babel}
\usepackage[latin1]{inputenc}
\usepackage{babelbib} % Esse pacote gera a bibliografia em Portugu�s. Acho que deve ser instalado algum pacote, n�o me recordo.
\usepackage{latexsym}
\usepackage{amsfonts}
\usepackage{amsthm}
\usepackage{amssymb}
\usepackage{amsfonts}
\usepackage{xypic}
\usepackage{ulem}
\usepackage[all]{xy}

\usepackage{makeidx} %cria o �ndice remissivo
\makeindex

\vfuzz2pt % Don't report over-full v-boxes if over-edge is small
\hfuzz2pt % Don't report over-full h-boxes if over-edge is small

% TEOREMAS -------------------------------------------------------
\theoremstyle{alpha}
\newtheorem{Th}{Teorema}[section]
\newtheorem{Cor}[Th]{Corol�rio}
\newtheorem{Le}[Th]{Lema}
\newtheorem{Pro}[Th]{Proposi��o}
\theoremstyle{definition}
\newtheorem{Df}[Th]{Defini��o}
\newtheorem{Obs}[Th]{Observa��o}
\newtheorem{Dig}[Th]{Digress�o}
\newtheorem{Ex}[Th]{Exemplo}
\newtheorem{Esc}[Th]{Esc�lio}
\newtheorem{Ass}[Th]{Asser��o}

% MATEMATICA E LOGICA -------------------------------------------

\newcommand{\lan}     {\langle}
\newcommand{\ran}     {\rangle}
\newcommand{\IN}      {\mathbb{N}}
\newcommand{\A}       {\mathcal{A}}
\newcommand{\B}       {\mathcal{B}}
\newcommand{\C}       {\mathcal{C}}
\newcommand{\D}       {\mathcal{D}}
\newcommand{\F}       {\mathcal{F}}
\newcommand{\I}       {\mathcal{I}}
\newcommand{\T}       {\mathcal{T}}
\newcommand{\prem}    {\textsf{Prem}}
\newcommand{\con}     {\textsf{Con}}
\newcommand{\ta}      {\textsf{T}}
\newcommand{\ax}      {\textsf{Ax}}

\newcommand{\VAL}     {\textsc{Val}}
\newcommand{\SSB}     {{\bf SSbe}}
\newcommand{\SSO}     {{\bf SSat}}
\newcommand{\SSC}     {{\bf SSco}}
\newcommand{\SSU}     {{\bf SSmu}}

\newcommand{\two}     {{\bf 2}}
\newcommand{\twi}     {{\bf \Omega}}
\newcommand{\Rra}     {\Rrightarrow}

\newcommand{\jul}{\mathnormal{^{1}\hspace{-0,12cm}/\hspace{-0,04cm}_{2}}} 

\DeclareMathAlphabet{\mathpzc}{OT1}{pzc}{m}{it}

\begin{document}

\title{\textbf{Criptografia RSA gaussiana}}
\author{\textbf{Luis Antonio Co\^{e}lho}\vspace{2cm}\\
Trabalho de Conclus\~{a}o de Curso - apresentado \`{a}\\ Faculdade de Tecnologia da\\ Universidade Estadual de Campinas \vspace{2cm}\\
Orientadora:
\textbf{Profa. Dra. Juliana Bueno}\vspace{2cm}\\}
 \maketitle



% ------------------------------------------------------------------------
%{
%
%%%%%%%%%%%%%%%%%%%%%%%%%%%%%%%%%%%%%%%%%%%%%%%%%%%%%DEDICAT�RIA%%%%%%%%%%%%%%%%%%%%%%%%%%%%%%%%%%%%%%%%%%%%%%%%%%%%%%%%%%%%%

\thispagestyle{empty}

\hspace{1cm} \vspace{8.5cm}


\hspace{2.83cm}\textit{\Large{Dedico este trabalho a minha turma e aminha fam\'ilia, por conta do apoio durante sua produ\c{c}\~ao. }}
                   % dedicat�ria
%}
% ------------------------------------------------------------------------
%{
%
%%%%%%%%%%%%%%%%%%%%%%%%%%%%%%%%%%%%%%%%%%%%%%%%%%%%%%%INVOCA��O%%%%%%%%%%%%%%%%%%%%%%%%%%%%%%%%%%%%%%%%%%%%%%%%%%%%%%%%%%%%%

\thispagestyle{empty}

%\vspace{20cm}

\noindent{AMO-TE TANTO, meu amor... n�o cante}\\
O humano cora��o com mais verdade...\\
Amo-te como amigo e como amante\\
Numa sempre deversa realidade.\\

 \vspace{0.5 cm}

\noindent{Amo-te afim, de um calmo amor prestante,}\\
E te amo al�m, presente na saudade.\\
Amo-te, enfim, com grande liberdade\\
Dentro da eternidade e a cada instante.\\

 \vspace{0.5 cm}

\noindent{Amo-te como um bicho, simplesmente,}\\
De um amor sem mist�rio e sem virtude\\
Com um desejo maci�o e permanente.\\

 \vspace{0.5 cm}

\noindent{E de te amar assim muito e ami�de,}\\
� que um dia em teu corpo de repente\\
Hei de morrer de amar mais do que pude.\\



(Vin�cius de Moraes, \textit{Soneto do amor total})
                   % invoca��o
%}
% ------------------------------------------------------------------------
%{
%\typeout{Acknowledgements}
%
%%%%%%%%%%%%%%%%%%%%%%%%%%%%%%%%%%%%%%%%%%AGRADECIMENTOS%%%%%%%%%%%%%%%%%%%%%%%%%%%%%%%%%%%%%%%%%%%%%%%%%%%%%%%%%%%%%%%%%%%%
\pagestyle{fancy}
%\pagestyle{fancy}
\fancyhead[C]{\textsl{Agradecimentos}}
\fancyhead[R]{7}
\fancyfoot[C]{}	


\chapter*{Agradecimentos}

\vspace{1.5cm}


\noindent Agrade�o neste trabalho primeiramente a Deus que permitiu que tudo isso acontecesse, ao longo de minha vida, e n�o somente nestes anos como universit�rio, mas que em todos os momentos � o maior mestre que algu�m pode conhecer.Agrade�o a todos os professores por me proporcionar o conhecimento n�o apenas racional, mas a manifesta��o do car�ter e afetividade da educa��o no processo de forma��o profissional, por tanto que dedicaram a mim, n�o somente por terem me ensinado, mas por terem me feito aprender. A palavra mestre, nunca far� justi�a aos professores dedicados aos quais sem nominar ter�o os meus eternos agradecimentos. Agrade�o a minha m�e Raquel, hero�na que me deu apoio, incentivo nas horas dif�ceis, de des�nimo e cansa�o e a todos que direta ou indiretamente fizeram parte da minha forma��o, o meu muito obrigado.\\
                   % agradecimentos
%}
% -----------------------------------------------------------------------
{ \typeout{Abstract}

%%%%%%%%%%%%%%%%%%%%%%%%%%%%%%%%%%%%%%%RESUMO%%%%%%%%%%%%%%%%%%%%%%%%%%%%%%%%%%%%%%%%%%%%%%%%%%%%%%%%%%%%%%%%%%%%%%%%%%%%%%%%

\thispagestyle{empty}

\hspace{1cm} \vspace{2.2cm}

\noindent {\Huge {\bf Resumo}}

\vspace{1.5cm}

\noindent O presente relat�rio exp�e o resultado parcial do projeto para TCC sobre o algoritmo de criptografia RSA gaussiano.
                    % resumo
}
% ------------------------------------------------------------------------

\setcounter{page}{1}
\tableofcontents                  % cria o �ndice

% ------------------------------------------------------------------------
{ %\typeout{Introducao}
%
%%%%%%%%%%%%%%%%%%%%%%%%%%%%%%%%%%%%%%%%%%%INTRODU��O%%%%%%%%%%%%%%%%%%%%%%%%%%%%%%%%%%%%%%%%%%%%%%%%%%%%%%%%%%%%%%%%%%%%%%%

\thispagestyle{empty}

\addcontentsline{toc}{chapter}{Introdu��o}

\chapter*{Introdu��o}


Introduza o que voc� pretende fazer no decorrer do seu trabalho.
					
\pagestyle{fancy}
\fancyhead[C]{\textsl{Introdu��o}}
\fancyhead[R]{\thepage}
\fancyfoot[C]{}
\fancyhead[L]{}

\addcontentsline{toc}{chapter}{Introdu��o} % insere no sum�rio a introdu��o

\chapter*{Introdu\c{c}\~{a}o}
\label{Intro}

O sigilo sempre foi uma arma explorada pelos seres humanos para vencer certas batalhas, e at\'e mesmo para a cotidiana miss\~{a}o de se comunicar. Foi a partir dessa necessidade que se criou a \textit{criptografia}, nome dado ao conjunto de t\'ecnicas usadas para se  comunicar em c\'odigos. Seu objetivo \'{e} garantir que apenas os envolvidos na comunica\c{c}\~ao possam compreender a mensagem codificada (ou criptogtafada), garantindo que terceiros n\~ao saibam o que foi conversado.

Para compreender como funciona o processo de codifica\c{c}\~ao e decodifica\c{c}\~ao faz-se necess\'ario o uso de uma s�rie de termos t\'ecnicos, e para fins pedag�gicos iremos introduzir tais conceitos apresentando um dos primeiros algoritmos criptogr\'aficos que se tem conhecimento, a criptografia de C\'esar. Para mais detalhes sobre o tema, veja Criptografia, por Coutinho\cite{coutinho}.

A chamada \textit{criptografia de C\'esar}, criada pelo imperador romano C\'esar Augusto, consistia em substituir cada letra da mensagem por outra que estivesse a tr\^es posi\c{c}\~oes a frente, como, por exemplo, a letra \textbf{A} que neste algoritmo \'e substitu\'ida pela letra \textbf{D}.  

Uma forma muito natural de se generalizar o algoritmo de C\'esar \'e fazer a troca de cada letra da mensagem por outra que venha em uma posi\c{c}\~ao qualquer fixada. A chamada \textit{criptografia de substitui\c{c}\~ao monoalfab\'etica} consiste em substituir cada letra por outra que ocupe $n$ posi\c{c}\~oes � sua frente, sendo que o n\'umero $n$ \'e conhecido apenas pelo emissor e pelo receptor da mensagem. O n\'umero $n$ \'e a \textit{chave criptogr\'afica}. Para decifrar a mensagem, precisamos substituir as letras que formam a mensagem criptografada pelas letras que est\~ao $n$ posi\c{c}\~oes antes.

O algoritmo monoalfab\'etico tem a caracter\'istica indesejada de ser de f\'acil decodifica\c{c}\~ao, pois possui apenas {26} chaves poss\'iveis, e isso faz com que no m\'aximo em {26} tentativas o c\'odigo seja decifrado. Com o intuito de dificultar a quebra do c\'odigo monoalfab\'etico foram propostas as \textit{cifras de substitui\c{c}\~ao polialfab\'eticas} em que a chave criptogr\'afica passa a ser uma \textit{palavra} ao inv\'es de um n\'umero. A ideia \'e usar as posi\c{c}\~oes ocupadas pelas letras da chave para determinar o n\'umero de posi\c{c}\~oes que devemos avan\c{c}ar para obter a posi\c{c}\~ao da letra encriptada. Vejamos, por meio de um exemplo, como funciona esse sistema criptogr\'afico.

Sejam ``SENHA'' a nossa chave criptogr\'afica e ``ABOBORA'' a mensagem a ser encriptada. Abaixo colocamos as letras do alfabeto com suas respectivas posi\c{c}\~oes. Observe que repetimos a primeira linha de letras para facilitar a localiza\c{c}\~ao da posi\c{c}\~ao da letra encriptada e usamos a barra para indicar que estamos no segundo ciclo. 

\[
\begin{array}{ccccccccccccc}
    1      & 2 & 3 & 4 & 5          & 6 & 7 & 8          & 9 & 10 &  11 & 12 & 13 \\  
\textbf{A} & B & C & D & \textbf{E} & F & G & \textbf{H} & I & J  &  K  & L  & M  \\ 
  &   &   &   &   &   &   &   &   &    &     &    &    \\ 
    14      & 15 & 16 & 17 & 18 & 19          & 20 & 21 & 22 & 23 & 24 & 25 & 26 \\
\textbf{N}  & O  & P  & Q  & R  & \textbf{S}  & T  & U  & V  & X  & Y  & W  & Z \\
&   &   &   &   &   &   &   &   &    &     &    &    \\ 
    27      & 28 & 29 & 30 & 31          & 32 & 33 & 34          & 35 & 36 &  37 & 38 & 39 \\  
\overline{A} & \overline{B} & \overline{C} & \overline{D} & \overline{E} & \overline{F} & \overline{G} & \overline{H} & \overline{I} & \overline{J}  &  \overline{K}  & \overline{L}  & \overline{M}  \\
\end{array}
\]

Vejamos como encriptar a palavra ``ABOBORA''. Iniciamos o processo escrevendo a mensagem. Ao lado de cada letra da mensagem aparece entre par\^enteses o n\'umero que indica a sua posi\c{c}\~ao. Abaixo da mensagem escrevemos as letras da chave criptogr\'afica, repetindo-as de forma c\'iclica quando necess\'ario. Analogamente, ao lado de cada letra da chave aparece entre par\^enteses o n\'umero da posi\c{c}\~ao ocupada de cada letra, e o sinal de soma indica que devemos avan\c{c}ar aquele n�mero de posi��es. Ao final do processo aparecem as letras encriptadas. Entre par�nteses est� a posi\c{c}\~ao resultante da combina\c{c}\~ao das posi\c{c}\~oes da mensagem e da chave.   

\footnotesize{
\[
\begin{array}{lllllll||l}
     A (1)  &      B (2)  &      O (15) &      B (2)  &      O (15) &     R (18)  &    A (1)	 & \textrm{Mensagem}  \\
\downarrow  & \downarrow  & \downarrow  & \downarrow  & \downarrow  & \downarrow  & \downarrow &\\ 
    S (+19) &     E (+5)  &     N (+14) &     H  (+8) &     A (+1)  &    S (+19)  &   E (+5)   &\textrm{Chave}  \\
\downarrow  & \downarrow  & \downarrow  & \downarrow  & \downarrow  & \downarrow  & \downarrow & \\
		 T (20) &      G (7)  &      C (29) &      J (10) &     P (16)  &    K (37)   &    F (6)   & \textrm{Mensagem encriptada}  \\
\end{array}
\]
}
  
Observe que a encripta\c{c}\~ao polialfab\'etica \'e mais dif\'icil de ser quebrada que a monoalfab\'etica uma vez que letras iguais n\~ao t\^em, necessariamente, a mesma encripta\c{c}\~ao. Neste tipo de criptografia o emissor precisa passar a chave para o receptor da mensagem de forma segura para que o receptor possa decifrar a mensagem, isto \'e, a chave usada para encriptar a mensagem \'e a mesma que deve ser usada para decifrar a mensagem. Veremos que esse \'e justamente o ponto fraco neste tipo de encripta\c{c}\~ao pois usa a chamada \textit{chave sim\'etrica}, ou seja, a chave usada pelo emissor para codificar a mensagem \'e a mesma usada pelo receptor para decodificar a mensagem. Nesse processo, a chave deve ser mantida em segredo e bem guardada para garantir que o c\'odigo n\~ao seja quebrado, e isso requer algum tipo de contato f\'isico entre emissor e receptor da mensagem.

Durante a  Primeira Guerra Mundial o contato f\'isico para a troca de chaves era complicado, e isso estimulou a cria\c{c}\~ao de m\'aquinas autom\'aticas de criptografia. O \textit{Enigma} foi uma dessas m\'aquinas e era utilizada pelos alem\~aes tanto para criptografar como para descriptografar c\'odigos de guerra. Semelhante a uma m\'aquina de escrever, os primeiros modelos foram patenteados por Arthur Scherbius em 1918. Essas m\'aquinas ganharam popularidade entre as for\c{c}as militares alem\~as devido � facilidade de uso e sua suposta indecifrabilidade do c\'odigo. 

O matem\'atico Alan Turing foi o respons\'avel por quebrar o c\'odigo dos alem\~aes durante a Segunda Guerra Mundial. A descoberta de Turing mostrou a fragilidade da criptografia baseada em chave sim\'etrica e colocou novos desafios \`a criptografia. O grande problema passou a ser a quest\~ao dos protocolos, isto \'e, como transmitir a chave para o receptor de forma segura sem que seja necess\'ario o contato f\'isico entre as partes? 

Em 1949, com a publica\c{c}\~ao do artigo \textit{Communication Theory of Secrecy Systems} \cite{shannon} de Shannon, temos a inaugura\c{c}\~ao da criptografia moderna. Neste artigo ele escreve matematicamente que cifras teoricamente inquebr\'aveis s\~ao semelhantes �s cifras polialfab\'eticas. Com isso ele transformou a criptografia que at\'e ent\~ao era uma arte, em uma ci\^encia.

Em 1976 Diffie e Hellman publicaram \textit{New Directions in Cryptography} \cite{newdirections}. Neste artigo h\'a a introdu\c{c}\~ao ao conceito de \textit{chave assim\'etrica}, onde h\'a chaves diferentes entre o emissor da mensagem e seu receptor. Com a assimetria de chaves n\~ao era mais necess\'ario um contato t\~ao pr\'oximo entre emissor e receptor. Neste mesmo artigo \'e apresentado o primeiro algoritmo de criptografia de chave assim\'etrica ou como \'e mais conhecido nos dias atuais \textit{Algoritmo de Criptografia de Chave P\'ublica}, o protocolo de Diffie-Hellman.

Um dos algoritmos mais famosos da criptografia de chave p\'ublica \'e o \textit{RSA} \cite{rivest}, algoritmo desenvolvido por Rivest, Shamir e Adleman. Este algoritmo se tornou popular por estar presente em muitas aplica\c{c}\~oes de alta seguran\c{c}a, como bancos, sistemas militares e servidores de internet.

Para que se possa compreender por completo o algoritmo faz-se necess�rio possuir alguns conhecimentos em teoria de n\'umeros como fatora\c{c}\~ao e aritm\'etica modular. Estes conhecimentos ser\~ao apresentados mais adiante neste trabalho.

No algoritmo RSA existe uma chave p\'ublica $n$, que \'e a multiplica\c{c}\~ao dos primos $p$ e $q$. O emissor E codifica a mensagem usando um n\'umero primo $p$. Em seguida E envia publicamente a mensagem codificada junto com a chave $n$ para o receptor R. R possui o n\'umero $q$, que juntamente ao n\'umero $n$ servem para decodificar a mensagem. 

Embora a quebra do RSA seja aparentemente simples, bastando fatorar $n$ para descobrir seus fatores, o grande problema \'e na realidade computacional, pois usa-se como $p$ e $q$ n\'umeros primos muito altos, pr\'oximos a $2^{512}$. Com um n\'umero t\~ao alto um computador comum levaria bem mais que uma vida humana para decifrar a mensagem.

Com base nestes conhecimentos sobre criptografia, temos que o objetivo deste trabalho \'e analisar a viabilidade de uma criptografia inspirada pelo algoritmo RSA cl\'assico, a qual substitui os n\'umeros primos pelo conjunto denominado de \textit{primos de Gauss} \cite{intGauss}, resultando, assim, no que chamamos por \textit{criptografia RSA gaussiana}. Para que tal algoritmo seja vi\'avel \'e necess\'ario adaptar uma s\'erie de resultados relativas aos n\'umeros primos aos n\'umero primos de Gauss. Dessa forma, nossa tarefa ser\'a adaptar tanto quanto o poss\'ivel os primos de Gauss \`as demosnta\c{c}\~oes desses teoremas.

Como se trata de uma proposta inovadora, deixamos para trabalhos futuros uma an\'alise comparativa entre as criptografias RSA cl\'assica e a RSA gaussiana.        
\chapter {Primos e Fatora\c{c}\~ao}
\label{Num}

\section{Ciclos e Restos}	
\subparagraph{
Para podermos compreender a aritm\'etica modular, precisamos come\c{c}ar entendendo o conceito de ciclicidade, que s\~ao os fatos que ocorrem sempre ap\'os um determinado per\'iodo constante. Um bom exemplo deste conceito \'e o nascer do sol, que \'e um evento que ocorre sempre ap\'os um ciclo de {24} horas, assim como o dia de seu anivers\'ario ocorre uma vez a cada ciclo de um ano.
}
\subparagraph{
O mesmo tipo de evento \'e observado com o resto dos n\'umeros inteiros. Tomemos por exemplo os restos de divis\~ao dos n\'umeros inteiros, abaixo mostrados de 1 \`a 12, pelo n\'umero inteiro {4}:
}

\[
\begin{array}{ccccccccccccc}
  {Inteiro} & 1 & 2 & 3 & 4 & 5 & 6 & 7 & 8 & 9 & 10 &  11 & 12 \\  
	{Resto} & 1 & 2 & 3 & 0 & 1 & 2 & 3 & 0 & 1 & 2  &  3 & 0 \\ 
\end{array}
\]

\subparagraph{
\'E vis\'ivel que ap\'os {4} n\'umeros o resto tende a se repetir. O mesmo feito ocorre a qualquer n\'umero inteiro $n$, onde o ciclo se repetir\'a sempre a cada $n$ itera\c{c}\~oes. Os n\'umeros que apresentam o resto {0} s\~ao conhecidos como m\'ultiplos de $n$.
}

\section{N\'{u}meros Primos e Compostos}

\subparagraph{
Existe um tipo especial de n\'umero que s\'o \'e m\'ultiplo, ou seja, possui resto {0}, em duas condi\c{c}\~oes, quando $n$ \'e igual a {1} ou quando ele \'e igual a $n$. A esse conjunto de n\'umeros atribui-se o nome de \textit{n\'umeros primos}.
}
\subparagraph{
\textit{Existem infinitos n\'umeros primos}, caso n\~ao acredite vamos supor que o conjunto finito de primos seja composto por $p_{1},  p_{2}, ..., p_{r} $. Considerando que o n\'umero inteiro $n=(p_{1})(p_{2})...(p_{r}) + 1$. $n$ deve possuir um fator $p$, que est\'a contido em $p_{1},  p_{2}, ..., p_{r} $, mas isso significa q $p$ divide $1$, o que \'e absurdo e prova que o conjunto n\~ao tem fim.
}
\subparagraph{
Todo o n\'umero que n\~ao \'e primo \'e chamado de \textit{N\'umero Composto}, sendo que este n\'umero composto pode ser escrito em \textit {uma combina\c{c}\~ao \'unica de fatores primos}. O processo de se descobrir estes fatores \'e chamado de \textit{fatora\c{c}\~ao} e \'e detalhado na pr\'oxima seção.
}

\section{Fatora\c{c}\~{a}o}

\subparagraph{
Anteriormente falamos que todo o n\'umero pode ser escrito por uma combina\c{c}\~ao de fatores primos, neste cap\'itulo vamos abordar como se pode obter estes fatores.
}
\subparagraph{
Come\c{c}amos por escolher o n\'umero inteiro $n$ ao qual iremos fatorar, em seguida testamos a sua divisibilidade por $2$, se for tente divid\'i-lo novamente por $2$, sen\~ao passa-se para o pr\'oximo n\'umero primo, o $3$. Repete-se esse procedimento at\'e chegarmos a $\sqrt{n}$, caso n\~ao achemos nenhum fator primo at\'e $\sqrt{n}$, $n$ \'e primo.
}
\subparagraph{
Quando acabamos de realizar a fatora\c{c}\~ao, chegamos a um n\'umero fatorado da forma $n = (2^{a_{1}})(3^{a_{2}}) ... (p^{a_{p}})$, todo o n\'umero inteiro pode ser escrito nessa forma, chamada forma fatorada, veja, por exemplo o $12 = (2^2)(3^1)$ e o $19 = (19^1)$.
}
\subparagraph{
Essa forma fatorada nos \'e formalmente apresentada pelo \textit{Teorema da Fatora\c{c}\~ao \'Unica}. Ele nos diz que dado um n\'umero inteiro $n\geq2$ pode-se escrev\^e-lo de forma \'unica como:
}
\[	
	\begin{array}{c}
		\textit{$n = (p^{e_{1}}_{1}) ... (p^{e_{k}}_{k}) $}
	\end{array}
\]
\paragraph{
onde $1 < p_1 < ... < p_k $ s\~ao primos e $e_1, ..., e_k$ s\~ao inteiros.
}
\subparagraph{
Mesmo algoritmo da fatora\c{c}\~ao sendo t\~ao simples de se compreender, ele \'e demorado at\'e para os mais modernos computadores. Para se ter uma ideia disto, um computador comum executa cerca de {50} divis\~oes por segundo, para se calcular com certeza que um n\'umero pr\'oximo a $10^{100}$ \'e primo ele levaria cerca de {317} decilh\~oes de anos. Essa demora computacional que torna os primos t\~ao atraentes a criptografia, pois sua multipli\c{c}\~ao \'e f\'acil para se obter o resultado, mas muito complexa para que se descubram quais os n\'umeros envolvidos nela apenas com o resultado final.
}


\chapter {Aritm\'{e}tica Modular}
\label{Mod}

\section{Ciclos e Restos}	
\subparagraph{
Para podermos compreender a aritm\'etica modular, precisamos come\c{c}ar entendendo o conceito de ciclicidade. Um bom exemplo deste conceito \'e o nascer do sol, que \'e um evento que ocorre sempre ap\'os um ciclo de {24} horas, assim como o dia de seu anivers\'ario ocorre uma vez a cada ciclo de um ano.
}

\section{N\'{u}meros Primos Naturais}	

\section{Per\'{i}odo e Fatora\c{c}\~{a}o}

\section{Inverso Multiplicativo}

\chapter {Inversos Modulares}
\label{InvMod}

\section{Inversos modulares}	
\subparagraph{
Nosso objetivo com o decorrer deste cap\'itulo \'e o de explicar a opera\c{c}\~ao matem\'atica mais importante para para o algoritmo RSA. Para podermos comprend\^e-la vamos relembrar do cenceito ensinado no col\'egio de inverso multiplicativo, que consiste em obter o n\'umero que multiplicado a um n\'umero $n$ qualquer resulte em $1$. A opera\c{c}\~ao do inverso modular parte do mesmo princ\'ipio.
}
\subparagraph{
Vamos supor que queremos obter o inverso modular de $6$ para o m\'odulo $7$, o que n\'os teremos que fazer ent\~ao \'e encontrar qual o n\'umero que multiplicado por $6$ tem resto $1$ quando dividido por $7$. Come\c{c}amos pelo $1$, teremos que $6 \cdot 1 = 6$, $6 \equiv 6 (mod7)$. Com $2$ o resultado ser� $12$, logo $12 \equiv 5 (mod7)$, que para n\'os tamb\'em n\~ao serve. Tentando o $3$ obtemos $4$ e com $4$ obtemos $3$. Com o $5$ nosso retorno ser\'a $2$. Finalmente quando chegamos ao $6$ n\'os temos que $6 \cdot 6 = 36$, $36 \equiv 1 (mod 7)$. Com isso podemos concluir que o inverso multiplicativo de $6$ no m\'odulo $7$ \'e o pr\'opio $6$.
}
\subparagraph{
Para simplificar o que foi dito acima, podemos dizer a opera\c{c}\~ao de inverso multiplicativo no m\'odulo $n$ para $a$ consiste em encontar um n\'umero $a'$ tal que:
}
\[	
	\begin{array}{c}
		\textit{$a \cdot a' \equiv 1 (mod n)$}
	\end{array}
\]

\section{Inexist\^encia e exist\^encia de inversos}	

\subparagraph{
Antes de come\c{c}armos vamos tentar calcular o inverso multiplicativo de $2$ no m\'odulo $8$, vamos l\'a: 
}
\[	
	\begin{array}{c}
		\textit{$2 \cdot 0 \equiv 0 \not\equiv 1(mod 8)$}\\
		\textit{$2 \cdot 1 \equiv 2 \not\equiv 1(mod 8)$}\\
		\textit{$2 \cdot 2 \equiv 4 \not\equiv 1(mod 8)$}\\
		\textit{$2 \cdot 3 \equiv 6 \not\equiv 1(mod 8)$}\\
		\textit{$2 \cdot 4 \equiv 8 \not\equiv 1(mod 8)$}\\
		\textit{$2 \cdot 5 \equiv 0 \not\equiv 1(mod 8)$}\\
		\textit{$2 \cdot 6 \equiv 2 \not\equiv 1(mod 8)$}\\
		\textit{$2 \cdot 7 \equiv 4 \not\equiv 1(mod 8)$}\\
	\end{array}
\]

\subparagraph{
N\~ao encontramos nenhuma resposta pois, simplesmente, n\~ao h\'a. Antes que se pergunte o motivo de n\~ao tentarmos com n\'umeros maiores que $7$, \'e v\'alido lembrar que a partir do $8$ ter\'iamos a repeti\c{c}\~ao de resultados por conta das congru\^encias.
}
\subparagraph{
A opera\c{c}\~ao de inverso multiplicativo s\'o possui resultado em casos onde o n\'umero $a$ ao qual queremos calcular o inverso e o m'odulo s\~ao \textit{primos entre si}, ou seja, n\~ao possuam nenhum fator em comum. Por conta disso usamos os n\'umeros primos no algoritmo RSA.
}
\subparagraph{
Para comprovar o que foi dito acima, vamos tomar um n\'umero $a$, tal que
}
\[	
	\begin{array}{c}
		\textit{$a \cdot a' \equiv 1(mod n)$}\\
	\end{array}
\]
\paragraph{
isso pode ser traduzido em linguagem humana como $n$ divide $ a \cdot a' - 1$. Isso em linguajar matem\'atico pode ser escrito como:
}
\[	
	\begin{array}{c}
		\textit{$a \cdot a' - 1 = n \cdot k$}\\
	\end{array}
\]
\paragraph{
como estamos atr\'as de saber se $a$ e $n$ n\~ao possuem fator comum, ent�o h\'a de haver um $k$ inteiro para a equa\c{c}\~ao acima. Nosso primeiro passo para provar isso ser\'a de se criar o conjunto $V(a,n)$, esse conjunto \'e formado por inteiros positivos e pode ser escrito como
}
\[	
	\begin{array}{c}
		\textit{$x \cdot a + y \cdot n$}\\
	\end{array}
\]
\subparagraph{
Em um primeiro momento este conjunto e esta nova f\'ormula podem parecer estranhos ao que se via antes, mas se comprovarmos que $ 1 \in V(a,n)$, conclu\'imos que devem haver dois inteiros $x_0$ e $y_0$, ou se preferir $a'$ e $k$, logo:
\[	
	\begin{array}{c}
		\textit{$1 = a \cdot a' - n \cdot k$}\\
	\end{array}
\]
}
\subparagraph{
Uma das propiedades deste conjunto \'e a de $n$ pertencer a ele quando $x = a' = 0$ e $y = k = 1$. Isto significa que os inteiros que podem completar a equa\c{c}\~ao est\~ao entre $1$ e $n$. Mas para podermos dar essa demonstra\c{c}\~ao como completa, precisamos provar que $m = 1$.
}

\subparagraph{
Estou achando que est\'a confuso, pe�o que marque muito nesse final
}
\chapter {Teorema chin\^es do resto}
\label{TCR}

\section{Introdu\c{c}\~ao a t\'ecnica}

\subparagraph{
Para sermos iniciados nesta t\'ecnica, vamos analisar o seguinte problema: Qual o menor inteiro que possui resto $1$ na divis\~ao por $3$ e resto $2$ na divis\~ao por $5$. Podemos vir a tranformar esse problema nas seguintes equa\c{c}\~oes:
}
\[	
	\begin{array}{c}
		\textit{$n = 3q_1 + 1$ e $n = 5q_2 + 2$}
	\end{array}
\]
\paragraph{
Essas equa\c{c}\~oes tamb\'em podem ser denotadas em forma modular como:
}
\[	
	\begin{array}{c}
		\textit{$n \equiv 1 (mod 3)$ e $n \equiv 2 (mod 5)$}
	\end{array}
\]
\paragraph{
Essa sa\'ida modular nos deixou com apenas uma vari\'avel, mas ainda n\~ao resolveu ao nosso problema. Para fazermos isso vamos substituir $n$ por $5q_2 + 2$, montando a seguine equa\c{c}\~ao modular:
}
\[	
	\begin{array}{c}
		\textit{$5q_2 + 2 \equiv 1 (mod 3)$}
	\end{array}
\]
\paragraph{
Como $5 \equiv 2(mod 3)$, substitu\'imos:
}
\[	
	\begin{array}{c}
		\textit{$ 2q_2 + 2 \equiv 1 (mod 3)$}
	\end{array}
\]
\paragraph{
Feito isso, passamos $2$ para o outro lado da equa\c{c}\~ao
}
\[	
	\begin{array}{c}
		\textit{$ 2q_2  \equiv -1 (mod 3)$}
	\end{array}
\]
\paragraph{
Como $-1 \equiv 2 (mod 3)$, n\'os substit\'imos novamente, e depois dividimos a equa\c{c}\~ao por $2$, e obtemos
}
\[	
	\begin{array}{c}
		\textit{$ q_2  \equiv 1 (mod 3)$}
	\end{array}
\]
\paragraph{
Com isso, conclu\'imos que
}
\[	
	\begin{array}{c}
		\textit{$ q_2  \equiv q_3 + 1 (mod 3)$}
	\end{array}
\]
\paragraph{
Sei que parece que mais uma equa\c{c}\~ao s\'o serve para tornar a resolu\c{c}\~ao mais complexa, mas vamos a reorganizar como
}
\[	
	\begin{array}{c}
		\textit{$ q_2 = 3q_3 + 1 $}
	\end{array}
\]
\paragraph{
Agora substitu\'imos
}
\[	
	\begin{array}{c}
		\textit{$n = 5(3q_3 + 1) + 2 = 15q_3 +7$}
	\end{array}
\]
\paragraph{
Feito isso, vamos por o $3$ em evid\^encia em todos os lugares, obtendo:
}
\[	
	\begin{array}{c}
		\textit{$n = 3(5q_3) +3(2) +1 = 3(5q_3 +2)+1$}
	\end{array}
\]
\paragraph{
Este procedimento foi feito apenas para provar que a equa\c{c}\~ao deixa resto 1 se dividida por 3, de forma an\'aloga, abaixo \'e mostrado como ela deixa resto $2$ quando dividida por $5$.
}
\[	
	\begin{array}{c}
		\textit{$n = 5(3q_3) +5(1) +2 = 5(3q_3 +1)+2$}
	\end{array}
\]
\subparagraph{
Ap\'os tudo isso feito ainda n\~ao possu\'imos a solu\c{c}\~ao final, mas j\'a sabemos que \'e um n\'umero da forma $15q_3 + 7$, substituindo $q_3$ or $0$, iremos obter $7$, que \'e o resultado procurado.
}

\section{Provas ao teorema}

\subparagraph{
O teorema chin\^es do resto \'e um procedimento tomado para resolver sistema de congru\^encias, como o descrito acima. Ele foi descrito pela primeira vez pelo Manual de aritm\'etica do mestre Sun, por volta do s\'eculo III d.C. 
}
\subparagraph{
Para ver a defini\c{c}\~ao formal desse teorema, vamos considerar o sistema
}
\[	
	\begin{array}{c}
		\textit{$x \equiv a (mod n)$}\\
		\textit{$x \equiv b (mod m)$}\\
	\end{array}
\]
\paragraph{
nele, $n$ e $m$ s\~ao inteiros diferentes entre si. Tomemos $x_0$ como um n\'umero cappaz de satisfazer ambas as congru\^encia de forma simult\^anea e teremos:
}
\[	
	\begin{array}{c}
		\textit{$x_0 \equiv a (mod m)$}\\
		\textit{$x_0 \equiv b (mod n)$}\\
	\end{array}
\]
\paragraph{
Para podermos juntar ambas as equa\c{c}\~oes converteremos uma em equa\c{c}\~ao, nesse caso teremos 
}
\[	
	\begin{array}{c}
		\textit{$x_0 = a + m\cdot k$, com $k$ sendo um inteiro qualquer}\\
	\end{array}
\]
\paragraph{
Feito isso, chegaremos em
}
\[	
	\begin{array}{c}
		\textit{$a + m\cdot k \equiv b (mod n)$}\\
	\end{array}
\]
\paragraph{
que pode ser substitu\'ida por
}
\[	
	\begin{array}{c}
		\textit{$ m\cdot k \equiv (b-a) (mod n)$}\\
	\end{array}
\]
\subparagraph{
Agora vamos supor que $m$ e $n$ s\~ao primos entre si. Pelo teorema apresentado no cap\'ituo sobre inversos multiplicativos n\'os j\'a sabemos que eles possuem inverso multiplicativo um para o outro. Tomemos $m'$ como o inverso de $m$ no m\'odulo $n$. Multipplicando toda a congru\^encia por $m'$ obtemos
}
\[	
	\begin{array}{c}
		\textit{$ k \equiv m'\cdot(b-a) (mod n)$}\\
	\end{array}
\]
\paragraph{
que pode ser escrita como:
}
\[	
	\begin{array}{c}
		\textit{$ k \equiv m'\cdot(b-a)+n \cdot t$, para um inteiro $t$ qualquer}\\
	\end{array}
\]
\subparagraph{
Substituindo a parte de $k$, n\'os obtemos
}
\[	
	\begin{array}{c}
		\textit{$ x_0 \equiv a + m (m'\cdot(b-a)+n \cdot t) $}\\
	\end{array}
\]
\paragraph{
Podemos ver agora que para qualquer $t$, $a + m (m'\cdot(b-a)+n \cdot t$ \'e parte da solu\c{c}\~ao da congru\^encia, sabendo disso, agora podemos descrever o teorema em si, que ser\'a feito na pr\'oxima se\c{c}\~ao.
}

\section{O teorema chin\^es do resto}
\subparagraph{
\textit{Teorema chin\^es do resto} - Sejam $m$ e $n$ inteiros positivos primos entre si. Se $a$ e $b$ s\~ao inteiros quaisquer, ent\~ao o sistema
}
\[	
	\begin{array}{c}
		\textit{$ x \equiv a (mod m) $}\\
		\textit{$ x \equiv b (mod n) $}
	\end{array}
\]
\paragraph{
sempre tem solu\c{c}\~ao e qualquer uma de suas solu\c{c}\~oes pode ser escrita na forma
}
\[	
	\begin{array}{c}
		\textit{$ a + m \cdot(m' \cdot (b-a) + n \cdot n) $}\\
	\end{array}
\]
\paragraph{
onde $t$ \'e um inteiro qualquer e $m'$ \'e o inverso de $m$ no m\'odulo $n$.
}

%%%%%%%%%%%%%CONSIDERA��ES FINAIS%%%%%%%%%%%%%%%%%%%%%%%%%%%%%%%%%%%%%%%%%%%%%%%%%%%%%%

\pagestyle{fancy}
\fancyhead[C]{\textsl{Considera��es Finais}}
\fancyhead[R]{\thepage}
\fancyfoot[C]{}

\addcontentsline{toc}{chapter}{Considera��es Finais}

\chapter*{Considera��es Finais}

Nesta sess\~ao faremos as \'ultimas considera\c{c}\~oes sobre a viabilidade do algoritmo RSA Gaussiano. Al\'em disso vamos ver o que outros pesquisadores j\'a est\~ao concluindo em suas pesquisas.

A primeira coisa que devemos prestar aten\c{c}\~ao \'e que no decorrer desta monografia fomos capazes de descrever o funcionamento do algoritmo de criptografia RSA.

Al\'em disso demos os primeiros passos rumo a uma criptografia RSA Gaussiana, e n\~ao encontramos nada que impedisse a sua realiza\c{c}\~ao, mas como foi visto no cap\'itulo \ref{IG}, ainda precisamos da comprava\c{c}\~ao de alguns teoremas matem\'aticos importantes para a realiza\c{c}\~ao deste algoritmo. 

O material publicado por \cite{koval} e \cite{elkassar} nos leva a crer na viabiliadade do algoritmo. O que ocorre \'e que ambos possuem vis\~oes bem diferentes com rela\c{c}\~ao ao RSA Gaussiano. \cite{koval} n\~ao defende o algoritmo, pois acredita que ele n\~ao acrescenta seguran\c{c}a ao algoritmo RSA, al\'em de deix\'a-lo menos pr\'atico. Abaixo citamos o trecho onde isso \'e afirmado:

\begin{quote}
``The extension of RSA algorithm into the field of Gaussian integers [...] is viable only if real primes p congruent to 3 modulo 4 are used [...]. The extended algorithm could add security only if breaking the original RSA is not as hard as factoring. Even in this case, it is not clear whether the extended algorithm would increase security. The Gaussian integer RSA is slightly less efficient than the original, therefore the original real integer RSA is more practical.''
\end{quote}

Enquanto isso, \cite{elkassar} defende o algoritmo Gaussiano por aumentar a seguran\c{c}a comparado ao cl\'assico, como pode ser lido abaixo:

\begin{quote}

``Arithmetic needed for the RSA cryptosystem in the domains of Gaussian integers and polynomials over finite fields were modified and computational procedures were described. There are advantages for the new schemes over the classical one. First, generating the odd prime numbers in both the classical and the modified methods requires the same amount of efforts. Second, the modified method provides an extension to the range of chosen messages and the trials will be more complicated. ''

\end{quote}

Baseado nos textos de ambos podemos concluir que al\'em da realiza\c{c}\~ao de tal algoritmo, outro problema a ser investigado em um trabalho futuro consiste na an\'alise de seguran\c{c}a e complexidade do algoritmo, visto que ainda n\~ao possu\'imos uma conclus\~ao definitiva sobre isso.
        %Considera��es Finais
%
%%%%%%%%%%%%%%%%%%%%%%%%%%%AP�NDICE 1: Ipcional%%%%%%%%%%%%%%%%%%%%%%%%%%%%%%%%%%%%%%%%%%%%

\addcontentsline{toc}{chapter}{Ap�ndice 1}

\chapter*{Ap�ndice 1\\ (Opcional)}

Exemplos dos mais interessantes  manuais de latex na rede s�o os
seguintes:


P�gina sobre criptografia do IME- USP Manuais de LaTeX em
portugu�s e ingl�s, inclusive com conversoires entre LaTeX e
outros formatos:
\\
http://www.ime.eb.br/~pinho/pessoal/latex/

Manual b�sico do IFGW- UNICAMP:
\\
http://www.ifi.unicamp.br/encontro/latex-exemplo.html
         %Ap�ndice 1: Opcional

\addcontentsline{toc}{chapter}{Bibliografia} %para aparecer a  bibliografia no �ndice
%%%%%%%%%%%%%%%%%%%%%%%%%%%%%%%%%%% Comandos para gerar a bibliografia em Portugu�s %%%%%%%%%%%%%%%%%%%%%%%%

\selectbiblanguage{brazil}
%\bibliographystyle{babplain}
\bibliographystyle{babalpha}
\bibliography{Bibliografia}

%%%%%%%%%%%%%%%%%%% Caso n�o gere a bibliografia por falta de pacotes rode com os comandos abaixo e retire o pacote \usepackage{babelbib} do in�cio do documento %%%%%%%%%%%

\bibliographystyle{alpha}
\nocite{*}
%\bibliography{Bibliografia}

%%%%%%%%%%%%%%%%%%%%%%%%%%%%%%%%%%%%%%%%%%%%%%%%%%%%%%%%%%%%%%%%%%%%%%%%%%%%%%%%%%%%%%%%%%%%%%%%%%%%%%%%%%%%%
\end{document}
