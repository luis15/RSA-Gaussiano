
%%%%%%%%%%%%%%%%%%%%%%%%%%%%%%%%%% DISSERTA��O OU TESE %%%%%%%%%%%%%%%%%%%%%%%%%%%%%%%%%%%%%%%%%
\documentclass[11pt]{report}
\usepackage{graphicx}
\usepackage[brazil]{babel}
\usepackage[latin1]{inputenc}
\usepackage{babelbib} % Esse pacote gera a bibliografia em Portugu�s. Acho que deve ser instalado algum pacote, n�o me recordo.
\usepackage{latexsym}
\usepackage{amsfonts}
\usepackage{amsthm}
\usepackage{amssymb}
\usepackage{amsfonts}
\usepackage{xypic}
\usepackage{ulem}
\usepackage[all]{xy}
\usepackage{amsmath}
\usepackage{amssymb}
\usepackage{mathrsfs}
\usepackage{amsfonts}

\usepackage{makeidx} %cria o �ndice remissivo
\makeindex

\vfuzz2pt % Don't report over-full v-boxes if over-edge is small
\hfuzz2pt % Don't report over-full h-boxes if over-edge is small
% TEOREMAS -------------------------------------------------------
\theoremstyle{alpha}
\newtheorem{Th}{Teorema}[section]
\newtheorem{Cor}[Th]{Corol�rio}
\newtheorem{Le}[Th]{Lema}
\newtheorem{Pro}[Th]{Proposi��o}
\theoremstyle{definition}
\newtheorem{Df}[Th]{Defini��o}
\newtheorem{Obs}[Th]{Observa��o}
\newtheorem{Dig}[Th]{Digress�o}
\newtheorem{Ex}[Th]{Exemplo}
\newtheorem{Esc}[Th]{Esc�lio}
\newtheorem{Ass}[Th]{Asser��o}

% MATEMATICA E LOGICA -------------------------------------------

\newcommand{\lan}     {\langle}
\newcommand{\ran}     {\rangle}
\newcommand{\IN}      {\mathbb{N}}
\newcommand{\A}       {\mathcal{A}}
\newcommand{\B}       {\mathcal{B}}
\newcommand{\C}       {\mathcal{C}}
\newcommand{\D}       {\mathcal{D}}
\newcommand{\F}       {\mathcal{F}}
\newcommand{\I}       {\mathcal{I}}
\newcommand{\T}       {\mathcal{T}}
\newcommand{\prem}    {\textsf{Prem}}
\newcommand{\con}     {\textsf{Con}}
\newcommand{\ta}      {\textsf{T}}
\newcommand{\ax}      {\textsf{Ax}}

\newcommand{\VAL}     {\textsc{Val}}
\newcommand{\SSB}     {{\bf SSbe}}
\newcommand{\SSO}     {{\bf SSat}}
\newcommand{\SSC}     {{\bf SSco}}
\newcommand{\SSU}     {{\bf SSmu}}

\newcommand{\two}     {{\bf 2}}
\newcommand{\twi}     {{\bf \Omega}}
\newcommand{\Rra}     {\Rrightarrow}

\newcommand{\jul}{\mathnormal{^{1}\hspace{-0,12cm}/\hspace{-0,04cm}_{2}}} 

\DeclareMathAlphabet{\mathpzc}{OT1}{pzc}{m}{it}

\begin{document}

\title{\textbf{Criptografia RSA gaussiana}}
\author{\textbf{Luis Antonio Co\^{e}lho}\vspace{2cm}\\
Trabalho de Conclus\~{a}o de Curso - apresentado \`{a}\\ Faculdade de Tecnologia da\\ Universidade Estadual de Campinas \vspace{2cm}\\
Orientadora:
\textbf{Profa. Dra. Juliana Bueno}\vspace{2cm}\\}
 \maketitle



% ------------------------------------------------------------------------
%{
%
%%%%%%%%%%%%%%%%%%%%%%%%%%%%%%%%%%%%%%%%%%%%%%%%%%%%%DEDICAT�RIA%%%%%%%%%%%%%%%%%%%%%%%%%%%%%%%%%%%%%%%%%%%%%%%%%%%%%%%%%%%%%

\thispagestyle{empty}

\hspace{1cm} \vspace{8.5cm}


\hspace{2.83cm}\textit{\Large{Dedico este trabalho a minha turma e aminha fam\'ilia, por conta do apoio durante sua produ\c{c}\~ao. }}
                   % dedicat�ria
%}
% ------------------------------------------------------------------------
%{
%
%%%%%%%%%%%%%%%%%%%%%%%%%%%%%%%%%%%%%%%%%%%%%%%%%%%%%%%INVOCA��O%%%%%%%%%%%%%%%%%%%%%%%%%%%%%%%%%%%%%%%%%%%%%%%%%%%%%%%%%%%%%

\thispagestyle{empty}

%\vspace{20cm}

\noindent{AMO-TE TANTO, meu amor... n�o cante}\\
O humano cora��o com mais verdade...\\
Amo-te como amigo e como amante\\
Numa sempre deversa realidade.\\

 \vspace{0.5 cm}

\noindent{Amo-te afim, de um calmo amor prestante,}\\
E te amo al�m, presente na saudade.\\
Amo-te, enfim, com grande liberdade\\
Dentro da eternidade e a cada instante.\\

 \vspace{0.5 cm}

\noindent{Amo-te como um bicho, simplesmente,}\\
De um amor sem mist�rio e sem virtude\\
Com um desejo maci�o e permanente.\\

 \vspace{0.5 cm}

\noindent{E de te amar assim muito e ami�de,}\\
� que um dia em teu corpo de repente\\
Hei de morrer de amar mais do que pude.\\



(Vin�cius de Moraes, \textit{Soneto do amor total})
                   % invoca��o
%}
% ------------------------------------------------------------------------
%{
%\typeout{Acknowledgements}

%%%%%%%%%%%%%%%%%%%%%%%%%%%%%%%%%%%%%%%%%%AGRADECIMENTOS%%%%%%%%%%%%%%%%%%%%%%%%%%%%%%%%%%%%%%%%%%%%%%%%%%%%%%%%%%%%%%%%%%%%
\pagestyle{fancy}
%\pagestyle{fancy}
\fancyhead[C]{\textsl{Agradecimentos}}
\fancyhead[R]{7}
\fancyfoot[C]{}	


\chapter*{Agradecimentos}

\vspace{1.5cm}


\noindent Agrade�o neste trabalho primeiramente a Deus que permitiu que tudo isso acontecesse, ao longo de minha vida, e n�o somente nestes anos como universit�rio, mas que em todos os momentos � o maior mestre que algu�m pode conhecer.Agrade�o a todos os professores por me proporcionar o conhecimento n�o apenas racional, mas a manifesta��o do car�ter e afetividade da educa��o no processo de forma��o profissional, por tanto que dedicaram a mim, n�o somente por terem me ensinado, mas por terem me feito aprender. A palavra mestre, nunca far� justi�a aos professores dedicados aos quais sem nominar ter�o os meus eternos agradecimentos. Agrade�o a minha m�e Raquel, hero�na que me deu apoio, incentivo nas horas dif�ceis, de des�nimo e cansa�o e a todos que direta ou indiretamente fizeram parte da minha forma��o, o meu muito obrigado.\\
                   % agradecimentos
%}
% -----------------------------------------------------------------------
{ \typeout{Abstract}

%%%%%%%%%%%%%%%%%%%%%%%%%%%%%%%%%%%%%%%RESUMO%%%%%%%%%%%%%%%%%%%%%%%%%%%%%%%%%%%%%%%%%%%%%%%%%%%%%%%%%%%%%%%%%%%%%%%%%%%%%%%%

\thispagestyle{empty}

\hspace{1cm} \vspace{2.2cm}

\noindent {\Huge {\bf Resumo}}

\vspace{1.5cm}

\noindent O presente relat�rio exp�e o resultado parcial do projeto para TCC sobre o algoritmo de criptografia RSA gaussiano.
                    % resumo
}
% ------------------------------------------------------------------------

\setcounter{page}{1}
\tableofcontents                  % cria o �ndice

% ------------------------------------------------------------------------
{ 				
\pagestyle{fancy}
\fancyhead[C]{\textsl{Introdu��o}}
\fancyhead[R]{\thepage}
\fancyfoot[C]{}
\fancyhead[L]{}

\addcontentsline{toc}{chapter}{Introdu��o} % insere no sum�rio a introdu��o

\chapter*{Introdu\c{c}\~{a}o}
\label{Intro}

O sigilo sempre foi uma arma explorada pelos seres humanos para vencer certas batalhas, e at\'e mesmo para a cotidiana miss\~{a}o de se comunicar. Foi a partir dessa necessidade que se criou a \textit{criptografia}, nome dado ao conjunto de t\'ecnicas usadas para se  comunicar em c\'odigos. Seu objetivo \'{e} garantir que apenas os envolvidos na comunica\c{c}\~ao possam compreender a mensagem codificada (ou criptogtafada), garantindo que terceiros n\~ao saibam o que foi conversado.

Para compreender como funciona o processo de codifica\c{c}\~ao e decodifica\c{c}\~ao faz-se necess\'ario o uso de uma s�rie de termos t\'ecnicos, e para fins pedag�gicos iremos introduzir tais conceitos apresentando um dos primeiros algoritmos criptogr\'aficos que se tem conhecimento, a criptografia de C\'esar. Para mais detalhes sobre o tema, veja Criptografia, por Coutinho\cite{coutinho}.

A chamada \textit{criptografia de C\'esar}, criada pelo imperador romano C\'esar Augusto, consistia em substituir cada letra da mensagem por outra que estivesse a tr\^es posi\c{c}\~oes a frente, como, por exemplo, a letra \textbf{A} que neste algoritmo \'e substitu\'ida pela letra \textbf{D}.  

Uma forma muito natural de se generalizar o algoritmo de C\'esar \'e fazer a troca de cada letra da mensagem por outra que venha em uma posi\c{c}\~ao qualquer fixada. A chamada \textit{criptografia de substitui\c{c}\~ao monoalfab\'etica} consiste em substituir cada letra por outra que ocupe $n$ posi\c{c}\~oes � sua frente, sendo que o n\'umero $n$ \'e conhecido apenas pelo emissor e pelo receptor da mensagem. O n\'umero $n$ \'e a \textit{chave criptogr\'afica}. Para decifrar a mensagem, precisamos substituir as letras que formam a mensagem criptografada pelas letras que est\~ao $n$ posi\c{c}\~oes antes.

O algoritmo monoalfab\'etico tem a caracter\'istica indesejada de ser de f\'acil decodifica\c{c}\~ao, pois possui apenas {26} chaves poss\'iveis, e isso faz com que no m\'aximo em {26} tentativas o c\'odigo seja decifrado. Com o intuito de dificultar a quebra do c\'odigo monoalfab\'etico foram propostas as \textit{cifras de substitui\c{c}\~ao polialfab\'eticas} em que a chave criptogr\'afica passa a ser uma \textit{palavra} ao inv\'es de um n\'umero. A ideia \'e usar as posi\c{c}\~oes ocupadas pelas letras da chave para determinar o n\'umero de posi\c{c}\~oes que devemos avan\c{c}ar para obter a posi\c{c}\~ao da letra encriptada. Vejamos, por meio de um exemplo, como funciona esse sistema criptogr\'afico.

Sejam ``SENHA'' a nossa chave criptogr\'afica e ``ABOBORA'' a mensagem a ser encriptada. Abaixo colocamos as letras do alfabeto com suas respectivas posi\c{c}\~oes. Observe que repetimos a primeira linha de letras para facilitar a localiza\c{c}\~ao da posi\c{c}\~ao da letra encriptada e usamos a barra para indicar que estamos no segundo ciclo. 

\[
\begin{array}{ccccccccccccc}
    1      & 2 & 3 & 4 & 5          & 6 & 7 & 8          & 9 & 10 &  11 & 12 & 13 \\  
\textbf{A} & B & C & D & \textbf{E} & F & G & \textbf{H} & I & J  &  K  & L  & M  \\ 
  &   &   &   &   &   &   &   &   &    &     &    &    \\ 
    14      & 15 & 16 & 17 & 18 & 19          & 20 & 21 & 22 & 23 & 24 & 25 & 26 \\
\textbf{N}  & O  & P  & Q  & R  & \textbf{S}  & T  & U  & V  & X  & Y  & W  & Z \\
&   &   &   &   &   &   &   &   &    &     &    &    \\ 
    27      & 28 & 29 & 30 & 31          & 32 & 33 & 34          & 35 & 36 &  37 & 38 & 39 \\  
\overline{A} & \overline{B} & \overline{C} & \overline{D} & \overline{E} & \overline{F} & \overline{G} & \overline{H} & \overline{I} & \overline{J}  &  \overline{K}  & \overline{L}  & \overline{M}  \\
\end{array}
\]

Vejamos como encriptar a palavra ``ABOBORA''. Iniciamos o processo escrevendo a mensagem. Ao lado de cada letra da mensagem aparece entre par\^enteses o n\'umero que indica a sua posi\c{c}\~ao. Abaixo da mensagem escrevemos as letras da chave criptogr\'afica, repetindo-as de forma c\'iclica quando necess\'ario. Analogamente, ao lado de cada letra da chave aparece entre par\^enteses o n\'umero da posi\c{c}\~ao ocupada de cada letra, e o sinal de soma indica que devemos avan\c{c}ar aquele n�mero de posi��es. Ao final do processo aparecem as letras encriptadas. Entre par�nteses est� a posi\c{c}\~ao resultante da combina\c{c}\~ao das posi\c{c}\~oes da mensagem e da chave.   

\footnotesize{
\[
\begin{array}{lllllll||l}
     A (1)  &      B (2)  &      O (15) &      B (2)  &      O (15) &     R (18)  &    A (1)	 & \textrm{Mensagem}  \\
\downarrow  & \downarrow  & \downarrow  & \downarrow  & \downarrow  & \downarrow  & \downarrow &\\ 
    S (+19) &     E (+5)  &     N (+14) &     H  (+8) &     A (+1)  &    S (+19)  &   E (+5)   &\textrm{Chave}  \\
\downarrow  & \downarrow  & \downarrow  & \downarrow  & \downarrow  & \downarrow  & \downarrow & \\
		 T (20) &      G (7)  &      C (29) &      J (10) &     P (16)  &    K (37)   &    F (6)   & \textrm{Mensagem encriptada}  \\
\end{array}
\]
}
  
Observe que a encripta\c{c}\~ao polialfab\'etica \'e mais dif\'icil de ser quebrada que a monoalfab\'etica uma vez que letras iguais n\~ao t\^em, necessariamente, a mesma encripta\c{c}\~ao. Neste tipo de criptografia o emissor precisa passar a chave para o receptor da mensagem de forma segura para que o receptor possa decifrar a mensagem, isto \'e, a chave usada para encriptar a mensagem \'e a mesma que deve ser usada para decifrar a mensagem. Veremos que esse \'e justamente o ponto fraco neste tipo de encripta\c{c}\~ao pois usa a chamada \textit{chave sim\'etrica}, ou seja, a chave usada pelo emissor para codificar a mensagem \'e a mesma usada pelo receptor para decodificar a mensagem. Nesse processo, a chave deve ser mantida em segredo e bem guardada para garantir que o c\'odigo n\~ao seja quebrado, e isso requer algum tipo de contato f\'isico entre emissor e receptor da mensagem.

Durante a  Primeira Guerra Mundial o contato f\'isico para a troca de chaves era complicado, e isso estimulou a cria\c{c}\~ao de m\'aquinas autom\'aticas de criptografia. O \textit{Enigma} foi uma dessas m\'aquinas e era utilizada pelos alem\~aes tanto para criptografar como para descriptografar c\'odigos de guerra. Semelhante a uma m\'aquina de escrever, os primeiros modelos foram patenteados por Arthur Scherbius em 1918. Essas m\'aquinas ganharam popularidade entre as for\c{c}as militares alem\~as devido � facilidade de uso e sua suposta indecifrabilidade do c\'odigo. 

O matem\'atico Alan Turing foi o respons\'avel por quebrar o c\'odigo dos alem\~aes durante a Segunda Guerra Mundial. A descoberta de Turing mostrou a fragilidade da criptografia baseada em chave sim\'etrica e colocou novos desafios \`a criptografia. O grande problema passou a ser a quest\~ao dos protocolos, isto \'e, como transmitir a chave para o receptor de forma segura sem que seja necess\'ario o contato f\'isico entre as partes? 

Em 1949, com a publica\c{c}\~ao do artigo \textit{Communication Theory of Secrecy Systems} \cite{shannon} de Shannon, temos a inaugura\c{c}\~ao da criptografia moderna. Neste artigo ele escreve matematicamente que cifras teoricamente inquebr\'aveis s\~ao semelhantes �s cifras polialfab\'eticas. Com isso ele transformou a criptografia que at\'e ent\~ao era uma arte, em uma ci\^encia.

Em 1976 Diffie e Hellman publicaram \textit{New Directions in Cryptography} \cite{newdirections}. Neste artigo h\'a a introdu\c{c}\~ao ao conceito de \textit{chave assim\'etrica}, onde h\'a chaves diferentes entre o emissor da mensagem e seu receptor. Com a assimetria de chaves n\~ao era mais necess\'ario um contato t\~ao pr\'oximo entre emissor e receptor. Neste mesmo artigo \'e apresentado o primeiro algoritmo de criptografia de chave assim\'etrica ou como \'e mais conhecido nos dias atuais \textit{Algoritmo de Criptografia de Chave P\'ublica}, o protocolo de Diffie-Hellman.

Um dos algoritmos mais famosos da criptografia de chave p\'ublica \'e o \textit{RSA} \cite{rivest}, algoritmo desenvolvido por Rivest, Shamir e Adleman. Este algoritmo se tornou popular por estar presente em muitas aplica\c{c}\~oes de alta seguran\c{c}a, como bancos, sistemas militares e servidores de internet.

Para que se possa compreender por completo o algoritmo faz-se necess�rio possuir alguns conhecimentos em teoria de n\'umeros como fatora\c{c}\~ao e aritm\'etica modular. Estes conhecimentos ser\~ao apresentados mais adiante neste trabalho.

No algoritmo RSA existe uma chave p\'ublica $n$, que \'e a multiplica\c{c}\~ao dos primos $p$ e $q$. O emissor E codifica a mensagem usando um n\'umero primo $p$. Em seguida E envia publicamente a mensagem codificada junto com a chave $n$ para o receptor R. R possui o n\'umero $q$, que juntamente ao n\'umero $n$ servem para decodificar a mensagem. 

Embora a quebra do RSA seja aparentemente simples, bastando fatorar $n$ para descobrir seus fatores, o grande problema \'e na realidade computacional, pois usa-se como $p$ e $q$ n\'umeros primos muito altos, pr\'oximos a $2^{512}$. Com um n\'umero t\~ao alto um computador comum levaria bem mais que uma vida humana para decifrar a mensagem.

Com base nestes conhecimentos sobre criptografia, temos que o objetivo deste trabalho \'e analisar a viabilidade de uma criptografia inspirada pelo algoritmo RSA cl\'assico, a qual substitui os n\'umeros primos pelo conjunto denominado de \textit{primos de Gauss} \cite{intGauss}, resultando, assim, no que chamamos por \textit{criptografia RSA gaussiana}. Para que tal algoritmo seja vi\'avel \'e necess\'ario adaptar uma s\'erie de resultados relativas aos n\'umeros primos aos n\'umero primos de Gauss. Dessa forma, nossa tarefa ser\'a adaptar tanto quanto o poss\'ivel os primos de Gauss \`as demosnta\c{c}\~oes desses teoremas.

Como se trata de uma proposta inovadora, deixamos para trabalhos futuros uma an\'alise comparativa entre as criptografias RSA cl\'assica e a RSA gaussiana.        
\chapter {Primos e Fatora\c{c}\~ao}
\label{Num}

\section{Ciclos e Restos}	
\subparagraph{
Para podermos compreender a aritm\'etica modular, precisamos come\c{c}ar entendendo o conceito de ciclicidade, que s\~ao os fatos que ocorrem sempre ap\'os um determinado per\'iodo constante. Um bom exemplo deste conceito \'e o nascer do sol, que \'e um evento que ocorre sempre ap\'os um ciclo de {24} horas, assim como o dia de seu anivers\'ario ocorre uma vez a cada ciclo de um ano.
}
\subparagraph{
O mesmo tipo de evento \'e observado com o resto dos n\'umeros inteiros. Tomemos por exemplo os restos de divis\~ao dos n\'umeros inteiros, abaixo mostrados de 1 \`a 12, pelo n\'umero inteiro {4}:
}

\[
\begin{array}{ccccccccccccc}
  {Inteiro} & 1 & 2 & 3 & 4 & 5 & 6 & 7 & 8 & 9 & 10 &  11 & 12 \\  
	{Resto} & 1 & 2 & 3 & 0 & 1 & 2 & 3 & 0 & 1 & 2  &  3 & 0 \\ 
\end{array}
\]

\subparagraph{
\'E vis\'ivel que ap\'os {4} n\'umeros o resto tende a se repetir. O mesmo feito ocorre a qualquer n\'umero inteiro $n$, onde o ciclo se repetir\'a sempre a cada $n$ itera\c{c}\~oes. Os n\'umeros que apresentam o resto {0} s\~ao conhecidos como m\'ultiplos de $n$.
}

\section{N\'{u}meros Primos e Compostos}

\subparagraph{
Existe um tipo especial de n\'umero que s\'o \'e m\'ultiplo, ou seja, possui resto {0}, em duas condi\c{c}\~oes, quando $n$ \'e igual a {1} ou quando ele \'e igual a $n$. A esse conjunto de n\'umeros atribui-se o nome de \textit{n\'umeros primos}.
}
\subparagraph{
\textit{Existem infinitos n\'umeros primos}, caso n\~ao acredite vamos supor que o conjunto finito de primos seja composto por $p_{1},  p_{2}, ..., p_{r} $. Considerando que o n\'umero inteiro $n=(p_{1})(p_{2})...(p_{r}) + 1$. $n$ deve possuir um fator $p$, que est\'a contido em $p_{1},  p_{2}, ..., p_{r} $, mas isso significa q $p$ divide $1$, o que \'e absurdo e prova que o conjunto n\~ao tem fim.
}
\subparagraph{
Todo o n\'umero que n\~ao \'e primo \'e chamado de \textit{N\'umero Composto}, sendo que este n\'umero composto pode ser escrito em \textit {uma combina\c{c}\~ao \'unica de fatores primos}. O processo de se descobrir estes fatores \'e chamado de \textit{fatora\c{c}\~ao} e \'e detalhado na pr\'oxima seção.
}

\section{Fatora\c{c}\~{a}o}

\subparagraph{
Anteriormente falamos que todo o n\'umero pode ser escrito por uma combina\c{c}\~ao de fatores primos, neste cap\'itulo vamos abordar como se pode obter estes fatores.
}
\subparagraph{
Come\c{c}amos por escolher o n\'umero inteiro $n$ ao qual iremos fatorar, em seguida testamos a sua divisibilidade por $2$, se for tente divid\'i-lo novamente por $2$, sen\~ao passa-se para o pr\'oximo n\'umero primo, o $3$. Repete-se esse procedimento at\'e chegarmos a $\sqrt{n}$, caso n\~ao achemos nenhum fator primo at\'e $\sqrt{n}$, $n$ \'e primo.
}
\subparagraph{
Quando acabamos de realizar a fatora\c{c}\~ao, chegamos a um n\'umero fatorado da forma $n = (2^{a_{1}})(3^{a_{2}}) ... (p^{a_{p}})$, todo o n\'umero inteiro pode ser escrito nessa forma, chamada forma fatorada, veja, por exemplo o $12 = (2^2)(3^1)$ e o $19 = (19^1)$.
}
\subparagraph{
Essa forma fatorada nos \'e formalmente apresentada pelo \textit{Teorema da Fatora\c{c}\~ao \'Unica}. Ele nos diz que dado um n\'umero inteiro $n\geq2$ pode-se escrev\^e-lo de forma \'unica como:
}
\[	
	\begin{array}{c}
		\textit{$n = (p^{e_{1}}_{1}) ... (p^{e_{k}}_{k}) $}
	\end{array}
\]
\paragraph{
onde $1 < p_1 < ... < p_k $ s\~ao primos e $e_1, ..., e_k$ s\~ao inteiros.
}
\subparagraph{
Mesmo algoritmo da fatora\c{c}\~ao sendo t\~ao simples de se compreender, ele \'e demorado at\'e para os mais modernos computadores. Para se ter uma ideia disto, um computador comum executa cerca de {50} divis\~oes por segundo, para se calcular com certeza que um n\'umero pr\'oximo a $10^{100}$ \'e primo ele levaria cerca de {317} decilh\~oes de anos. Essa demora computacional que torna os primos t\~ao atraentes a criptografia, pois sua multipli\c{c}\~ao \'e f\'acil para se obter o resultado, mas muito complexa para que se descubram quais os n\'umeros envolvidos nela apenas com o resultado final.
}


\chapter {Inversos Modulares}
\label{InvMod}

\section{Inversos modulares}	
\subparagraph{
Nosso objetivo com o decorrer deste cap\'itulo \'e o de explicar a opera\c{c}\~ao matem\'atica mais importante para para o algoritmo RSA. Para podermos comprend\^e-la vamos relembrar do cenceito ensinado no col\'egio de inverso multiplicativo, que consiste em obter o n\'umero que multiplicado a um n\'umero $n$ qualquer resulte em $1$. A opera\c{c}\~ao do inverso modular parte do mesmo princ\'ipio.
}
\subparagraph{
Vamos supor que queremos obter o inverso modular de $6$ para o m\'odulo $7$, o que n\'os teremos que fazer ent\~ao \'e encontrar qual o n\'umero que multiplicado por $6$ tem resto $1$ quando dividido por $7$. Come\c{c}amos pelo $1$, teremos que $6 \cdot 1 = 6$, $6 \equiv 6 (mod7)$. Com $2$ o resultado ser� $12$, logo $12 \equiv 5 (mod7)$, que para n\'os tamb\'em n\~ao serve. Tentando o $3$ obtemos $4$ e com $4$ obtemos $3$. Com o $5$ nosso retorno ser\'a $2$. Finalmente quando chegamos ao $6$ n\'os temos que $6 \cdot 6 = 36$, $36 \equiv 1 (mod 7)$. Com isso podemos concluir que o inverso multiplicativo de $6$ no m\'odulo $7$ \'e o pr\'opio $6$.
}
\subparagraph{
Para simplificar o que foi dito acima, podemos dizer a opera\c{c}\~ao de inverso multiplicativo no m\'odulo $n$ para $a$ consiste em encontar um n\'umero $a'$ tal que:
}
\[	
	\begin{array}{c}
		\textit{$a \cdot a' \equiv 1 (mod n)$}
	\end{array}
\]

\section{Inexist\^encia e exist\^encia de inversos}	

\subparagraph{
Antes de come\c{c}armos vamos tentar calcular o inverso multiplicativo de $2$ no m\'odulo $8$, vamos l\'a: 
}
\[	
	\begin{array}{c}
		\textit{$2 \cdot 0 \equiv 0 \not\equiv 1(mod 8)$}\\
		\textit{$2 \cdot 1 \equiv 2 \not\equiv 1(mod 8)$}\\
		\textit{$2 \cdot 2 \equiv 4 \not\equiv 1(mod 8)$}\\
		\textit{$2 \cdot 3 \equiv 6 \not\equiv 1(mod 8)$}\\
		\textit{$2 \cdot 4 \equiv 8 \not\equiv 1(mod 8)$}\\
		\textit{$2 \cdot 5 \equiv 0 \not\equiv 1(mod 8)$}\\
		\textit{$2 \cdot 6 \equiv 2 \not\equiv 1(mod 8)$}\\
		\textit{$2 \cdot 7 \equiv 4 \not\equiv 1(mod 8)$}\\
	\end{array}
\]

\subparagraph{
N\~ao encontramos nenhuma resposta pois, simplesmente, n\~ao h\'a. Antes que se pergunte o motivo de n\~ao tentarmos com n\'umeros maiores que $7$, \'e v\'alido lembrar que a partir do $8$ ter\'iamos a repeti\c{c}\~ao de resultados por conta das congru\^encias.
}
\subparagraph{
A opera\c{c}\~ao de inverso multiplicativo s\'o possui resultado em casos onde o n\'umero $a$ ao qual queremos calcular o inverso e o m'odulo s\~ao \textit{primos entre si}, ou seja, n\~ao possuam nenhum fator em comum. Por conta disso usamos os n\'umeros primos no algoritmo RSA.
}
\subparagraph{
Para comprovar o que foi dito acima, vamos tomar um n\'umero $a$, tal que
}
\[	
	\begin{array}{c}
		\textit{$a \cdot a' \equiv 1(mod n)$}\\
	\end{array}
\]
\paragraph{
isso pode ser traduzido em linguagem humana como $n$ divide $ a \cdot a' - 1$. Isso em linguajar matem\'atico pode ser escrito como:
}
\[	
	\begin{array}{c}
		\textit{$a \cdot a' - 1 = n \cdot k$}\\
	\end{array}
\]
\paragraph{
como estamos atr\'as de saber se $a$ e $n$ n\~ao possuem fator comum, ent�o h\'a de haver um $k$ inteiro para a equa\c{c}\~ao acima. Nosso primeiro passo para provar isso ser\'a de se criar o conjunto $V(a,n)$, esse conjunto \'e formado por inteiros positivos e pode ser escrito como
}
\[	
	\begin{array}{c}
		\textit{$x \cdot a + y \cdot n$}\\
	\end{array}
\]
\subparagraph{
Em um primeiro momento este conjunto e esta nova f\'ormula podem parecer estranhos ao que se via antes, mas se comprovarmos que $ 1 \in V(a,n)$, conclu\'imos que devem haver dois inteiros $x_0$ e $y_0$, ou se preferir $a'$ e $k$, logo:
\[	
	\begin{array}{c}
		\textit{$1 = a \cdot a' - n \cdot k$}\\
	\end{array}
\]
}
\subparagraph{
Uma das propiedades deste conjunto \'e a de $n$ pertencer a ele quando $x = a' = 0$ e $y = k = 1$. Isto significa que os inteiros que podem completar a equa\c{c}\~ao est\~ao entre $1$ e $n$. Mas para podermos dar essa demonstra\c{c}\~ao como completa, precisamos provar que $m = 1$.
}

\subparagraph{
Estou achando que est\'a confuso, pe�o que marque muito nesse final
}
\chapter {Teorema chin\^es do resto}
\label{TCR}

\section{Introdu\c{c}\~ao a t\'ecnica}

\subparagraph{
Para sermos iniciados nesta t\'ecnica, vamos analisar o seguinte problema: Qual o menor inteiro que possui resto $1$ na divis\~ao por $3$ e resto $2$ na divis\~ao por $5$. Podemos vir a tranformar esse problema nas seguintes equa\c{c}\~oes:
}
\[	
	\begin{array}{c}
		\textit{$n = 3q_1 + 1$ e $n = 5q_2 + 2$}
	\end{array}
\]
\paragraph{
Essas equa\c{c}\~oes tamb\'em podem ser denotadas em forma modular como:
}
\[	
	\begin{array}{c}
		\textit{$n \equiv 1 (mod 3)$ e $n \equiv 2 (mod 5)$}
	\end{array}
\]
\paragraph{
Essa sa\'ida modular nos deixou com apenas uma vari\'avel, mas ainda n\~ao resolveu ao nosso problema. Para fazermos isso vamos substituir $n$ por $5q_2 + 2$, montando a seguine equa\c{c}\~ao modular:
}
\[	
	\begin{array}{c}
		\textit{$5q_2 + 2 \equiv 1 (mod 3)$}
	\end{array}
\]
\paragraph{
Como $5 \equiv 2(mod 3)$, substitu\'imos:
}
\[	
	\begin{array}{c}
		\textit{$ 2q_2 + 2 \equiv 1 (mod 3)$}
	\end{array}
\]
\paragraph{
Feito isso, passamos $2$ para o outro lado da equa\c{c}\~ao
}
\[	
	\begin{array}{c}
		\textit{$ 2q_2  \equiv -1 (mod 3)$}
	\end{array}
\]
\paragraph{
Como $-1 \equiv 2 (mod 3)$, n\'os substit\'imos novamente, e depois dividimos a equa\c{c}\~ao por $2$, e obtemos
}
\[	
	\begin{array}{c}
		\textit{$ q_2  \equiv 1 (mod 3)$}
	\end{array}
\]
\paragraph{
Com isso, conclu\'imos que
}
\[	
	\begin{array}{c}
		\textit{$ q_2  \equiv q_3 + 1 (mod 3)$}
	\end{array}
\]
\paragraph{
Sei que parece que mais uma equa\c{c}\~ao s\'o serve para tornar a resolu\c{c}\~ao mais complexa, mas vamos a reorganizar como
}
\[	
	\begin{array}{c}
		\textit{$ q_2 = 3q_3 + 1 $}
	\end{array}
\]
\paragraph{
Agora substitu\'imos
}
\[	
	\begin{array}{c}
		\textit{$n = 5(3q_3 + 1) + 2 = 15q_3 +7$}
	\end{array}
\]
\paragraph{
Feito isso, vamos por o $3$ em evid\^encia em todos os lugares, obtendo:
}
\[	
	\begin{array}{c}
		\textit{$n = 3(5q_3) +3(2) +1 = 3(5q_3 +2)+1$}
	\end{array}
\]
\paragraph{
Este procedimento foi feito apenas para provar que a equa\c{c}\~ao deixa resto 1 se dividida por 3, de forma an\'aloga, abaixo \'e mostrado como ela deixa resto $2$ quando dividida por $5$.
}
\[	
	\begin{array}{c}
		\textit{$n = 5(3q_3) +5(1) +2 = 5(3q_3 +1)+2$}
	\end{array}
\]
\subparagraph{
Ap\'os tudo isso feito ainda n\~ao possu\'imos a solu\c{c}\~ao final, mas j\'a sabemos que \'e um n\'umero da forma $15q_3 + 7$, substituindo $q_3$ or $0$, iremos obter $7$, que \'e o resultado procurado.
}

\section{Provas ao teorema}

\subparagraph{
O teorema chin\^es do resto \'e um procedimento tomado para resolver sistema de congru\^encias, como o descrito acima. Ele foi descrito pela primeira vez pelo Manual de aritm\'etica do mestre Sun, por volta do s\'eculo III d.C. 
}
\subparagraph{
Para ver a defini\c{c}\~ao formal desse teorema, vamos considerar o sistema
}
\[	
	\begin{array}{c}
		\textit{$x \equiv a (mod n)$}\\
		\textit{$x \equiv b (mod m)$}\\
	\end{array}
\]
\paragraph{
nele, $n$ e $m$ s\~ao inteiros diferentes entre si. Tomemos $x_0$ como um n\'umero cappaz de satisfazer ambas as congru\^encia de forma simult\^anea e teremos:
}
\[	
	\begin{array}{c}
		\textit{$x_0 \equiv a (mod m)$}\\
		\textit{$x_0 \equiv b (mod n)$}\\
	\end{array}
\]
\paragraph{
Para podermos juntar ambas as equa\c{c}\~oes converteremos uma em equa\c{c}\~ao, nesse caso teremos 
}
\[	
	\begin{array}{c}
		\textit{$x_0 = a + m\cdot k$, com $k$ sendo um inteiro qualquer}\\
	\end{array}
\]
\paragraph{
Feito isso, chegaremos em
}
\[	
	\begin{array}{c}
		\textit{$a + m\cdot k \equiv b (mod n)$}\\
	\end{array}
\]
\paragraph{
que pode ser substitu\'ida por
}
\[	
	\begin{array}{c}
		\textit{$ m\cdot k \equiv (b-a) (mod n)$}\\
	\end{array}
\]
\subparagraph{
Agora vamos supor que $m$ e $n$ s\~ao primos entre si. Pelo teorema apresentado no cap\'ituo sobre inversos multiplicativos n\'os j\'a sabemos que eles possuem inverso multiplicativo um para o outro. Tomemos $m'$ como o inverso de $m$ no m\'odulo $n$. Multipplicando toda a congru\^encia por $m'$ obtemos
}
\[	
	\begin{array}{c}
		\textit{$ k \equiv m'\cdot(b-a) (mod n)$}\\
	\end{array}
\]
\paragraph{
que pode ser escrita como:
}
\[	
	\begin{array}{c}
		\textit{$ k \equiv m'\cdot(b-a)+n \cdot t$, para um inteiro $t$ qualquer}\\
	\end{array}
\]
\subparagraph{
Substituindo a parte de $k$, n\'os obtemos
}
\[	
	\begin{array}{c}
		\textit{$ x_0 \equiv a + m (m'\cdot(b-a)+n \cdot t) $}\\
	\end{array}
\]
\paragraph{
Podemos ver agora que para qualquer $t$, $a + m (m'\cdot(b-a)+n \cdot t$ \'e parte da solu\c{c}\~ao da congru\^encia, sabendo disso, agora podemos descrever o teorema em si, que ser\'a feito na pr\'oxima se\c{c}\~ao.
}

\section{O teorema chin\^es do resto}
\subparagraph{
\textit{Teorema chin\^es do resto} - Sejam $m$ e $n$ inteiros positivos primos entre si. Se $a$ e $b$ s\~ao inteiros quaisquer, ent\~ao o sistema
}
\[	
	\begin{array}{c}
		\textit{$ x \equiv a (mod m) $}\\
		\textit{$ x \equiv b (mod n) $}
	\end{array}
\]
\paragraph{
sempre tem solu\c{c}\~ao e qualquer uma de suas solu\c{c}\~oes pode ser escrita na forma
}
\[	
	\begin{array}{c}
		\textit{$ a + m \cdot(m' \cdot (b-a) + n \cdot n) $}\\
	\end{array}
\]
\paragraph{
onde $t$ \'e um inteiro qualquer e $m'$ \'e o inverso de $m$ no m\'odulo $n$.
}
\chapter {Potencia\c{c}\~ao}
\label{Pot}
\subparagraph{
Ao longo deste cap\'itulo vamos estudar como tornar as opera\c{c}\~oes de potencia\c{c}\~ao e a obten\c{c}ao de seus restos calcal\'aveis de forma simpls e r\'apida. Para isso vamos dispor de algumas artimanhas matem\'aticas, como o famoso \textit{Teorema de Fermat} e o Teorema chin\^es do resto.
}
\section{Restos na potencia\c{c}\~ao}	
\subparagraph{
Vamos come�ar tentando uma coisa que aparentemente \'e complexa, mas se converter\'a em uma opera\c{c}\~ao bem simples: Calcular o resto da divis\~ao de $10^{135}$ por $7$. Podemos fazer da forma tradicional, mas dividir um n\'umero t\~ao alto n\~ao seria nada pr\'atico.
}
\subparagraph{
O que faremos \'e tomar uma propiedade da multiplica\c{c}\~ao e da potencia\c{c}\~ao emprestadas, a do elemento neutro, nesse caso o $1$. O que faremos \'e calcular em qual pot\^encia $10$ \'e congruente a $1$ no m\'odulo $7$. Logo teremos a tabela:
}
\[
\begin{array}{c}
  \textit{$10^1 \equiv 3(mod7)$} \\  
	\textit{$10^2 \equiv 2(mod7)$}\\
	\textit{$10^3 \equiv 6(mod7)$}\\ 
	\textit{$10^4 \equiv 4(mod7)$}\\ 
	\textit{$10^5 \equiv 5(mod7)$}\\ 
	\textit{$10^6 \equiv 1(mod7)$}\\ 
\end{array}
\]
\subparagraph{
Como sabemos agora que $10^6$ \'e o n\'umero que quer\'iamos, vamos decompor o $135$ em raz\~ao de $6$ e teremos que $135 = (6 \cdot 22)+3$, essa express\~ao nos levar\'a a seguinte congru\^encia:
}
\[
\begin{array}{c}
  \textit{$10^{135} \equiv (10^6)^{22} \cdot 10^3 \equiv (1)^{22} \cdot 10^3 \equiv 10^3 \equiv 6 (mod7)$} \\  
\end{array}
\]
\subparagraph{
Agora n\'os vamos deixar essa opera\c{c}\~ao um pouco mais complexa, ao passo que vamos calcular o resto por $31$ de $2^{124512}$. Vamos pelo mesmos caminho que anteriormente, buscando a pot\^encia de $2$ que \'e congruente a $1$ no m\'odulo $31$. Obtemos:
}
\[
\begin{array}{c}
  \textit{$2^1 \equiv 2(mod31)$} \\  
	\textit{$2^2 \equiv 4(mod31)$}\\
	\textit{$2^3 \equiv 8(mod31)$}\\ 
	\textit{$2^4 \equiv 16(mod31)$}\\ 
	\textit{$2^5 \equiv 1(mod31)$}\\ 
\end{array}
\]
\subparagraph{
Vamos dividir $124512$ por $5$ e obteremos $4016$ com resto $2$, obtendo assim que $2^{124512} \equiv 2^2 \equiv 4 (mod31)$.
}
\subparagraph{
Tornando um pouco mais dif\'icil, podemos calcular o resto de $2^{13}^{98765}$, \'e descobrir o resto de ${13}^{98765}$ por $5$, podemos dizer que ${13}^{98765} \equiv {3}^{98765}(mod 5)$ como se sabe que $3^4 = 81 \equiv 1(mod 5)$, podemos usar isso em nosso favor, pois teremos ${3}^{98765}\equiv {3}^{4\cdot24691 + 1}\equiv 3 (mod 5)$, logo o resultado de ${13}^{98765}$ \'e um n\'umero da forma $5q'+3$. Como isso, n\'os podemos dizer que $2^{13}^{98765} \equiv 2^{5q'+3} \equiv 2^{5q'}\cdot{2^3}\equiv {1}^{q'}\cdot{2^3} \equiv 8 (mod 31)$.
}
\section{O teorema de Fermat}
\subparagraph{
	\textit{Teorema de Fermat} - Se $p$ \'e um n\'umero primo e $a$ \'e um inteiro n\~ao divis\'ivel por $p$, ent\~ao:
}
\[
\begin{array}{c}
  \textit{$a^{p-1}\equiv 1(mod p)$} \\  
\end{array}
\]
\subparagraph{
Embora esse seja denominado como o pequeno teorema de Fermat, ele possui suma import\^ancia para o algoritmo RSA cl\'assico. Vamos apresentar abaixo uma de suas demonstra\c{c}\~oes.
}
\subparagraph{
Sabemos que os poss\'iveis res\'iduos no m\'odulo $p$ s\~ao todos os inteiros entre $1$ e $p-1$. Vamos multiplic\'a-los por $a$, obtendo assim: 
}
\[
\begin{array}{c}
  \textit{$a \cdot 1,a \cdot 2,a \cdot 3,...,a \cdot (p-1)$} \\  
\end{array}
\]
\subparagraph{
Vamos levar em conta que $r_1 \equiv a\cdot 1(mod p)$,$r_2 = a\cdot 2(mod p)$, e assim por diante at\'e $r_{p-1} = a\cdot (p-1)(mod p)$. Tomemos par $r_k$ e $r_l$ um par de inteiros $k$ e $l$ que est\'a entre $1$ e $p-1$. Com isso teremos:
}
\[
\begin{array}{c}
  \textit{$a \cdot k \equiv k \equiv l \equiv a\cdot l (mod p)$} \\  
\end{array}
\]
\paragraph{
que equivale \`a:
}
\[
\begin{array}{c}
  \textit{$a \cdot k \equiv a\cdot l (mod p)$} \\  
\end{array}
\]
\subparagraph{
Se viermos a cancelar pela equival\^encia, obteremos que $k \equiv l (mod p)$, mas sendo $k$ e $l$ positivos, inteiros e menores que $p$, estes s� podem ser congruentes se forem iguais, logo se:
}
\[
\begin{array}{c}
  \textit{$r_k = r_l$ ent�o $k = l$} \\  
\end{array}
\]
\subparagraph{
Isto demonstra que $r_1, r_2, r_3,...r_{p-1}$ s�o $p-1$ res\'iduos n\~ao nulos de m\'odulo $p$, que ser�o $1, 2, 3, ..., p-1$, o que nos permite dizer que a primeira sequ�ncia n\~ao \'e nada al\'em de um reordenamento da segunda. Com isso podemos dizer que:
}
\[
\begin{array}{c}
  \textit{$r_1 \cdot r_2 \cdot r_3\cdot...\cdot r_{p-1} = 1 \cdot 2 \cdot 3\cdot ...\cdot p-1$} \\  
\end{array}
\]
\paragraph{
Sabendo disso vemos que:
}
\[
\begin{array}{c}
  \textit{$ a^{p-1}(1 \cdot 2 \cdot 3\cdot ...\cdot p-1) \equiv (1 \cdot 2 \cdot 3\cdot ...\cdot p-1) (mod p)$} \\  
\end{array}
\]
\paragraph{
E apenas cortando os fatores iguais:
}
\[
\begin{array}{c}
  \textit{$ a^{p-1} \equiv 1(mod p)$} \\  
\end{array}
\]
\paragraph{
Provando assim o Teorema de Fermat.
}
\section{Aplicando o teorema de Fermat}
\paragraph{
Antes de prosseguirmos para o pr\'oximo cap\'itulo, vamos utilizar o Teorema de Fermat para resolver uma congru\^encia. Neste caso vamos tentar descobrir quem \'e congruente a $3^{1034}^{2}$ no m\'odulo $1033$. Como $1033$ \'e primo n\'os podemos usar o teorema de Fermat. Neste caso teremos que:
}
\[
\begin{array}{c}
  \textit{$ 3^{1032} \equiv 1 (mod 1033)$} \\  
\end{array}
\]
\subparagraph{
O que faremos agora consiste em ``dividir'' $1034$ por $1032$, de forma a obter o resto da divis\~ao. e com isso vamos veirficar que:
}
\[
\begin{array}{c}
  \textit{$ 1034^2 \equiv 2^2 \equiv 4 (mod 1033)$} \\  
\end{array}
\]
\paragraph{
e com essa simplifica\c{c}\~ao chegamos \`a:
}
\[
\begin{array}{c}
  \textit{$ 3^{1034} \equiv 3^{1032}\cdot q + 4 \equiv (3^{1032})^{q} + 3^4 (mod 1033)$} \\  
\end{array}
\]
\subparagraph{
Agora com a simples aplica\c{c}\~ao do Teorema de Fermat, podemos chegar a conclus\~ao que: 
}
\[
\begin{array}{c}
  \textit{$ {3^{1034}}^2 \equiv 1 \cdot 81 (mod 1033)$} \\  
\end{array}
\]
\paragraph{
verificando assim que ${3^{1034}}^2$ deixa resto $81$ na divis\~ao por $1033$.
}
\section{Teorema de Fermat para pot\^encias compostas}
\subparagraph{
Embora aplicar o teorema de Fermat diretamente sobre os n\'umeros compostos n\~ao seja poss\'ivel, n\'os ainda podemos resolver a estas congru\^encias com o aux\'ilio do teorema chin\^es d restos, como veremos a seguir.
}
\subparagraph{
Para que possamos entender como resolver este problema com n\'umeros compostos vamos tentar resolver um problema n\'umerico, nesse caso o c\'alculo do m\'odulo de $2^{6754}$ por $1155$.
}
\subparagraph{
Nosso primeiro passo \'e fatorar o $1155$. Ao fim da fatora\c{c}\~ao vamos obter que $1155 = 3 \cdot 5 \cdot 7 \cdot 11$. Em seguida vamos aplicar o teorema de Fermat a cada um dos primos, obtendo assim:
}
\[
\begin{array}{c}
  \textit{$ 2^2 \equiv 1 (mod 3)$} \\  
	\textit{$ 2^4 \equiv 1 (mod 5)$} \\  
	\textit{$ 2^6 \equiv 1 (mod 7)$} \\  
	\textit{$ 2^{10} \equiv 1 (mod 11)$} \\  
\end{array}
\]
\subparagraph{
Agora dividimos $6754$ por $p-1$ para cada um dos m\'ultiplos:
}
\[
\begin{array}{c}
  \textit{$6754 = 2 \cdot 3377 $} \\  
	\textit{$6754 = 4 \cdot 1688 + 2$} \\  
	\textit{$6754 = 6 \cdot 1125 + 4$} \\  
	\textit{$6754 = 10 \cdot 675 + 4$} \\  
\end{array}
\]
\paragraph{
Em seguida substitu\'imos nas congru\^encias e as reduzimos
 }
\[
\begin{array}{c}
  \textit{$2^{6754} \equiv {2^{3377}}^{2} \equiv 1 (mod 3) $} \\  
	\textit{$2^{6754} \equiv {2^{1688}}^{4} \cdot 2^2 \equiv 1 \cdot 4 \equiv  4 (mod 5) $} \\  
	\textit{$2^{6754} \equiv {2^{1125}}^{6} \cdot 2^4 \equiv 1 \cdot 16\equiv  2 (mod 7) $} \\  
	\textit{$2^{6754} \equiv {2^{675}}^{10} \cdot 2^4 \equiv 1 \cdot 16\equiv  5 (mod 11) .$} \\  
\end{array}
\]
\paragraph{
Logo, nossa tarefa consiste em resolver o sistema
 }
\[
\begin{array}{c}
  \textit{$x \equiv 1 (mod 3) $} \\  
	\textit{$x \equiv 4 (mod 5) $} \\  
	\textit{$x \equiv  2 (mod 7) $} \\  
	\textit{$x \equiv  5 (mod 11)$} \\  
\end{array}
\]
\subparagraph{
Podemos resolver esse sistema usando o algoritmo chin\^es, vamos come�ar substituindo na primeira congru�ncia, onde $x = 3y + 1$, em seguida substitu\'imos $x$ por $y$ na segunda congru\^encia, tornando-a $3y + 1 \equiv 4 (mod 5)$, que equivale a $y \equiv 1 (mod 5)$ 
como $3$ \'e invers\'ivel no m\'odulo $5$ ele pode ser anulado na equa��o. Com isso temos $x = 4+15z$ que se substiutindo na terceira equa\c{c}\~ao e resolvendo obtemos $z \equiv 5 (mod 7)$, que significa que $x = 79 + 105t$. Finalmente substituindo na \'ultima equa\c{c}\~ao, teremos que $t \equiv 6 (mod 11)$, o que resulta em $x = 709+1155u$. Conclu�mos com isso que $26754 \equiv 709 (mod 1155)$. 
 }
\pagestyle{fancy}
\fancyhead[C]{\textsl{2. Sobre a Criptografia RSA}}
\fancyhead[R]{\thepage}
\fancyfoot[C]{}
\chapter {Aplicando a criptografia RSA}
\label{RSA}

A criptografia RSA tem suma import\^ancia para toda a comunica\c{c}\~ao moderna. Ela \'e t\~ao importante que a descoberta de uma forma de se desencript\'a-la colocaria em risco a sociedade como a conhecemos. Ao longo deste cap\'itulo vamos ver como \'e seu funcionamento usando os conte\'udos do cap\'itulo anterior.

\section{Preparando-se para criptografar}

Para que o algoritmo RSA possa encriptar de forma eficiente, precisaremos seguir uma s\'erie de passos necess\'arios para que o RSA funcione, mas que ainda n\~ao s\~ao parte do algoritmo.

O primeiro passo \'e a convers\~ao das letras da mensagem em n\'umeros. A essa etapa chamaremos de pr\'e-codifica\c{c}\~ao. Para que o RSA venha a funcionar, precisamos seguir uma tabela como a apresentada abaixo:
\[
\begin{array}{ccccccccccccc}
A & B & C & D & E & F & G & H & I & J  &  K  & L  & M  \\ 
10 & 11 & 12 & 13 & 14 & 15 & 16 & 17 & 18 & 19 &  20 & 21 & 22 \\ 
\\
N & O  & P  & Q  & R  & S & T  & U  & V  & X  & Y  & W  & Z \\
23 & 24 & 25 & 26 & 27 & 28 & 29 & 30 & 31 & 32 & 33 & 34 & 35 \\
\end{array}
\]

Para representar espa?os vamos usar o 99. Avisamos que esta \'e uma tabela apenas com finalidade did\'atica, e, por isso h\'a v\'arios caracteres desconsiderados. Como exemplo vamos pr\'e-encriptar o poema Amor, de Oswald de Andrade. O texto do poema a ser pr\'e-encriptado \'e o seguinte:

\begin{center}
Amor  \\ 
Humor. \\ 
\end{center}

Como primeiro passo vamos converter todas as letras em n\'umeros, resultando em:
 
\begin{center}
10 22 24 27 99 17 30 22 24 27
\end{center}

Feito isso, n\'os agrupamos o conjunto em um bloco \'unico de caracteres:

\begin{center}
10222427991730222427
\end{center}

Atente-se ao fato de todo o caractere convertido possuir sempre o mesmo n\'umero de algarismos. Isso \'e \'util para evitar ambiguidades na fase de desencripta\c{c}\~ao.

Nosso pr\'oximo passo nesta fase que antecede a encripta\c{c}\~ao, consiste em definir quais ser\~ao os primos $p$ e $q$. Para nosso exemplo vamos usar $p=17$ e $q=23$, como mencionado na Introdu\c{c}\~ao desta obra, temos que $n = pq$, logo $n=391$.

O \'ultimo passo da pr\'e-encripta\c{c}\~ao consiste em quebrar o n\'umero que obtemos acima em blocos menores. Esses blocos devem obedecer \`a duas regras b\'asicas: serem menores que $n$, ou no nosso exemplo $391$, pois iremos trabalhar com m\'odulos de $391$ durante a encripta\c{c}\~ao, e n\~ao podem se iniciar por $0$, para n?o haver ambiguidades na desencripta\c{c}\~ao. Vejamos como a nossa mensagem fica quando pr\'e-encriptada.

\begin{center}
$102$ | $224$ | $279$ | $91$ | $7$ | $30$ | $222$ | $42$ | $7$
\end{center}

Perceba que n\~ao h\'a rela\c{c}\~ao entre nenhum dos n\'umeros obtidos com um caractere espec\'ifico, o que torna imposs\'ivel a associa\c{c}\~ao de um n\'umero a uma letra por frequ\^encia de aparecimento. 

\section{Codificando e decodificando mensagens}

Encerrada a fase de pr\'e-codifica\c{c}\~ao vamos agora codificar nossas mensagens. Manteremos os valores e exemplos da se\c{c}\~ao anterior a fim de facilitar a compreens\~ao.

\subsection{Codificando uma mensagem}

A esta altura n\'os j\'a conhecemos o n\'umero $n$, que em nosso exemplo possui o valor de $391$. O outro n\'umero que iremos usar ser\'a o $\textbf{e}$. Tomaremos que o $mdc(\textbf{e}, \phi(n)) = 1$. Para calcularmos o valor de $\phi(n)$ precisaremos aplicar a seguinte receita:

$$\phi(n) = (p-1)(q-1)$$

Que em nosso exemplo resulta em:

$$\phi(391) = (17 - 1)(23 - 1) = 16 \cdot 22 = 352$$

Para determinarmos o $\textbf{e}$ basta escolher o menor primo que $mdc(\textbf{e}, 352) = 1$, que no nosso caso ser\'a o $3$. Optamos por um primo n\~ao m\'ultiplo ao inv\'es de um composto para que possamos usar o teorema de Fermat mais adiante. Ao conjunto $(n, \textbf{e} )$ denominamos chave de encripta\c{c}\~ao.

Vamos chamar o bloco codificado que iremos encriptar de $b$, lembrando que $b$ \'e um n\'umero inteiro menor que $n$. Tambem chamaremos o bloco ap\'os a codifica\c{c}\~ao de $C(b)$. Para obtermos $C(b)$ devemos aplicar a seguinte f\'ormula:

$$C(b) \equiv b^\textbf{e} \pmod{n} $$

Podemos para facilitar dizer que $C(b)$ ? o res?duo de $b^\textbf{e}$ pelo m\'odulo $n$. Vamos \`a uma demonstra\c{c}\~ao pr\'atica com o primeiro bloco de nossa mensagem, que possui o valor $102$. Para simplificar o nosso trabalho vamos utilizar as opera\c{c}\~oes modulares.

$$102^3 \equiv 24276 \equiv 34 \pmod{391}$$

Faremos o mesmo procedimento para todos nossos blocos:

$$224^3 \equiv 11239424 \equiv 129 \pmod{391} $$
$$279^3 \equiv 21717639 \equiv 326 \pmod{391} $$
$$91^3  \equiv 753571   \equiv 114 \pmod{391} $$
$$7^3   \equiv 343      \equiv 343 \pmod{391} $$
$$30^3  \equiv 27000    \equiv 21  \pmod{391} $$
$$222^3 \equiv 10941048 \equiv 86  \pmod{391} $$
$$42^3  \equiv 74088    \equiv 189 \pmod{391} $$
$$7^3   \equiv 343      \equiv 343 \pmod{391} $$

Portanto, ``Amor Humor'', encriptado pelo RSA com as chaves $(391 , 3)$ \'e: 

\begin{center}
	 $34$ | $129$ | $326$ | $114$ | $343$ | $21$ | $86$ | $189$ | $343$
\end{center}

\subsection{Decodificando uma mensagem}

Para podermos desencriptar uma mensagem n\'os precisamos de dois n?meros. O primeiro \'e a nossa chave p\'ublica $n$. O segundo n\'umero \'e $d$, que consiste no inverso de $\texbf{e}$ em $\phi(n)$. Para o nosso exemplo $d= 235$.

Agora que estamos de posse de $d$, podemos usar o par $(n,d)$ para desencriptar a mensagem, onde $a$ \'e o bloco encriptado e $D(a)$ a mensagem desencriptada, usando a f\'ormula:

$$D(a) \equiv a^d \pmod{n}$$

Note que na fun\c{c}\~ao acima n\'os assumimos o compromisso de que $D(C(b)) = b$. Para comprov\'a-la vamos na pr\'oxima sess\~ao fazer sua demonstra\c{c}\~ao. Neste momento vamos apenas aplic\'a-la ao nosso exemplo. Sabemos que o primeiro passo consiste em calcular $d$. Vamos tomar que $p$ e $q$ deixam resto $5$ na divis\~ao por $6$. Com isso podemos afirmar que:

$$(p-1)(q-1) \equiv 4 \cdot 4 \equiv 16 \equiv 4 \equiv -2 \pmod{6}$$
$$(p-1)(q-1) = 6 \cdot k -2$$

No entanto, podemos dizer que $6 \cdot k - 2 \equiv 4 \cdot k - 1 \pmod{3}$. Podendo dizer assim que $d$ \'e igual a $4 \cdot k -1$. Feito isso vamos aos n\'umeros primos de nosso exemplo: $17$ e $23$. Com eles iremos obter:

$$(p-1)(q-1) = 16 \cdot 22 = 352 = 6 \cdot 58 + 4 = 6\cdot 59 -2$$

Com isso obtemos que $k=59$. Aplicando $k$, n\'os teremos que:

$$d = 4 \cdot 59 - 1 = 235$$

Agora que j\'a conhecemos a $d$ podemos decodificar a mensagem. Vamos fazer isso em nosso primeiro bloco codificado, que possui o valor $34$. Para achar a resposta precisaremos calcular $D(34) \equiv 34^{235} \pmod{391}$. Esse c\'alculo seria praticamente imposs\'ivel sem o uso dos Teoremas: chin\^es do resto e de Fermat.

Nosso primeiro passo ser\'a o de calcular $34^{235}$ nos m\'odulos $17$ e $23$, que s\~ao os primos resultantes da fatora\c{c}\~ao de $n$. Neste caso, come\c{c}amos com:

$$34 \equiv 0 \pmod{17}$$
$$34 \equiv 11 \pmod{23}$$

Assim teremos que $34^{235} \equiv 0^{235} \equiv 0 \pmod{17}$. Aplicando o teorema de Fermat na outra congr\^encia teremos:

$$11^{235} \equiv (11^{22})^{10} \cdot 11^{15} \equiv 11^{15} \pmod{23}$$

Como $ 11 \equiv -12 \equiv -4 \cdot 3 \pmod{23}$, n\'os podemos afirmar que:

$$11^{235} \equiv 11^{15} \equiv -4^{15} \cdot 3^{15}\pmod{23}$$

Com isso, teremos:

$$415 \equiv 230 \equiv (2^{11})^2 \cdot 2^8 \equiv 2^8 \equiv 3 \pmod{23}$$
$$315 \equiv 3^{11} \cdot 3^4 \equiv 3^4 \equiv 12 \pmod{23}$$

Concluindo assim:

$$11235 \equiv -415 \cdot 315 \equiv -3 \cdot 12 \equiv 10 \pmod{23}$$

Temos assim as congru\^encias $34^{235} \equiv 0 \pmod{17}$ e $34^{235} \equiv 10 \pmod{23}$. Com isso podemos aplicar o teorema chin\^es do resto no sistema:

$$x \equiv 0 \pmod{17}$$
$$x \equiv 10 \pmod{23})$$

Com ele iremos obter: 

$$10 + 23y \equiv 0 \pmod{17})$$

Obtendo assim:

$$6y \equiv 7 \pmod{17}$$

Por\'em, $3$ \'e o inverso de $6$ no m\'odulo $17$, e por isso teremos:

$$y \equiv 3 \cdot 7 \equiv 4 \pmod{17}$$

Com isso iremos chegar at\'e $x = 10 + 23y = 10 + 23 \codt 4 = 102$. Caso voc? venha a conferir na se\c{c}\~ao sobre codifica\c{c}\~ao de mensagens poder\'a confirmar o resultado. Os demais blocos podem ser decodificados da mesma forma, apenas n\~ao ser\~ao mostradas nesta obra por necessitar de muitos passos, o que tornaria este cap\'itulo inutilmente mais longo.

\section{Provando a funcionalidade do RSA}

Ao longo desta se\c{c}\~ao vamos provar que o RSA funciona no processo de decodifica\c{c}\~ao. Para podermos fazer isso tudo, o que teremos que fazer \'e provar que:

$$b \equiv DC(b) \pmod{n}$$

N\'os j\'a vimos que $C(b) \equiv b^\textbf{e} \pmod{n}$ e $D(a) \equiv a^d\pmod{n}$. Se combinarmos ambas as congru\^encia iremos obter:

$$D(C(b)) \equiv {(b^\textbf{e})}^d = b^{ed}\pmod{n}$$

Logo o que temos de provar \'e que $b^{ed} \equiv b \pmod{n}$. Como por defini\c{c}\~ao $\textbf{e}d \equiv 1 \pmod{(p-1)(q-1)}$, o que nos leva at\'e:

$$ed = 1+k(p-1)(q-1)$$

Pelo Teorema Chin\^es do Resto e levando em conta a express\~ao para obter $3d$ temos que:

$$b^{ed} \equiv b(b^{p-1})^{k(q-1)}$$

Tomando que $p$ n\~ao divide $b$, n\'os podemos vir usar o Teorema de Fermat, de modo a obter:

$$b^{p-1} \equiv 1 \pmod{p}$$

Obtendo assim:

$$b^{ed} \equiv b \cdot (1)^{k(q-1)}\equiv b \pmod{p}$$

Mesmo considerando que $b$ seja m\'ultiplo de $p$, teremos que $b$ e $b^{ed}$ s\~ao congruentes a $0$, logo nesse caso tamb\'em \'e v\'alida a congru\^encia, tendo assim: 
}

$$b^{ed} \equiv b \pmod{p}$$

Pelo mesmo m\'etodo podemos  obter $q$, obtendo o par: 

$$b^{ed} \equiv b \pmod{p}$$
$$b^{ed} \equiv b \pmod{q}$$

Veja que $b$ \'e solu\c{c}\~ao de: 

$$x \equiv b \pmod{p}$$
$$x \equiv b \pmod{q}$$

Pelo Teorema Chin\^es do resto esse sistema tem solu\c{c}\~ao igual:

$$b + p \cdot q \cdot t$$

Onde $t\in \mathbb{Z}$. Logo, como provamos anteriormente, temos que:

$$b^{ed} \equiv b + p \cdot q \cdot k$$

Para algum inteiro $k$. Isso equivale a $b^{3d} \equiv b \pmod{p}$. Isso comprova que $b = D(C(b))$.

\section{Discutindo a seguran\c{c}a do RSA}

Antes de mudarmos o foco deste trabalho, vamos prestar aten\c{c}\~ao no que tange a seguran\c{c}a do RSA. Vamos supor que algu\'em, que vamos denominar Irineu, esteja com uma escuta em nossas mensagens, tendo assim acesso tanto \`a mensagem codificada quanto \`a chave p\'ublica $n$. Vamos lembrar que $n$ \'e a multiplica\c{c}\~ ao dos primos $p$ e $q$. Sabendo disso, bastaria para Irineu fatorar $n$ para obter $p$ e $q$ e depois descobrir $d$ para poder decodificar a mensagem, como j\'a foi explicado neste cap\'itulo.

Isso pode parecer muito simples, mas como j\'a mostramos na se\c{c}\~ao sobre fatora\c{c}\~ao, n\~ao h\'a um algoritmo conhecido que possa fazer isso de forma eficiente. O que ocorre \'e que um algoritmo que fa\c{c}a a fatora\c{c}\~ao de forma eficiente pode vir a surgir a qualquer momento, do ponto que n\~ao h\'a nenhuma prova matem\'atica de que esse algoritmo n\~ao exista.

O que iremos fazer nos pr\'oximos cap\'itulos, consiste em propor uma varia\c{c}\~ao da criptografia RSA que n\~ao seja completamente vulner\'avel caso uma fatora\c{c}\~ao eficiente seja descoberta: a \textit{Criptografia RSA Gaussiana}.

\chapter {Inteiros e Primos de Gauss}
\label{IG}

\subparagraph{
At\'e o momento apenas os n\'umeros inteiros foram neste projeto, mas para podermos entender a RSA Gaussiana \'e necess\'ario conhecer os inteiros gaussianos. Neste cap\'itulos vamos apresentar os inteiros e os primos gaussianos e suas propiedades b\'asicas. 
}

\section{Inteiros de Gauss e suas propiedades}

\subparagraph{
O inteiros gaussianos, que a partir de agora iremos nos referenciar por $Z[i]$, s\~ao um subconjuntos dos n\'umeros complexos, relembrando que os n\'umeros complexos s\~ao os n\'umeros de forma $a+b\textbf{i}$, onde $a$ e$b$ s\~ao reais e $\textbf{i}$ \'e a $\sqrt{-1}$. A diferen\c{c}a entre o conjunto $Z[i]$ e o conjunto $C$ reside no fato de em $Z[i]$ $a$ e $b$ serem n\'umeros inteiros.
}
\subparagraph{
Por $Z[i]$ estar contido em $C$, as opera\c{c}\~oes deste conjunto podem ser realizadas, por exemplo, se tomarmos $z_1= a + b\textbf{i}$ e $z_2= c + d\textbf{i}$ n\'os iremos obter:
}
\[
	\begin{array}{c}
		\textit{$z_1   +   z_2 = (a + c) + (b + d)\textbf{i}$}\\
		\textit{$z_1 \cdot z_2 = (ac - bd) + (ad + bc)\textbf{i}$}
	\end{array}
\]
\subparagraph{
Outra propiedade herdada \'e a dos elementos neutros, o $0 = 0 + 0\textbf{i}$ continua sendo o elemento neutro da adi\c{c}\~ao. O $1 = 1 + 0\textbf{i}$ tamb\'em continua sendo o elemento neutro da multiplica\c{c}\~ao. As propiedades associativa da adi\c{c}\~ao e da multiplica\c{c}\~o, comutativa da adi\c{c}\~o e multiplica\c{c}\~o e distributiva tamb\'em s\~ao herdadas do conjunto $C$.
}
\subparagraph{
Observe que todo inteiro $n$ tem no conjunto $Z[i]$ uma nota\c{c}\~ao na forma $n + 0 \textbf{i}$ que pode ser suprimida. Feito isso vamos nos ater aos crit\'erios de divisibilidade em $Z[i]$. Vamos fatorar o n\'umero $5$, que no conjunto $Z$ \'e primo:
}
\[
	\begin{array}{c}
		\textit{$(1 + 2\textbf{i}).(1 - 2\textbf{i}) = 1 - 2\textbf{i} + 2\textbf{i} - 4\textbf{i}^2 = 1 - 4(-1) = 5$}\\
	\end{array}
\]
\paragraph{
Preste aten\c{c}\~ao ao fato de que os n\'umeros primos do conjunto inteiro n\~ao s\~ao necessariamente primos do conjunto gaussiano.}
\subparagraph{
Vamos definir o que vem a ser divisibilidade em $Z[i]$ para que possamos progredir. Vamos supor que $x$ e $y$ perte\c{c}am a $Z[i]$ e sejam diferentes entre si e diferentes de $0$. Dizemos que $y$ divide $x$ e indicamos na forma de $y|x$, se e somente se existe um inteiro gaussiano $w$ tal que $x=yw$. Tome de exemplo $(1 + \textbf{i})|2$, pois $ 2 = (1 + \textbf{i})(1 - \textbf{i})$ e $(1 + \textbf{i})|(1 - \textbf{i})$, pois $ 1 + \textbf{i} = \textbf{i}(1 - \textbf{i})$.
}
\subparagraph{
Como os resultados das fatora\c{c}\~oes n\~ao se equivalem, seria muito \'util a n\'os saber se a defini\c{c}\~ao de divisibilidade \'e a mesma tanto nos inteiros quanto nos gaussianos. Para podermos checar isso vamos supor que $x$ e $y$ existem em $Z$ e que $y|x$ em $Z[i]$. Com isso deve existir em $Z[i]$ um n\'umero $w = c + d\textbf{i}$ tal que $x=wy$, ou seja $x=(c + d\textbf{i})y = cy + d\textbf{i}y)$. Com isso temos que $x=cy$ e $0 = dy$, o que implica em $d = 0$ e $w=c$, o que faz de $w$ um membro do conjunto $Z$. Logo $x=wy=cy$ e com isso conclu\'imos que se $y|x$ em $Z[i]$, ele tamb\'em o faz em $Z$.
}
\subparagraph{
Sabemos que $\pm 1$ dividem qualquer elemento da conjunto inteiro, analogamente $\pm 1$ e $\pm \textbf{i}$ fazem isso no conjunto complexo e gaussiano, esses n\'umeros s\~ao as unidades b\'asicas do conjunto. Sendo $w$ uma unidade de inteiro gaussiano, e $x$ e $y$ inteiros gaussianos tais que $x = wy$ dizemos que $x$ e $y$ s\~ao elementos associados.
}
\subparagraph{
Agora que sabemos sobre os elementos associados e as unidades b\'asicas podemos definir os primos gaussianos. Um \textit{primo gaussiano} \'e um inteiro gaussiano que \'e divi\'ivel apenas pelos seus elementos associados e pelas unidades de $Z[i]$. Al\'em de possuir primos, assim com em $Z$ eles s\~ao infinitos, isso lhe ser\'a mostrado mais a frente.
}
\section{Fatora\c{c}\~ao \'unica}
\subparagraph{
A fatora\c{c}\~ao \'unica \'e a propiedade base de toda a Teoria de n\'umeros, para que possamos construir um algoritmo RSA gaussiano tal qual desejamos se torna necess\'aria que essa propiedade esteja presente no conjunto de inteiros gaussianos. Para podermos entender se isso \'e vi\'avel ou n\~ao, antes devemos conhecer que a \textit{norma} de um n\'umero gaussiano $x=a+b\textbf{i}$ \'e igual a $a^2 + b^2$, a fun\c{c}\~ao da norma \'e verificar rela\c{c}\~oes de semelhan\c{c}a e diferen\c{c}a no conjunto gaussiano e seu s\'imbolo \' $N(x)$.
}
\subparagraph{
Antes de provarmos a fatora\c{c}\~ao \'unica, provemos que todo o inteiro de Gauss com norma maior que $1$ pode ser escrito como produto e um ou mais primos de Gauss. Se $N(x)=2$, como $2$ \'e primo e a norma multiplicativa temos que $2$ \'e primo. Da mesma forma podemos estender para $N(x)>2$, se $x$ \'e primo a fatora\c{c}\~ao ser\'a imediata, se $x$ n\~ao for primos n\'os teremos que $x=a \cdot b \Rightarrow N(x) = N(a) \cdot N(b)$, com $ N(a), N(b) > 1$, logo $ N(a), N(b) < N(x)$. Podemos supor que $N(y) < N(x)$, $y$ \'e fator\'avel. Logo $a$, $b$ e $x$ tambem s\~ao fator\'aveis.
}
\subparagraph{
Agora iremos provar a fatora\c{c}\~ao \'unica, para isso, vamos considerar as fatora\c{c}\~oes $p_1 \cdot p_2 \cdot ... \cdot p_n$ e $q_1 \cdot q_2 \cdot ... \cdot q_m$, sendo $\epsilon$  uma unidade que implica em que a sequencia$(p_i)$ seja uma permuta\c{c}\~ao, exceto em casos de multiplica\c{c}\~ao por unidade, de $(q_i)$. Se $max(m;n) = 1$, o resultado ser\'a imediato. Supondo que ele vale se $max(n';m') < max(m;n)$, pelo Lema de Euclides, que diz que se $n$ \'e um n\'umero inteiro e divide um produto $ab$ e \'e primo entre si com um fator, ent\~ao $n$ divide o outro fator, vemos que para algum $i, p_n| q_i$.
}
\subparagraph{
Para n\~ao perdermos a generalidade vamos tomar que $i=m$. Como $p_n$  $q_m$ s\~ao primos, ent\~ao $q_m = \epsilon' p_n$, com $\epsilon'$ sendo uma unidade. Logo $p_1 \cdot p_2 \cdot ... \cdot p_n = \epsilon'q_1 \cdot q_2 \cdot ... \cdot q_m \Leftrightarrow p_1 \cdot p_2 \cdot ... \cdot p_{n-1} = \epsilon\epsilon'q_1 \cdot q_2 \cdot ... \cdot q_{m-1} $. Como $p_1 \cdot p_2 \cdot ... \cdot p_n$ \'e uma permuta\c{c}\~ao de $q_1 \cdot q_2 \cdot ... \cdot q_m$, exceto em casos de multiplica\c{c}\~ao por unidades, fica provada por indu\c{c}\~ao a fatora\c{c}\~ao \'unica dos inteiros gaussianos.
}
\section{Primos de Gauss}
\subparagraph{
Neste cap\'itulo veremos quem s\~ao os n\'umeros considerados primos em $Z[i]$, os famosos primos de Gauss. Observe que se $N(\pi)$ \'e primo em $Z$, n\'os teremos $\pi$ sendo primo em $Z[i]$, de acordo com a demontra\c{c}\~ao de fatora\c{c}\~ao \'unica. }
\subparagraph{
Atente-se ao fato de que todo o primo $\pi$ divide $N(\pi)$, portanto ele deve dividir ao menos um fator primo em $Z$ de $N(\pi)$. Caso $\pi$ venha a dividir dois fatores distintos $x$ e $y$, ambos primos em Z, n\'os ter\'iamos que $x|1$, o que seria um absurdo. Com isso \'e poss\'ivel se concluir que todo o primo de Gauss divide $1$ e somente $1$ primo inteiro positivo(al\'em de se oposto negativo).
}
\subparagraph{
Considerando o caso acima e tomando o n\'umero primo com um inteiro positivo $p$, n\'os temos tr\^es casos que podemos prestar aten\c{c}\~ao. O primeiro ocorre quando $p$ \'e par, sendo que nesse caso $p=2$. Nesse caso obtemos como primos gaussianos $1+i$,$ 1-i$,$ -1+i $e$ -1-i $}
\subparagraph{
Caso tomemos $p \equiv 3(mod 4)$ sempre viremos a obter n\'umeros primos. J\'a no caso de $p \equiv 1(mod 4)$, apenas os casos em que $a^2+b^2=p$ resultam em n\'umeros primos gaussianos.
}
\subparagraph{
Como toda a criptografia de chave p\'ublica necessita de um conjunto de chaves, foi-se definido que os n\'umeros primos gaussianos viriam a ser as chaves para a criptografia RSA Gaussiana. No pr\'oximo cap\'itulo ser\'a apresentado o que j\'a est\'a feito neste algoritmo e o que ficar\'a como implica\c{c}\~ao futura para desenvolvimento.}
\subparagraph{
Nosso pr\'oximo passo consiste na encripta\c{c}\~ao, embora ela ainda n\~ao possua um algoritmo concluído, assim como a desencripta\c{c}\~ao ela j\'a possui um esbo\c{c}o
}
\chapter {RSA Gaussiano e Conclus�es}
\label{RSAG}

\hspace{7mm}Chegamos ao \'ultimo cap\'itlo desta obra, aqui ser\'a debatido sobre tudo o que foi alcan�ado at\'e o momento no que diz respeito ao RSA Gaussiano, veremos como sua f\'ormula \'e planejada e o que ter\'a de ser feito no futuro para que este algoritmo se torne uma op��o entre os algoritmos criptogr\'aficos.

\section{O RSA Gaussiano e seus trabalhos futuros}

\hspace{7mm}Ao longo desse se\c{c}\~ao lhe ser\'a apresentado como o RSA Gaussiano dever\'a vir a funcionar. Para iniciarmos, teremos que, assim como na criptografia RSA fazer uma pr�-encripta\c{c}\~ao vindo a transformar todas as letras em n\'umero inteiros, da mesma forma que j\'a ocorre.

A segunda parte do processo, que \'e a encripta\c{c}\~ao depende de algumas garantias as quais ainda n\~ao possu\'imos, embora j\'a saibamos que a propiedade da fatora\c{c}\~ao \'unica \'e v\'alida para o conjunto $Z[i]$. Um dos meios para que possamos seguir consiste em definir a opera\c{c}\~ao de congru\^encia modular para o conjunto gaussiano, tamb\'em se fazem necess\'arias propiedades an\'alogas ao Teorema de Fermat e ao Teorema chin\^es do resto para que possamos seguir a mesma linha de encripta\c{c}\~ao do algoritmo RSA.

Para o RSA Gaussiano \'e planejada a encripta\c{c}\~ao com a f\'ormula:

$$C'(a) \equiv a^3 \pmod{n'}$$

Nesta f\'ormula $a$ seria o n\'umero a ser encriptado, $n'$ seria a chave p\'ublica gaussiana, derivada da multiplica\c{c}\~ao de $p'$ e $q'$, que s\~ao n\'umeros primos gaussianos. Os n\'umeros $p'$ e $q'$ s\~ao as chaves privadas de encripta\c{c}\~ao.

Para a desencripta\c{c}\~ao, embora ainda n\~ao tenhamos uma prova de seu funcionamento pelos motivos j\'a citados acima, ela \'e planejada em um primeiro momento pelo n\'umero $d'$, que pode manter sua f\'ormula similar a do RSA ou n\~ao. Caso ele n\~ao precise de mudan\c{c}as por conta do conjunto gaussiano, dever\'a ser da seguinte f\'ormula:

$$3d' \equiv 1 \pmod{(p'-1)(q'-1)}$$

Supondo que a f\'ormula de $d'$ acima seja v\'alida, teremos que para podermos concluir desencripta\c{c}\~ao a f\'ormula an\'aloga a original, que seria:}

$$D'(b) \equiv b^{d'} \pmod{n'}$$

Nesta f\'ormula temos que $b$ \'e um n\'umero encriptado pelo RSA Gaussiano, $d'$ a constante cuja f\'ormula foi mostrada anteriormente, $n'$ a chave p\'ublica e $D'(b)$ o resultado da desencripta\c{c}\~ao.
Nesta f\'ormula temos que $b$ \'e um n\'umero encriptado pelo RSA Gaussiano, $d'$ a constante cuja f\'ormula foi mostrada anteriormente, $n'$ a chave p\'ublica e $D'(b)$ o resultado da desencripta\c{c}\~ao.

Com isso conclu\'imos este projeto, aguardando que muito em breve tudo o que ficou como trabalho futuro aqui venha a ser realizado, muito obrigado por sua aten\c{c}\~ao a este projeto e esperamos que ele venha a ser uma fonte de inspira\c{c}\~ao para seus projetos futuros.



\addcontentsline{toc}{chapter}{Bibliografia} %para aparecer a  bibliografia no �ndice
%%%%%%%%%%%%%%%%%%%%%%%%%%%%%%%%%%% Comandos para gerar a bibliografia em Portugu�s %%%%%%%%%%%%%%%%%%%%%%%%

\selectbiblanguage{brazil}
%\bibliographystyle{babplain}
\bibliographystyle{babalpha}
\bibliography{Bibliografia}

%%%%%%%%%%%%%%%%%%% Caso n�o gere a bibliografia por falta de pacotes rode com os comandos abaixo e retire o pacote \usepackage{babelbib} do in�cio do documento %%%%%%%%%%%

\bibliographystyle{alpha}
\nocite{*}
%\bibliography{Bibliografia}

%%%%%%%%%%%%%%%%%%%%%%%%%%%%%%%%%%%%%%%%%%%%%%%%%%%%%%%%%%%%%%%%%%%%%%%%%%%%%%%%%%%%%%%%%%%%%%%%%%%%%%%%%%%%%
\end{document}
