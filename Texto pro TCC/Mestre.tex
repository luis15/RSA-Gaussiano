
%%%%%%%%%%%%%%%%%%%%%%%%%%%%%%%%%% DISSERTA��O OU TESE %%%%%%%%%%%%%%%%%%%%%%%%%%%%%%%%%%%%%%%%%
\documentclass[11pt]{report}
\usepackage{graphicx}
\usepackage[brazil]{babel}
\usepackage[latin1]{inputenc}
\usepackage{babelbib} % Esse pacote gera a bibliografia em Portugu�s. Acho que deve ser instalado algum pacote, n�o me recordo.
\usepackage{latexsym}
\usepackage{amsfonts}
\usepackage{amsthm}
\usepackage{amssymb}
\usepackage{amsfonts}
\usepackage{xypic}
\usepackage{ulem}
\usepackage[all]{xy}

\usepackage{makeidx} %cria o �ndice remissivo
\makeindex

\vfuzz2pt % Don't report over-full v-boxes if over-edge is small
\hfuzz2pt % Don't report over-full h-boxes if over-edge is small

% TEOREMAS -------------------------------------------------------
\theoremstyle{alpha}
\newtheorem{Th}{Teorema}[section]
\newtheorem{Cor}[Th]{Corol�rio}
\newtheorem{Le}[Th]{Lema}
\newtheorem{Pro}[Th]{Proposi��o}
\theoremstyle{definition}
\newtheorem{Df}[Th]{Defini��o}
\newtheorem{Obs}[Th]{Observa��o}
\newtheorem{Dig}[Th]{Digress�o}
\newtheorem{Ex}[Th]{Exemplo}
\newtheorem{Esc}[Th]{Esc�lio}
\newtheorem{Ass}[Th]{Asser��o}

% MATEMATICA E LOGICA -------------------------------------------

\newcommand{\lan}     {\langle}
\newcommand{\ran}     {\rangle}
\newcommand{\IN}      {\mathbb{N}}
\newcommand{\A}       {\mathcal{A}}
\newcommand{\B}       {\mathcal{B}}
\newcommand{\C}       {\mathcal{C}}
\newcommand{\D}       {\mathcal{D}}
\newcommand{\F}       {\mathcal{F}}
\newcommand{\I}       {\mathcal{I}}
\newcommand{\T}       {\mathcal{T}}
\newcommand{\prem}    {\textsf{Prem}}
\newcommand{\con}     {\textsf{Con}}
\newcommand{\ta}      {\textsf{T}}
\newcommand{\ax}      {\textsf{Ax}}

\newcommand{\VAL}     {\textsc{Val}}
\newcommand{\SSB}     {{\bf SSbe}}
\newcommand{\SSO}     {{\bf SSat}}
\newcommand{\SSC}     {{\bf SSco}}
\newcommand{\SSU}     {{\bf SSmu}}

\newcommand{\two}     {{\bf 2}}
\newcommand{\twi}     {{\bf \Omega}}
\newcommand{\Rra}     {\Rrightarrow}

\newcommand{\jul}{\mathnormal{^{1}\hspace{-0,12cm}/\hspace{-0,04cm}_{2}}} 

\DeclareMathAlphabet{\mathpzc}{OT1}{pzc}{m}{it}

\begin{document}

\title{\textbf{Criptografia RSA gaussiana}}
\author{\textbf{Luis Antonio Co\^{e}lho}\vspace{2cm}\\
Trabalho de Conclus\~{a}o de Curso - apresentado \`{a}\\ Faculdade de Tecnologia da\\ Universidade Estadual de Campinas \vspace{2cm}\\
Orientadora:
\textbf{Profa. Dra. Juliana Bueno}\vspace{2cm}\\}
 \maketitle



% ------------------------------------------------------------------------
%{
%
%%%%%%%%%%%%%%%%%%%%%%%%%%%%%%%%%%%%%%%%%%%%%%%%%%%%%DEDICAT�RIA%%%%%%%%%%%%%%%%%%%%%%%%%%%%%%%%%%%%%%%%%%%%%%%%%%%%%%%%%%%%%

\thispagestyle{empty}

\hspace{1cm} \vspace{8.5cm}


\hspace{2.83cm}\textit{\Large{Dedico este trabalho primeiramente ? Deus q?? permitiu q?? tudo isso acontecesse, ?? longo d? minha vida, ? n�? somente nestes anos como universit�rio, m?s que ?m todos ?s momentos � o maior mestre q?? algu�m pode conhecer.Agrade�o ? todos ?s professores p?r m? proporcionar ? conhecimento n�? apenas racional, m?s ? manifesta��o d? car�ter ? afetividade d? educa��o n? processo d? forma��o profissional, p?r tanto q?? s? dedicaram ? mim, n�? somente p?r terem m? ensinado, m?s por terem m? feito aprender. ? palavra mestre, nunca far� justi�a ??s professores dedicados ??s quais s?m nominar ter�o ?s meus eternos agradecimentos. Agrade�o ? minha m�? Raquel, hero�na q?? m? d?? apoio, incentivo n?s horas dif�ceis, de des�nimo ? cansa�o e a todos q?? direta ?? indiretamente fizeram parte d? minha forma��o, ? m?? muito obrigado. }}
                   % dedicat�ria
%}
% ------------------------------------------------------------------------
%{
%
%%%%%%%%%%%%%%%%%%%%%%%%%%%%%%%%%%%%%%%%%%%%%%%%%%%%%%%INVOCA��O%%%%%%%%%%%%%%%%%%%%%%%%%%%%%%%%%%%%%%%%%%%%%%%%%%%%%%%%%%%%%

\thispagestyle{empty}

%\vspace{20cm}

\noindent{AMO-TE TANTO, meu amor... n�o cante}\\
O humano cora��o com mais verdade...\\
Amo-te como amigo e como amante\\
Numa sempre deversa realidade.\\

 \vspace{0.5 cm}

\noindent{Amo-te afim, de um calmo amor prestante,}\\
E te amo al�m, presente na saudade.\\
Amo-te, enfim, com grande liberdade\\
Dentro da eternidade e a cada instante.\\

 \vspace{0.5 cm}

\noindent{Amo-te como um bicho, simplesmente,}\\
De um amor sem mist�rio e sem virtude\\
Com um desejo maci�o e permanente.\\

 \vspace{0.5 cm}

\noindent{E de te amar assim muito e ami�de,}\\
� que um dia em teu corpo de repente\\
Hei de morrer de amar mais do que pude.\\



(Vin�cius de Moraes, \textit{Soneto do amor total})
                   % invoca��o
%}
% ------------------------------------------------------------------------
%{
%\typeout{Acknowledgements}
%
%%%%%%%%%%%%%%%%%%%%%%%%%%%%%%%%%%%%%%%%%%AGRADECIMENTOS%%%%%%%%%%%%%%%%%%%%%%%%%%%%%%%%%%%%%%%%%%%%%%%%%%%%%%%%%%%%%%%%%%%%
\pagestyle{fancy}
%\pagestyle{fancy}
\fancyhead[C]{\textsl{Agradecimentos}}
\fancyhead[R]{7}
\fancyfoot[C]{}	


\chapter*{Agradecimentos}

\vspace{1.5cm}


\noindent Agrade�o \`a todos que me apoiaram no decorrer deste projeto, desde de minha orientadora Profa. Dra. Juliana Bueno-Soler at\'e voc�, meu caro leitor.\\
                   % agradecimentos
%}
% -----------------------------------------------------------------------
{ \typeout{Abstract}

%%%%%%%%%%%%%%%%%%%%%%%%%%%%%%%%%%%%%%%RESUMO%%%%%%%%%%%%%%%%%%%%%%%%%%%%%%%%%%%%%%%%%%%%%%%%%%%%%%%%%%%%%%%%%%%%%%%%%%%%%%%%

\thispagestyle{empty}

\hspace{1cm} \vspace{2.2cm}

\noindent {\Huge {\bf Resumo}}

\vspace{1.5cm}

\noindent O presente relat�rio exp�e o resultado parcial do projeto para TCC sobre o algoritmo de criptografia RSA gaussiano.
                    % resumo
}
% ------------------------------------------------------------------------

\setcounter{page}{1}
\tableofcontents                  % cria o �ndice

% ------------------------------------------------------------------------
{ %\typeout{Introducao}
%
%%%%%%%%%%%%%%%%%%%%%%%%%%%%%%%%%%%%%%%%%%%INTRODU��O%%%%%%%%%%%%%%%%%%%%%%%%%%%%%%%%%%%%%%%%%%%%%%%%%%%%%%%%%%%%%%%%%%%%%%%

\thispagestyle{empty}

\addcontentsline{toc}{chapter}{Introdu��o}

\chapter*{Introdu��o}


Introduza o que voc� pretende fazer no decorrer do seu trabalho.
					
\chapter {Introdu\c{c}\~{a}o}
\label{Intro}

\hspace{7mm}O sigilo sempre foi uma arma explorada pelos seres humanos para vencer certas batalhas, e at\'e mesmo que na cotidiana miss\~{a}o de se comunicar. Foi a partir dessa necessidade que se criou a \textit{criptografia}, nome dado ao conjunto de t\'ecnicas usadas para se  comunicar em c\'odigos. Seu objetivo \'{e} garantir que apenas os envolvidos na comunica\c{c}\~ao possam compreender a mensagem codificada (ou criptogtafada), garantindo que terceiros n\~ao saibam o que foi conversado.

Para compreender como funciona o processo de codifica\c{c}\~ao e decodifica\c{c}\~ao faz-se necess\'ario o uso de uma s�rie de termos t\'ecnicos, e para fins pedag�gicos iremos introduzir tais conceitos apresentando um dos primeiros algoritmos criptogr\'aficos que se tem conhecimento, a criptografia de C\'esar.Para mais detalhes sobre o tema, veja \cite{coutinho}.

A chamada \textit{criptografia de C\'esar}, criada pelo imperador romano C\'esar Augusto, consistia em substituir cada letra da mensagem por outra que estivesse a tr\^es posi\c{c}\~oes a frente, como, por exemplo, a letra \textbf{A} ser substitu\'ida pela letra \textbf{D}.  

Uma forma muito natural de  generalizar o algoritmo de C\'esar \'e fazer a troca de cada letra da mensagem por outra que venha em uma posi\c{c}\~ao qualquer fixada. A chamada \textit{criptografia de substitui\c{c}\~ao monoalfab\'etica} consite em substituir cada letra por outra que ocupe $n$ posi\c{c}\~oes a sua frente, sendo que o n\'umero $n$ \'e conhecido apenas pelo emissor e pelo receptor da mensagem. O n\'umero $n$ \'e a \textit{chave criptogr\'afica}. Para decifrar a mensagem, precisamos substituir as letras que formam a mensagem criptografada pelas letras que est\~ao $n$ posi\c{c}\~oes antes.

O algoritmo monoalfab\'etico tem a caracter\'istica indesejada de ser de f\'acil decodifica\c{c}\~ao, pois possui apenas {26} chaves poss\'iveis, e isso faz com que no m\'aximo em {26} tentativas o c\'odigo seja decifrado. Com o intuito de dificultar a quebra do c\'odigo monoalfab\'etico foram propostas as \textit{cifras de substitui\c{c}\~ao polialfab\'eticas} em que a chave criptogr\'afica passa a ser uma \textit{palavra} ao inv\'es de um n\'umero. A ideia \'e usar as posi\c{c}\~oes ocupadas pelas letras da chave para determinar o n\'umero de posi\c{c}\~oes que devemos avan\c{c}ar para obter a posi\c{c}\~ao da letra encriptada. Vejamos, por meio de um exemplo, como funciona esse sistema criptogr\'afico.

Sejam ``SENHA'' a nossa chave criptogr\'afica e ``ABOBORA'' a mensagem a ser encriptada. Abaixo colocamos as letras do alfabeto com suas respectivas posi\c{c}\~oes. Observe que repetimos a primeira linha de letras para facilitar a localiza\c{c}\~ao da posi\c{c}\~ao da letra encriptada e usamos a barra para indicar que estamos no segundo ciclo. 

\[
\begin{array}{ccccccccccccc}
    1      & 2 & 3 & 4 & 5          & 6 & 7 & 8          & 9 & 10 &  11 & 12 & 13 \\  
\textbf{A} & B & C & D & \textbf{E} & F & G & \textbf{H} & I & J  &  K  & L  & M  \\ 
  &   &   &   &   &   &   &   &   &    &     &    &    \\ 
    14      & 15 & 16 & 17 & 18 & 19          & 20 & 21 & 22 & 23 & 24 & 25 & 26 \\
\textbf{N}  & O  & P  & Q  & R  & \textbf{S}  & T  & U  & V  & X  & Y  & W  & Z \\
&   &   &   &   &   &   &   &   &    &     &    &    \\ 
    27      & 28 & 29 & 30 & 31          & 32 & 33 & 34          & 35 & 36 &  37 & 38 & 39 \\  
\overline{A} & \overline{B} & \overline{C} & \overline{D} & \overline{E} & \overline{F} & \overline{G} & \overline{H} & \overline{I} & \overline{J}  &  \overline{K}  & \overline{L}  & \overline{M}  \\
\end{array}
\]

Vejamos como encriptar a palavra ``ABOBORA''. Iniciamos o processo escrevendo a mensagem. Ao lado de cada letra da mensagem aparece entre par\^enteses o n\'umero que indica a sua posi\c{c}\~ao. Abaixo da mensagem escrevemos as letras da chave criptogr\'afica, repetindo-as de forma c\'iclica quando necess\'ario. Analogamente, ao lado de cada letra da chave aparece entre par\^enteses o n\'umero da posi\c{c}\~ao ocupada de cada letra, e o sinal de soma indica que devemos avan\c{c}ar aquele n�mero de posi��es. Ao final do processo aparecem as letras encriptadas. Entre par�nteses est� a posi\c{c}\~ao resultante da combina\c{c}\~ao das posi\c{c}\~oes da mensagem e da chave.   

\[
\begin{array}{lllllll||l}
     A (1)  &      B (2)  &      O (15) &      B (2)  &      O (15) &     R (18)  &    A (1)	 & \textrm{Mensagem}  \\
\downarrow  & \downarrow  & \downarrow  & \downarrow  & \downarrow  & \downarrow  & \downarrow &\\ 
    S (+19) &     E (+5)  &     N (+14) &     H  (+8) &     A (+1)  &    S (+19)  &   E (+5)   &\textrm{Chave}  \\
\downarrow  & \downarrow  & \downarrow  & \downarrow  & \downarrow  & \downarrow  & \downarrow & \\
		 T (20) &      G (7)  &      C (29) &      J (10) &     P (16)  &    K (37)   &    F (6)   & \textrm{Mensagem encriptada}  \\
\end{array}
\]
  
Observe que a encripta\c{c}\~ao polialfab\'etica \'e mais dif\'icil de ser quebrada que a monoalfab\'etica uma vez que letras iguais n\~ao t\^em, necessariamente, a mesma encripta\c{c}\~ao. Observe que neste tipo de criptografia o emissor precisa passar a chave para o receptor da mensagem de forma segura para que o receptor possa decifrar a mensagem, isto \'e, a chave usada para encriptar a mensagem \'e a mesma que deve ser usada para decifrar a mensagem. Veremos que esse \'e justamente o ponto fraco nesse tipo de encripta\c{c}\~ao pois usa a chamada \textit{chave sim\'etrica}, ou seja, a chave usada pelo emissor para codificar a mensagem \'e a mesma usada pelo receptor para decodificar a mensagem. Nesse processo, a chave deve ser mantida em segredo e bem guardada para garantir que o c\'odigo n\~ao seja quebrado e isso requer algum tipo de contato f\'isico entre emissor e receptor da mensagem.

Durante a  Primeira Guerra Mundial o contato f\'isico para a troca de chaves era complicado, e isso estimulou a cria\c{c}\~ao de m\'aquinas autom\'aticas de criptografia. O \textit{Enigma} foi uma dessas m\'aquinas e era utilizada pelos alem\~aes tanto para criptografar como para descriptografar c\'odigos de guerra. Semelhante a uma m\'aquina de escrever, os primeiros modelos foram patenteados por Arthur Scherbius em 1918. Essas m\'aquinas ganharam popularidade entre as for\c{c}as militares alem\~aes devido a facilidade de uso e sua suposta indecifrabilidade do c\'odigo. 

O matem\'atico Alan Turing foi o respons\'avel por quebrar o c\'odigo dos alem\~aes durante a Segunda Guerra Mundial. A descoberta de Turing mostrou a fragilidade da criptografia baseada em chave sim\'etrica e colocou novos desafios \`a criptografia. O grande problema passou a ser a quest\~ao dos protocolos, isto \'e, como transmitir a chave para o receptor de forma segura sem que seja necess\'ario o contato f\'isico entre as partes? 

Em 1949, com a publica\c{c}\~ao do artigo \textit{Communication Theory of Secrecy Systems} \cite{shannon} de Shannon, temos a inaugura\c{c}\~ao da criptografia moderna. Neste artigo ele escreve matematicamente que cifras teoricamente inquebr\'aveis s\~ao semelhantes as cifras polialfab\'eticas. Com isso ele transformou a criptografia que at\'e ent\~ao era uma arte em uma ci\^encia.

Em 1976 Diffie e Hellman publicaram \textit{New Directions in Cryptography} \cite{newdirections}. Neste artigo h\'a a introdu\c{c}\~ao ao conceito de \textit{chave assim\'etrica}, onde h\'a chaves diferentes entre o emissor da mensagem e seu receptor. Com a assimetria de chaves n\~ao era mais necess\'ario um contato t\~ao pr\'oximo entre emissor e receptor. Neste mesmo artigo \'e apresentado o primeiro algoritmo de criptografia de chave assim\'etrica ou como \'e mais conhecido nos dias atuais \textit{Algoritmo de Criptografia de Chave P\'ublica}, o protocolo de Diffie-Hellman.

Um dos algoritmos mais famosos da criptografia de chave p\'ublica \'e o \textit{RSA}(RIVEST et al, 1983) \cite{rivest}, algoritmo desenvolvido por Rivest, Shamir e Adleman. Este algoritmo se tornou poular por estar presente em muitas aplica\c{c}\~oes de alta seguran\c{c}a, como bancos, sistemas militares e servidores de internet.

Para que se possa compreender por completo o algoritmo faz-se necess�rio possuir alguns conhecimentos em Teoria de n\'umeros como fatora\c{c}\~ao e aritm\'etica modular. Estes conhecimentos ser\~ao apresentados mais adiante neste trabalho.

No algoritmo RSA existe uma chave p\'ublica $n$, que \'e a multiplica\c{c}\~ao dos primos $p$ e $q$. O emissor E codifica a mensagem usando um n\'umero primo $p$. Em seguida E envia publicamente a mensagem codificada junto com a chave $n$ para o receptor R. R possui o n\'umero $q$, que juntamente ao n\'umero $n$ servem para decodificar a mensagem. 

Embora a quebra do RSA seja aparentemente simples, bastando fatorar $n$ para descobrir seus fatores, o grande problema \'e na realidade computacional, pois usa-se como $p$ e $q$ n\'umeros primos muito altos, pr\'oximos a $2^{512}$. Com um n\'umero t\~ao alto um computador comum levaria bem mais que uma vida humana para decifrar a mensagem.

Com base nestes conhecimentos sobre criptografia, temos que o objetivo deste trabalho \'e analisar a viabilidade de uma criptografia inspirada pelo algoritmo RSA cl\'assico, a qual substitui os n\'umeros primos pelo conjunto denominado de \textit{primos de Gauss} \cite{intGauss}, resultando, assim, no que chamamos por \textit{criptografia RSA gaussiana}. Para que tal algoritmo seja vi\'avel \'e necess\'ario adaptar uma s\'erie de resultados relativas aos n\'umeros primos aos n\'umero primos de Gauss. Dessa forma, nossa tarefa ser\'a adaptar tanto quanto o poss\'ivel os primos de Gauss \`as demosnta\c{c}\~oes desses teoremas.

Como se trata de uma proposta inovadora, deixamos para trabalhos futuros uma an\'alise comparativa entre as criptografias RSA cl\'assica e a RSA gaussiana.        
\chapter {Um passeio pela Teoria de N\'umeros}
\label{Num}


\hspace{7mm}A teoria de n\'umeros \'e umas das mais antigas \'areas da matem\'atica e dedica-se ao estudo relacionado \`as propriedades relativas aos n\'umeros inteiros tais como a quest\~ao da fatora\c{c}\~ao, m\'aximo divisor comum, entre outras. Ao longo deste cap\'itulo mostraremos os principais resultados de teoria de n\'umeros essenciais para a compreens\~ao do m\'etodo de criptografia de chave p\'ublica conhecido como RSA.



\section{N\'umeros Primos e Fatora\c{c}\~ao \'Unica}

\hspace{7mm}Os n\'umeros primos ocupam lugar importante tanto na teoria de n\'umeros quanto na criptografia RSA: na primeira por serem capazes de gerar todos os elementos do conjunto dos n\'umeros inteiros e suas consequentes propriedades; na segunda pelo fato de formarem um conjunto infinito e isso permite que se tome um primo de dimens\~ao extrondosa para codificar uma mensagem, consequentemente dificultando que o c\'odigo seja decifrado por terceiros em tempo razo\'avel,como mostraremos ao longo deste trabalho.

Os primos atuam como \'atomos dentro do conjunto dos n\'umeros inteiros no sentido em que todo n\'umero pode ser escrito como um produto de primos. Esse fato \'e consequencia do chamado \textit{Teorema da Fatora\c{c}\~ao \'Unica} tamb\'em conhecido por \textit{Teorema Fundamental da Aritm\'etica}. Al\'em de ser um resultado fundamental para a teoria de n\'umeros, ele tamb\'em \'e um dos pilares da criptografia RSA, pois a decodifica\c{c}\~ao de uma mensagem vai passar pela fatora\c{c}\~ao de um n\'umero, e para demonstrar esse teorema \'e preciso ter a disposi\c{c}\~ao o chamado \textit{Teorema de Divis\~ao}. 


\begin{Th}
[Teorema de divis\~ao]
Sejam $a$ e $b$ inteiros positivos. Existem n\'umeros inteiros $q$ (quociente) e $r$ (resto) tais que:	
	\begin{center}
		$a=bq+r$ e $0\leq r <b$
	\end{center}
Al\'em disso, os valores de $q$ e $r$ satisfazendo as rela\c{c}\~oes acima s\~ao \'unicos.
\end{Th} 

\noindent{\textbf{\textit{Demonstra\c{c}\~ao}}}\\
Confira em Coutinho \cite{coutinho}, se\c{c}\~ao 3 do cap\'itulo 1, p. 22.
\hfill\newline

O teorema acima faz duas afirma\c{c}\~oes: a primeira que o quociente e o resto da divis\~ao sempre existem; a segunda, que o quociente e o resto s\~ao \'unicos. A garantia da unicidade \'e o ponto crucial na aplica\c{c}\~ao \`a criptografia RSA, pois assim temos a garantia de que uma mensagem possa ser decodificada de maneira \'unica. Um outro resultado 
igualmente importante \'e o \textit{Algoritmo de Euclides} mais conhecido como m\'etodo para se calcular o m\'aximo divisor comum entre dois n\'umeros. Esse resultado \'e importante para definir o que entendemos por n\'umeros primos e consequentemente para o teorema da fatora\c{c}\~ao \'unica que mostra como expressar um n\'umero em fatores primos de forma \'unica. Para este trabalho vamos precisar da vers\~ao estendida desse m\'etodo.

\begin{Th}[Teorema do M\'aximo Divisor Comum]
Sejam $a$ e $b$ inteiros positivos e seja $d$ o m\'aximo divisor comum entre $a$ e $b$. Existem
inteiros $\alpha$ e $\beta$ tais que:
	$$\alpha\cdot a+\beta\cdot b=d$$
\end{Th}

Diferente do teorema da divis\~ao, o teorema do m\'aximo divisor comum n\~ao garante a unicidade com rela\c{c}\~ao aos inteiros $\alpha$ e $\beta$. Isso acaba sendo um complicador para a criptografia RSA, mas veremos que esse problema acaba sendo contornado por temos a disposi\c{c}\~ao um m\'etodo eficiente para calcular esses n\'umeros.

Tendo os resultados acima a nossa disposi\c{c}\~ao podemos, agora, definir o que entendemos por n\'umeros primos para ent\~ao atingir nossa meta com este cap\'itulo: o Teorema da Fatora\c{c}\~ao \'Unica.

\begin{Df}
Um n\'umero inteiro $p$ \'e \textit{primo} se $p\neq \pm 1$ e os \'unicos divisores de $p$ s\~ao $\pm 1$ e $\pm p$. 
\end{Df} 

S\~ao exemplos de n\'umeros primos: $\pm 2$, $\pm 3$, $\pm 5$, $\pm 7$, $\pm 11$, $\pm 13$, etc.

Um n\'umero inteiro, diferente de $\pm 1$, que n\~ao \'e primo \'e chamado \textit{composto}. Observe que os n\'umeros $1$ e $-1$ n\~ao s\~ao nem primos e nem compostos. A exclus\~ao desses n\'umeros do conjunto dos primos \'e para garantir a unicidade da fatora\c{c}\~ao no teorema a seguir. Um outro aspecto a se destacar acerca desse par de n\'umeros \'e que eles s\~ao os \'unicos que admitem inverso multiplicativo, que vir\'a a ser explicado mais adiante.

\begin{Th}
[Teorema da Fatora\c{c}\~ao \'Unica]\label{fat.unica} 
Dado um inteiro positivo $n\geq 2$ podemos sempre escrev\^e-lo, de modo \'unico, na forma
$$n=p_{1}^{e_1}\dots p_{k}^{e_k}$$
onde $1<p_1<p_2<p_3<\cdots<p_k$ s\~ao n\'umeros primos e $e_1, \cdots, e_k$ s\~ao inteiros positivos.
\end{Th}

No teorema acima, os expoentes $e_i$, para $1\leq i\leq k$ s\~ao chamados de \textit{multiplicidades}, pois indicam a quantidade de vezes que um mesmo n\'umero primo ocorre na fatora\c{c}\~ao. A prova de que \'e sempre poss\'ivel encontrar os fatores de usados decompor o n\'umero em fatores primos consiste no procedimento para fatorar um n\'umero, esse procedimento \'e chamado \textit{Algoritmo de Euclides}: trata-se do m\'etodo que se aprende na escola para fatorar
um n\'umero e que n\~ao iremos detalhar aqui. Como bem sabemos, esse m\'etodo \'e bastante ineficaz quando pensamos em n\'umeros muito grande, pois depende de realizar uma sequ\^encia bem grande de divis\~oes, dependendo do n\'umero. A prova garante que o procedimento termina, mas o que se nota \'e que tal procedimento \'e muito ineficiente no sentido em que demanda muito tempo para se chegar a uma resposta dependendo do n\'umero de desejamos fatorar. Na literatura existem v\'arios algoritmos de fatora\c{c}\~ao que tornam o m\'etodo mais eficiente, no entanto nenhum desses m\'etodos funciona bem para todos os n\'umeros inteiros. A criptografia RSA aproveita a inefici\^encia do m\'etodo para fatorar um n\'umero para garantir a seguran\c{c}a do seu sistema. \'E um problema em aberto saber se existe ou n\~ao um m\'etodo r\'apido para fatorar n\'umeros inteiros.  

A demonstra\c{c}\~ao do teorema acima requer uma s\'erie de resultados acerca de n\'umeros primos os quais detalharemos abaixo. O interesse em apresentar as demonstra\c{c}\~oes de tais resultados \'e devido ao fato de estarmos interessados em adaptar tais provas para o primos de Gauss se quisermos implementar um m\'etodo de criptografia RSA baseado nesses primos. 

\begin{Th}\label{propriedade_de_primos}
Sejam $a$ e $b$ inteiros positivos e suponhamos que $a$ e $b$ s\~ao primos entre si.
\begin{enumerate}
\item Se $b$ divide o produto $ac$ ent\~ao $b$ divide $c$.
\item Se $a$ e $b$ dividem $c$ ent\~ao o produto $ab$ divide $c$.
\end{enumerate}
\end{Th}

\noindent{\textbf{\textit{Demonstra\c{c}\~ao}}}\\
\begin{enumerate}
\item Se $a$ e $b$ s\~ao primos entre si, ent\~ao o m\'aximo divisor comum entre $a$ e $b$ \'e 1, isto \'e,$mdc(a,b)=1$. Pelo algoritmo euclideano estendido, temos que existem inteiros $\alpha$ e $\beta$ tais que $\alpha\cdot a+\beta\cdot b=1$. Da\'i, multiplicando toda a express\~ao por $c$ temos que: $\alpha\cdot ac+\beta\cdot bc=c\quad\quad (1)$. Dado que $b$ divide $ac$ e $b$ divide $cb$, ent\~ao $b$ divide $\alpha\cdot ac+\beta\cdot bc$. Portanto, a partir da igualdade (1) temos que $b$ divide $c$.

\item Se $a$ divide $c$, ent\~ao existe $t\in \mathbb{Z}$ tal que $c=at$. Como, por hip\'otese, $b$ divide $c$ e $a$ e $b$ s\~ao primos entre si, ent\~ao $b$ tem que dividir $t$. Logo, para algum $t$ vale que $t=bk$. Portanto, $c=at=a(bk)=(ab)k$ \'e divis\'ivel por $ab$.  	
\end{enumerate}	
\hfill\newline

\begin{Th}[Propriedade Fundamental dos Primos]
Seja $p$ um n\'umero primo e $a$ e $b$ inteiros positivos. 
Se $p$ divide o produto $ab$ ent\~ao $p$ divide $a$ ou $p$ divide $b$. 
\end{Th}

\noindent{\textbf{\textit{Demonstra\c{c}\~ao}}}\\
Se $a$ e $p$ n\~ao forem primos entre si ent\~ao m\'aximo divisor comum entre eles \'e $p$, logo $p$ divide $a$. Suponhamos que $a$ e $p$ s\~ao primos entre si, isto \'e, $mdc(p,a)=1$. Como, por hip\'otese, $p$ divide $a\dot b$, ent\~ao pelo Teorema \ref{propriedade_de_primos} segue que $p$ divide $b$.
\hfill\newline

Estamos interessados, agora, em mostrar que a lista de primos \'e infinita, para tal vejamos o seguinte resultado intermedi\'ario.

\begin{Th}[Exist\^encia de Divisor Primo]\label{divisor_primo}
Se $n$ \'e um n\'umero inteiro positivo composto, ent\~ao $n$ tem um divisor primo
$p$ tal que $p\leq\sqrt{p}$.
\end{Th}

\noindent{\textbf{\textit{Demonstra\c{c}\~ao:}}}\newline
Se $n$ \'e um n\'umero composto e positivo, podemos supor que $n=a\cdot b$, com $1<a\leq b$.
\begin{enumerate}
\item De $1<a$, temos que existe um primo $p$ que divide $a$ (Teorema~\ref{fat.unica}), com $p\leq a$, da\'i $p^2\leq a^2$.
\item De $a\leq b$ temos que $a^2\leq a\cdot a=n$
\end{enumerate}
De (1) e (2) segue que $p^2\leq n$, logo $p\leq\sqrt{n}$.
\hfill \newline

\begin{Th}[Infinidade de Primos]\label{inf.primos}
Existe uma quantidade infinita de n\'umeros primos.
\end{Th}

\noindent{\textbf{\textit{Demonstra\c{c}\~ao:}}}\newline
Suponha, por redu\c{c}\~ao ao absurdo, que exite apenas uma quantidade finita de n\'umeros primos: $p_1, p_2,\cdots p_n$. Tome $a=1+p_1\cdot p_2\cdot \cdots \cdot p_n$ um n\'umero inteiro. Claramente $a>p_i$ para cada $1\leq i\leq n$, ent\~ao $a$ deve ser um n\'umero composto, caso contr\'ario a 
lista acima estaria incompleta. Dessa forma, pelo Teorema\ref{divisor_primo} existe um primo $p_i$ tal que divide $a$. Mas se $p_i$ divide $a$, ent\~ao $p_1$ divide $1$ e $p_i$ divide $p_1\cdot p_2\cdot \cdots \cdot p_n$. Absurdo, pois o \'unico divisor de $1$ \'e ele mesmo. Portanto, \'e falso supor que a lista de primos seja finita, logo ela deve ser infinita.
\hfill\newline

Existe, ainda, um outro debate acerca dos n\'umeros primos: como gerar os n\'umeros primos. Existem diversos m\'etodos para gerar os primos, como por exemplo, o Crivo de Erat\'ostenes, o mais antigos deles, mas n\~ao envolve nenhuma f\'ormula espec\'ifica. No entanto, todos esses m\'etodos s\~ao ineficazes. Para mais detalhes sobre esse tema recomendamos a leitura de Criptografia, de Coutinho\cite{coutinho}.

Como mostramos, temos o Teorema \ref{fat.unica} que garante que um n\'umero possa ser decomposto
em fatores primos de forma \'unica e o Terema \ref{inf.primos} que garante uma infinidade de n\'umeros primos, no entanto, como dissemos, os procedimentos atrelados a esses resultados s\~ao todos muito ineficientes em termos computacionais. Para implementar a criptografia RSA vamos precisar de procedimentos mais eficazes e por essa raz\~ao ser\'a conveniente trabalhar com o conjunto de n\'umeros inteiros. Para isso vamos separar os n\'umeros inteiros em classes de equival\^encias, pois dessa forma ser\'a poss\'ivel operar com essas classes de forma semelhante como fazemos com os inteiros. Esse \'e justamente o tema da pr\'oxima se\c{c}\~ao.  


\section{Aritm\'etica Modular}

\hspace{7mm}Para compreender a intui\c{c}\~ao por tr\'as da aritm\'etica modular \'e interessante pensar na ideia de \textit{ciclicidade}, isto \'e, fatos que ocorrem ap\'os um determinado per\'iodo constante. Por exemplo, o nascer do sol \'e um evento que ocorre sempre ap\'os um ciclo de 24 horas; a data de seu anivers\'ario ocorre a cada ciclo de um ano. Trabalhar com ciclos requer que tenhamos uma nova forma de operar com n\'umeros, pois quando somamos 13 com 15 o resultado pode ser 4 se estivermos pensando em termos de horas, pois ap\'os 24 horas retornamos ao marco zero e reiniciamos a contagem para facilitar o reconhecimento da hora em quest\~ao. 

Quando mostramos o processo de codifica\c{c}\~ao e de decodifica\c{c}\~ao de um c\'odigo em 
\textit{cifras de substitui\c{c}\~ao monoalfab\'eticas} voc\^e deve ter notado que precisamos repetir o alfabeto a fim de podermos operar com as posi\c{c}\~oes ocupadas por uma determinada letra do alfabeto. A repeti\c{c}\~ao do alfabeto foi usada para mostrar as diferentes representa\c{c}\~oes das letras dentro do ciclo estipulado.

Podemos observar que os ciclos geram classes de n\'umeros, isto \'e, os n\'umeros 0, 24, 48, 72, etc. indicam todos o marco zero do rel\'ogio se estivermos iniciando a contagem das horas no marco zero. Esses n\'umeros formam a \textsl{classe da zero hora}. Da mesma forma podemos compor a \textsl{classe da uma hora}, \textsl{classe das duas horas} e assim por diante. No exemplo das posi\c{c}\~oes ocupadas pelas letras do alfabeto temos que a \textsl{classe da letra $A$} \'e composta pelos n\'umeros 1, 27, 53, etc. 

Nosso interesse est\'a voltado para a uniformiza\c{c}\~ao do modo de separar tais classes para podemos operar com essas classes e da\'i fazer uso dessa intui\c{c}\~ao de forma operacionalizada. 

\begin{Df}
	Uma classe \'e chamada de \textsl{equival\^encia} se satisfaz as seguintes propriedades:
	\begin{itemize}
		\item Reflexividade: $\forall x, x=x$; 		
		\item Simetria: se $x=y$, ent\~ao $y=x$; 
		\item Transitividade: se $x=y$ e $y=z$, ent\~ao $x=z$.
	\end{itemize}
\end{Df}   

Para exemplificar a defini\c{c}\~ao acima, cosidere o exemplo das horas: a propriedade reflexiva afirma que toda hora \'e igual a ela pr\'opria; a simetria afirma que se $0h$ \'e igual a $24h$, ent\~ao temos tamb\'em que $24h$ \' igual a $0h$; a transitividade afirma que $0h$ \'e igual a $24h$ e $24h$ \'e igual a $72h$, então $0h$ \'e igual a $72h$.

Seja $X$ um conjunto e $\sim$ uma rela\c{c}\~ao de equival\^encia definida em $X$. Denotamos por $\overline{X}$ a classe de equival\^encia de $x$ e escrevemos, em s\'imbolos, da seguinte forma:	
	                     $$\overline{x}=\{y\in X: y\sim x\}$$

Nosso interesse ser\'a separar em classe de equival\^encia os n\'umeros inteiros, dessa forma o $X$ representa o conjunto $\mathbb{Z}$ enquanto que $x$ representa um n\'umero inteiro, enquanto que $\sim$ representa alguma rela\c{c}\~ao estabelecida entre os n\'umeros, por exemplo o resto da divis\~ao pelo n\'umero 5. Dessa forma, podemos formar cinco classes distintas:

	\begin{itemize}
		\item Classe dos restos 0: $\overline{0}=\{0, 5, 10, 15, 20, \cdots\}$
		\item Classe dos restos 1: $\overline{1}=\{1, 6, 11, 16, 21, \cdots\}$
		\item Classe dos restos 2: $\overline{2}=\{2, 7, 12, 17, 22, \cdots\}$
		\item Classe dos restos 3: $\overline{3}=\{3, 8, 13, 18, 23, \cdots\}$
		\item Classe dos restos 4: $\overline{4}=\{4, 9, 14, 19, 24, \cdots\}$
	\end{itemize}

O conjunto das classes de equival\^encia em $X$ \'e chamado \textit{conjunto quociente de $X$ por $\sim$}. No nosso exemplo $\{\bar{0}, \bar{1}, \bar{2}, \bar{3}, \bar{4}\}$ representa conjunto quociente de $\mathbb{Z}$ pela divis\~ao por 5. Esse conjunto ser\'a denotado por $\mathbb{Z}_{5}$. Em termos gerais, o conjunto quociente dos n\'umeros inteiros \'e denotado por:

$$\mathbb{Z}_{n}=\{\overline{0}, \overline{1}, \overline{2}, \cdots, \overline{n-1}\}$$ 
\chapter {Aritm\'{e}tica Modular}
\label{Mod}

\section{Ciclos e Restos}	
\subparagraph{
Para podermos compreender a aritm\'etica modular, precisamos come\c{c}ar entendendo o conceito de ciclicidade, que s\~ao os fatos que ocorrem sempre ap\'os um determinado per\'iodo de tempo constante. Um bom exemplo deste conceito \'e o nascer do sol, que \'e um evento que ocorre sempre ap\'os um ciclo de {24} horas, assim como o dia de seu anivers\'ario ocorre uma vez a cada ciclo de um ano.
}
\subparagraph{
O mesmo tipo de evento \'e observado com o resto dos n\'umeros inteiros. Tomemos por exemplo os restos de divis\~ao pelo n\'umero inteiro {4}
}

\[
\begin{array}{ccccccccccccc}
  {Inteiro} & 1 & 2 & 3 & 4 & 5 & 6 & 7 & 8 & 9 & 10 &  11 & 12 \\  
	{Resto} & 1 & 2 & 3 & 0 & 1 & 2 & 3 & 0 & 1 & 2  &  3 & 0 \\ 
\end{array}
\]

\subparagraph{
\'E vis\'ivel que ap\'os {4} n\'umeros o resto tende a se repetir. O mesmo feito ocorre a qualquer n\'umero inteiro $n$, onde o ciclo se repetir\'a sempre a cada $n$ itera\c{c}\~oes.
}

\section{N\'{u}meros Primos Naturais}	

\section{Per\'{i}odo e Fatora\c{c}\~{a}o}

\section{Inverso Multiplicativo}

\chapter {Inversos Modulares}
\label{InvMod}

\section{Inversos modulares}	
\subparagraph{
Nosso objetivo com o decorrer deste cap\'itulo \'e o de explicar a opera\c{c}\~ao matem\'atica mais importante para para o algoritmo RSA. Para podermos comprend\^e-la vamos relembrar do cenceito ensinado no col\'egio de inverso multiplicativo, que consiste em obter o n\'umero que multiplicado a um n\'umero $n$ qualquer resulte em $1$. A opera\c{c}\~ao do inverso modular parte do mesmo princ\'ipio.
}
\subparagraph{
Vamos supor que queremos obter o inverso modular de $6$ para o m\'odulo $7$, o que n\'os teremos que fazer ent\~ao \'e encontrar qual o n\'umero que multiplicado por $6$ tem resto $1$ quando dividido por $7$. Come\c{c}amos pelo $1$, teremos que $6 \cdot 1 = 6$, $6 \equiv 6 (mod7)$. Com $2$ o resultado ser� $12$, logo $12 \equiv 5 (mod7)$, que para n\'os tamb\'em n\~ao serve. Tentando o $3$ obtemos $4$ e com $4$ obtemos $3$. Com o $5$ nosso retorno ser\'a $2$. Finalmente quando chegamos ao $6$ n\'os temos que $6 \cdot 6 = 36$, $36 \equiv 1 (mod 7)$. Com isso podemos concluir que o inverso multiplicativo de $6$ no m\'odulo $7$ \'e o pr\'opio $6$.
}
\subparagraph{
Para simplificar o que foi dito acima, podemos dizer a opera\c{c}\~ao de inverso multiplicativo no m\'odulo $n$ para $a$ consiste em encontar um n\'umero $a'$ tal que:
}
\[	
	\begin{array}{c}
		\textit{$a \cdot a' \equiv 1 (mod n)$}
	\end{array}
\]

\section{Inexist\^encia e exist\^encia de inversos}	

\subparagraph{
Antes de come\c{c}armos vamos tentar calcular o inverso multiplicativo de $2$ no m\'odulo $8$, vamos l\'a: 
}
\[	
	\begin{array}{c}
		\textit{$2 \cdot 0 \equiv 0 \not\equiv 1(mod 8)$}\\
		\textit{$2 \cdot 1 \equiv 2 \not\equiv 1(mod 8)$}\\
		\textit{$2 \cdot 2 \equiv 4 \not\equiv 1(mod 8)$}\\
		\textit{$2 \cdot 3 \equiv 6 \not\equiv 1(mod 8)$}\\
		\textit{$2 \cdot 4 \equiv 8 \not\equiv 1(mod 8)$}\\
		\textit{$2 \cdot 5 \equiv 0 \not\equiv 1(mod 8)$}\\
		\textit{$2 \cdot 6 \equiv 2 \not\equiv 1(mod 8)$}\\
		\textit{$2 \cdot 7 \equiv 4 \not\equiv 1(mod 8)$}\\
	\end{array}
\]

\subparagraph{
N\~ao encontramos nenhuma resposta pois, simplesmente, n\~ao h\'a. Antes que se pergunte o motivo de n\~ao tentarmos com n\'umeros maiores que $7$, \'e v\'alido lembrar que a partir do $8$ ter\'iamos a repeti\c{c}\~ao de resultados por conta das congru\^encias.
}
\subparagraph{
A opera\c{c}\~ao de inverso multiplicativo s\'o possui resultado em casos onde o n\'umero $n$ ao qual queremos calcular o inverso e o m'odulo s\~ao \textit{primos entre si}, ou seja, n\~ao possuam nenhum fator em comum. Por conta disso usamos os n\'umeros primos no algoritmo RSA.
}
\chapter {Teorema chin\^es do resto}
\label{TCR}

\section{Introdu\c{c}\~ao a t\'ecnica}

\subparagraph{
Para sermos iniciados nesta t\'ecnica, vamos analisar o seguinte problema: Qual o menor inteiro que possui resto $1$ na divis\~ao por $3$ e resto $2$ na divis\~ao por $5$. Podemos vir a tranformar esse problema nas seguintes equa\c{c}\~oes:
}
\[	
	\begin{array}{c}
		\textit{$n = 3q_1 + 1$ e $n = 5q_2 + 2$}
	\end{array}
\]
\paragraph{
Essas equa\c{c}\~oes tamb\'em podem ser denotadas em forma modular como:
}
\[	
	\begin{array}{c}
		\textit{$n \equiv 1 (mod 3)$ e $n \equiv 2 (mod 5)$}
	\end{array}
\]
\paragraph{
Essa sa\'ida modular nos deixou com apenas uma vari\'avel, mas ainda n\~ao resolveu ao nosso problema. Para fazermos isso vamos substituir $n$ por $5q_2 + 2$, montando a seguine equa\c{c}\~ao modular:
}
\[	
	\begin{array}{c}
		\textit{$5q_2 + 2 \equiv 1 (mod 3)$}
	\end{array}
\]
\paragraph{
Como $5 \equiv 2(mod 3)$, substitu\'imos:
}
\[	
	\begin{array}{c}
		\textit{$ 2q_2 + 2 \equiv 1 (mod 3)$}
	\end{array}
\]
\paragraph{
Feito isso, passamos $2$ para o outro lado da equa\c{c}\~ao
}
\[	
	\begin{array}{c}
		\textit{$ 2q_2  \equiv -1 (mod 3)$}
	\end{array}
\]
\paragraph{
Como $-1 \equiv 2 (mod 3)$, n\'os substit\'imos novamente, e depois dividimos a equa\c{c}\~ao por $2$, e obtemos
}
\[	
	\begin{array}{c}
		\textit{$ q_2  \equiv 1 (mod 3)$}
	\end{array}
\]
\paragraph{
Com isso, conclu\'imos que
}
\[	
	\begin{array}{c}
		\textit{$ q_2  \equiv q_3 + 1 (mod 3)$}
	\end{array}
\]
\paragraph{
Sei que parece que mais uma equa\c{c}\~ao s\'o serve para tornar a resolu\c{c}\~ao mais complexa, mas vamos a reorganizar como
}
\[	
	\begin{array}{c}
		\textit{$ q_2 = 3q_3 + 1 $}
	\end{array}
\]
\paragraph{
Agora substitu\'imos
}
\[	
	\begin{array}{c}
		\textit{$n = 5(3q_3 + 1) + 2 = 15q_3 +7$}
	\end{array}
\]
\paragraph{
Feito isso, vamos por o $3$ em evid\^encia em todos os lugares, obtendo:
}
\[	
	\begin{array}{c}
		\textit{$n = 3(5q_3) +3(2) +1 = 3(5q_3 +2)+1$}
	\end{array}
\]
\paragraph{
Este procedimento foi feito apenas para provar que a equa\c{c}\~ao deixa resto 1 se dividida por 3, de forma an\'aloga, abaixo \'e mostrado como ela deixa resto $2$ quando dividida por $5$.
}
\[	
	\begin{array}{c}
		\textit{$n = 5(3q_3) +5(1) +2 = 5(3q_3 +1)+2$}
	\end{array}
\]
\subparagraph{
Ap\'os tudo isso feito ainda n\~ao possu\'imos a solu\c{c}\~ao final, mas j\'a sabemos que \'e um n\'umero da forma $15q_3 + 7$, substituindo $q_3$ or $0$, iremos obter $7$, que \'e o resultado procurado.
}

\section{O teorema}

\subparagraph{
O teorema chin\^es do resto \'e um procedimento tomado ara resolver sistema de congru\^encias, como o descrito acima. Ele foi descrito pela primeira vez pelo Manual de aritm\'etica do mestre Sun, por volta do s\'eculo III d.C. 
}
\subparagraph{
Para ver a defini\c{c}\~ao formal desse teorema, vamos considerar o sistema
}
\[	
	\begin{array}{c}
		\textit{$x \equiv a (mod n)$}\\
		\textit{$x \equiv b (mod m)$}\\
	\end{array}
\]
\paragraph{
nele, $n$ e $m$ s\~ao inteiros diferentes entre si. Tomemos $x_0$ como um n\'umero cappaz de satisfazer ambas as congru\^encia de forma simult\^anea e teremos:
}
\[	
	\begin{array}{c}
		\textit{$x_0 \equiv a (mod m)$}\\
		\textit{$x_0 \equiv b (mod n)$}\\
	\end{array}
\]
\paragraph{
Para podermos juntar ambas as equa\c{c}\~oes converteremos uma em equa\c{c}\~ao, nesse caso teremos 
}
\[	
	\begin{array}{c}
		\textit{$x_0 = a + m\cdot k$, com $k$ sendo um inteiro qualquer}\\
	\end{array}
\]
\paragraph{
Feito isso, chegaremos em
}
\[	
	\begin{array}{c}
		\textit{$a + m\cdot k \equiv b (mod n)$}\\
	\end{array}
\]
\paragraph{
que pode ser substitu\'ida por
}
\[	
	\begin{array}{c}
		\textit{$ m\cdot k \equiv (b-a) (mod n)$}\\
	\end{array}
\]
\subparagraph{
Agora vamos supor que $m$ e $n$ s\~ao primos entre si. Pelo teorema apresentado no cap\'ituo sobre inversos multiplicativos n\'os j\'a sabemos que eles possuem inverso multiplicativo um para o outro. Tomemos $m'$ como o inverso de $m$ no m\'odulo $n$. Multipplicando toda a congru\^encia por $m'$ obtemos
}
\[	
	\begin{array}{c}
		\textit{$ k \equiv m'\cdot(b-a) (mod n)$}\\
	\end{array}
\]
\chapter {Potencia\c{c}\~ao}
\label{Pot}

\section{Restos na potencia\c{c}\~ao}	
\subparagraph{
Ao longo deste cap\'itulo vamos estudar como tornar as opera\c{c}\~oes de potencia\c{c}\~ao e a obten\c{c}ao de seus restos calcal\'aveis de forma simpls e r\'apida. Para isso vamos dispor de algumas artimanhas matem\'aticas, al\'em do famoso \textit{Teorema de Fermat}.
}
\subparagraph{
Vamos come�ar tentando uma coisa que aparentemente \'e complexa, mas se converter\'a em uma opera\c{c}\~ao bem simples: Calcular o resto da divis\~ao de $10^{135}$ por $7$. Podemos fazer da forma tradicional, mas dividir um n\'umero t\~ao alto n\~ao seria nada pr\'atico.
}
\subparagraph{
O que faremos \'e tomar uma propiedade da multiplica\c{c}\~ao e da potencia\c{c}\~ao emprestadas, a do elemento neutro, nesse caso o $1$. O que faremos \'e calcular em qual pot\^encia $10$ \'e congruente a $1$ no m\'odulo $7$. Logo teremos a tabela:
}
\[
\begin{array}{c}
  \textit{$10^1 \equiv 3(mod7)$} \\  
	\textit{$10^2 \equiv 2(mod7)$}\\
	\textit{$10^3 \equiv 6(mod7)$}\\ 
	\textit{$10^4 \equiv 4(mod7)$}\\ 
	\textit{$10^5 \equiv 5(mod7)$}\\ 
	\textit{$10^6 \equiv 1(mod7)$}\\ 
\end{array}
\]
\subparagraph{
Como sabemos agora que $10^6$ \'e o n\'umero que quer\'iamos, vamos decompor o $135$ em raz\~ao de $6$ e teremos que $135 = (6 \cdot 22)+3$, essa express\~ao nos levar\'a a seguinte congru\^encia:
}
\[
\begin{array}{c}
  \textit{$10^{135} \equiv (10^6)^{22} \cdot 10^3 \equiv (1)^{22} \cdot 10^3 \equiv 10^3 \equiv 6 (mod7)$} \\  
\end{array}
\]
\subparagraph{
Agora n\'os vamos deixar essa opera\c{c}\~ao um pouco mais complexa, ao passo que vamos calcular o resto por $31$ de $2^{124512}$. Vamos pelo mesmos caminho que anteriormente, buscando a pot\^encia de $2$ que \'e congruente a $1$ no m\'odulo $31$. Obtemos:
}
\[
\begin{array}{c}
  \textit{$2^1 \equiv 2(mod31)$} \\  
	\textit{$2^2 \equiv 4(mod31)$}\\
	\textit{$2^3 \equiv 8(mod31)$}\\ 
	\textit{$2^4 \equiv 16(mod31)$}\\ 
	\textit{$2^5 \equiv 1(mod31)$}\\ 
\end{array}
\]
\subparagraph{
Vamos dividir $124512$ por $5$ e obteremos $4016$ com resto $2$, obtendo assim que $2^{124512} \equiv 2^2 \equiv 4 (mod31)$.
}
\subparagraph{
Tornando um pouco mais dif\'icil, podemos calcular o resto de $2^{13}^{98765}$, \'e descobrir o resto de ${13}^{98765}$ por $5$, podemos dizer que ${13}^{98765} \equiv {3}^{98765}(mod 5)$ como se sabe que $3^4 = 81 \equiv 1(mod 5)$, podemos usar isso em nosso favor, pois teremos ${3}^{98765}\equiv {3}^{4\cdot24691 + 1}\equiv 3 (mod 5)$, logo o resultado de ${13}^{98765}$ \'e um n\'umero da forma $5q'+3$.
}
\subparagraph{
Como isso, n\'os podemos dizer que $2^{13}^{98765} \equiv 2^{5q'+3} \equiv 2^{5q'}\cdot{2^3}\equiv {1}^{q'}\cdot{2^3} \equiv 8 (mod 31)$.
}
\section{O teorema de Fermat}
\subparagraph{
	\textit{Teorema de Fermat} - Se $p$ \'e um n\'umero primo e $a$ \'e um inteiro n\~ao divis\'ivel por $p$, ent\~ao:
}
\[
\begin{array}{c}
  \textit{$a^{p-1}\equiv 1(mod p)$} \\  
\end{array}
\]
\subparagraph{
Embora esse seja denominado como o pequeno teorema de Fermat, ele possui suma import\^ancia para o algoritmo RSA cl\'assico. Vamos apresentar abaixo uma de suas demonstra\c{c}\~oes.
}
\subparagraph{
Sabemos que os poss\'iveis res\'iduos no m\'odulo $p$ s\~ao todos os inteiros entre $1$ e $p-1$. Vamos multiplic\'a-los por $a$, obtendo assim: 
}
\[
\begin{array}{c}
  \textit{$a \cdot 1,a \cdot 2,a \cdot 3,...,a \cdot (p-1)$} \\  
\end{array}
\]
\subparagraph{
Vamos levar em conta que $r_1 \equiv a\cdot 1(mod p)$,$r_2 = a\cdot 2(mod p)$, e assim por diante at\'e $r_{p-1} = a\cdot (p-1)(mod p)$. Tomemos par $r_k$ e $r_l$ um par de inteiros $k$ e $l$ que est\'a entre $1$ e $p-1$. Com isso teremos:
}
\[
\begin{array}{c}
  \textit{$a \cdot k \equiv k \equiv l \equiv a\cdot l (mod p)$} \\  
\end{array}
\]
\paragraph{
que equivale \`a:
}
\[
\begin{array}{c}
  \textit{$a \cdot k \equiv a\cdot l (mod p)$} \\  
\end{array}
\]
\subparagraph{
Se viermos a cancelar pela equival\^encia, obteremos que $k \equiv l (mod p)$, mas sendo $k$ e $l$ positivos, inteiros e menores que $p$, estes s� podem ser congruentes se forem iguais, logo se:
}
\[
\begin{array}{c}
  \textit{$r_k = r_l$ ent�o $k = l$} \\  
\end{array}
\]
\subparagraph{
Isto demonstra que $r_1, r_2, r_3,...r_{p-1}$ s�o $p-1$ res\'iduos n\~ao nulos de m\'odulo $p$, que ser�o $1, 2, 3, ..., p-1$, o que nos permite dizer que a primeira sequ�ncia n\~ao \'e nada al\'em de um reordenamento da segunda. Com isso podemos dizer que:
}
\[
\begin{array}{c}
  \textit{$r_1 \cdot r_2 \cdot r_3\cdot...\cdot r_{p-1} = 1 \cdot 2 \cdot 3\cdot ...\cdot p-1$} \\  
\end{array}
\]
\paragraph{
Sabendo disso vemos que:
}
\[
\begin{array}{c}
  \textit{$ a^{p-1}(1 \cdot 2 \cdot 3\cdot ...\cdot p-1) \equiv (1 \cdot 2 \cdot 3\cdot ...\cdot p-1) (mod p)$} \\  
\end{array}
\]
\paragraph{
E apenas cortando os fatores iguais:
}
\[
\begin{array}{c}
  \textit{$ a^{p-1} \equiv 1(mod p)$} \\  
\end{array}
\]
\paragraph{
Provando assim o Teorema de Fermat.
}
\section{Aplicando o teorema de Fermat}
\paragraph{
Antes de prosseguirmos para o pr\'oximo cap\'itulo, vamos utilizar o Teorema de Fermat para resolver uma congru\^encia. Neste caso vamos tentar descobrir quem \'e congruente a $3^{1034}^{2}$ no m\'odulo $1033$. Como $1033$ \'e primo n\'os podemos usar o teorema de Fermat. Neste caso teremos que:
}
\[
\begin{array}{c}
  \textit{$ 3^{1032} \equiv 1 (mod 1033)$} \\  
\end{array}
\]
\subparagraph{
O que faremos agora consiste em ``dividir'' $1034$ por $1032$, de forma a obter o resto da divis\~ao. e com isso vamos veirficar que:
}
\[
\begin{array}{c}
  \textit{$ 1034^2 \equiv 2^2 \equiv 4 (mod 1033)$} \\  
\end{array}
\]
\paragraph{
e com essa simplifica\c{c}\~ao chegamos \`a:
}
\[
\begin{array}{c}
  \textit{$ 3^{1034} \equiv 3^{1032}\cdot q+ 4} \equiv (3^{1032})^{q} + 3^4 (mod 1033)$} \\  
\end{array}
\]
\subparagraph{
Agora com a simples aplica\c{c}\~ao do Teorema de Fermat, podemos chegar a conclus\~ao que: 
}
\[
\begin{array}{c}
  \textit{$ {3^{1034}}^2 \equiv 1 \cdot 81 (mod 1033)$} \\  
\end{array}
\]
\paragraph{
verificando assim que ${3^{1034}}^2$ deixa resto $81$ na divis\~ao por $1033$.
}
\section{Teorema de Fermat para pot\^encias compostas}
\subparagraph{
Embora aplicar o teorema de Fermat diretamente sobre os n\'umeros compostos n\~ao seja poss\'ivel, n\'os ainda podemos resolver a estas congru\^encias com o aux\'ilio do teorema chin\^es d restos, como veremos a seguir.
}
\subparagraph{
Para que possamos entender como resolver este problema com n\'umeros compostos vamos tentar resolver um problema n\'umerico, nesse caso o c\'alculo do m\'odulo de $2^{6754}$ por $1155$.
}
\subparagraph{
Nosso primeiro passo \'e fatorar o $1155$. Ao fim da fatora\c{c}\~ao vamos obter que $1155 = 3 \cdot 5 \cdot 7 \cdot 11$. Em seguida vamos aplicar o teorema de Fermat a cada um dos primos, obtendo assim:
}
\[
\begin{array}{c}
  \textit{$ 2^2 \equiv 1 (mod 3)$} \\  
	\textit{$ 2^4 \equiv 1 (mod 5)$} \\  
	\textit{$ 2^6 \equiv 1 (mod 7)$} \\  
	\textit{$ 2^{10} \equiv 1 (mod 11)$} \\  
\end{array}
\]
\subparagraph{
Agora dividimos $6754$ por $p-1$ para cada um dos m\'ultiplos:
}
\[
\begin{array}{c}
  \textit{$6754 = 2 \cdot 3377 $} \\  
	\textit{$6754 = 4 \cdot 1688 + 2$} \\  
	\textit{$6754 = 6 \cdot 1125 + 4$} \\  
	\textit{$6754 = 10 \cdot 675 + 4$} \\  
\end{array}
\]
\paragraph{
Em seguida substitu\'imos nas congru\^encias e as reduzimos
 }
\[
\begin{array}{c}
  \textit{$2^{6754} \equiv {2^{3377}}^{2} \equiv 1 (mod 3) $} \\  
	\textit{$2^{6754} \equiv {2^{1688}}^{4} \cdot 2^2 \equiv 1 \cdot 4 \equiv  4 (mod 5) $} \\  
	\textit{$2^{6754} \equiv {2^{1125}}^{6} \cdot 2^4 \equiv 1 \cdot 16\equiv  2 (mod 7) $} \\  
	\textit{$2^{6754} \equiv {2^{675}}^{10} \cdot 2^4 \equiv 1 \cdot 16\equiv  5 (mod 11) .$} \\  
\end{array}
\]
\paragraph{
Logo, nossa tarefa consiste em resolver o sistema
 }
\[
\begin{array}{c}
  \textit{$x \equiv 1 (mod 3) $} \\  
	\textit{$x \equiv 4 (mod 5) $} \\  
	\textit{$x \equiv  2 (mod 7) $} \\  
	\textit{$x \equiv  5 (mod 11)$} \\  
\end{array}
\]
\subparagraph{
Podemos resolver esse sistema usando o algoritmo chin\^es, vamos come�ar substituindo na primeira congru�ncia, onde $x = 3y + 1$, em seguida substitu\'imos $x$ por $y$ na segunda congru\^encia, tornando-a $3y + 1 \equiv 4 (mod 5)$, que equivale a $y \equiv 1 (mod 5)$ 
como $3$ \'e invers\'ivel no m\'odulo $5$ ele pode ser anulado na equa��o. Com isso temos $x = 4+15z$ que se substiutindo na terceira equa\c{c}\~ao e resolvendo obtemos $z \equiv 5 (mod 7)$, que significa que $x = 79 + 105t$. Finalmente substituindo na \'ultima equa\c{c}\~ao, teremos que $t \equiv 6 (mod 11)$, o que resulta em $x = 709+1155u$. Conclu�mos com isso que $26754 \equiv 709 (mod 1155)$. 
 }
\pagestyle{fancy}
\fancyhead[C]{\textsl{2. Sobre a Criptografia RSA}}
\fancyhead[R]{\thepage}
\fancyfoot[C]{}
\chapter {Criptografia RSA: Um estudo de caso}
\label{RSA}

A criptografia RSA tem suma import\^ancia para toda a comunica\c{c}\~ao moderna. Ela \'e t\~ao importante que a descoberta de uma forma de se desencript\'a-la colocaria em risco todo o sistema financeiro e portanto toda a sociedade. Ao longo deste cap\'itulo vamos ver como \'e seu funcionamento usando os conte\'udos do cap\'itulo anterior.

\section{Preparando-se para criptografar}

Para que o algoritmo RSA possa encriptar de forma eficiente, precisaremos seguir uma s\'erie de passos necess\'arios para que o RSA funcione, mas que ainda n\~ao s\~ao parte do algoritmo.

O primeiro passo \'e a convers\~ao das letras da mensagem em n\'umeros. A essa etapa chamaremos de pr\'e-codifica\c{c}\~ao. Para que o RSA venha a funcionar, precisamos seguir uma tabela como a apresentada abaixo:
\[
\begin{array}{ccccccccccccc}
A & B & C & D & E & F & G & H & I & J  &  K  & L  & M  \\ 
10 & 11 & 12 & 13 & 14 & 15 & 16 & 17 & 18 & 19 &  20 & 21 & 22 \\ 
\\
N & O  & P  & Q  & R  & S & T  & U  & V  & X  & Y  & W  & Z \\
23 & 24 & 25 & 26 & 27 & 28 & 29 & 30 & 31 & 32 & 33 & 34 & 35 \\
\end{array}
\]

Para representar espa\c{c}os vamos usar o 99. Alertamos que esta \'e uma tabela apenas com finalidade did\'atica, e, por isso h\'a v\'arios caracteres que n\~ao est\~ao sendo considerado. Como exemplo vamos pr\'e-encriptar o poema Amor, de Oswald de Andrade. O texto do poema a ser pr\'e-encriptado \'e o seguinte:

\begin{center}
Amor  \\ 
Humor. \\ 
\end{center}

Como primeiro passo vamos converter todas as letras em n\'umeros, resultando em:
 
\begin{center}
10 22 24 27 99 17 30 22 24 27
\end{center}

Feito isso, agrupamos o conjunto em um bloco \'unico de caracteres:

\begin{center}
10222427991730222427
\end{center}

Observe que inicamos a pr\'e-codifica\c{c}\~ao atribuindo a letra \textbf{A} o n\'umero 10 e n\~ao o 1. Isso evitar\'a ambiguidades, pois se tiv\'essemos iniciados com o 1 quando fossemos desencriptar por exemplo o n\'umero 11 n\~ao ser\'iamos capazes de diferenciar \textbf{AA} e \textbf{K}, por exemplo.

Nosso pr\'oximo passo nesta fase que antecede a encripta\c{c}\~ao consiste em definir quais ser\~ao os primos $p$ e $q$. Para nosso exemplo vamos usar $p=17$ e $q=23$. Como mencionado na Introdu\c{c}\~ao desta monografia, temos que $n = pq$, logo $n=391$.

O \'ultimo passo da pr\'e-encripta\c{c}\~ao consiste em quebrar o n\'umero que obtivemos acima em blocos menores. Esses blocos devem obedecer \`a duas regras b\'asicas: serem menores que $n$, ou no nosso exemplo $391$, pois iremos trabalhar com  o m\'odulo $391$ durante a encripta\c{c}\~ao, e n\~ao podem se iniciar por $0$, para n\~ao haver ambiguidade na desencripta\c{c}\~ao. Vejamos como nossa mensagem fica quando pr\'e-encriptada.

\begin{center}
$102$ | $224$ | $279$ | $91$ | $7$ | $30$ | $222$ | $42$ | $7$
\end{center}

Perceba que n\~ao h\'a rela\c{c}\~ao entre nenhum dos n\'umeros obtidos com um caractere espec\'ifico, o que torna imposs\'ivel a associa\c{c}\~ao de um n\'umero a uma letra por frequ\^encia de aparecimento. 

\section{Codificando e decodificando mensagens}

Encerrada a fase de pr\'e-codifica\c{c}\~ao vamos agora codificar a mensagem. Manteremos os valores e exemplos da se\c{c}\~ao anterior a fim de facilitar a compreens\~ao.

\subsection{Codificando uma mensagem}

A esta altura n\'os j\'a conhecemos o n\'umero $n$, que em nosso exemplo possui o valor de $391$. O outro n\'umero que iremos usar ser\'a o número $\textbf{e}$, tal que o $mdc(\textbf{e}, \phi(n)) = 1$. Para calcularmos o valor de $\phi(n)$, conforme j\'a comentamos pelo Teorema ~\ref{totiente} precisamos aplicar a seguinte receita:

$$\phi(n) = \phi(p \cdot q) = (p-1)(q-1)$$

Que em nosso exemplo resulta em:

$$\phi(391) = (17 - 1)(23 - 1) = 16 \cdot 22 = 352$$

Para determinarmos o $\textbf{e}$ basta escolher o menor primo ímpar tal que $mdc(\textbf{e}, 352) = 1$, que no nosso caso ser\'a o $3$. Ao conjunto $(n, \textbf{e} )$ denominamos chave de encripta\c{c}\~ao.

Vamos chamar o bloco codificado que iremos encriptar de $b$, lembrando que $b$ \'e um n\'umero inteiro menor que $n$. Também chamaremos o bloco ap\'os a codifica\c{c}\~ao de $C(b)$. Para obtermos $C(b)$ devemos aplicar a seguinte f\'ormula:

$$C(b) \equiv b^\textbf{e} \pmod{n} $$

Podemos para facilitar, dizer que $C(b)$ \'e o res\'iduo de $b^\textbf{e}$ pelo m\'odulo $n$. Vejamos como o procedimento funciona com o primeiro bloco de nossa mensagem, que possui o valor $102$. Para simplificar o nosso trabalho vamos utilizar as opera\c{c}\~oes modulares.

$$102^3 \equiv 24276 \equiv 34 \pmod{391}$$

Faremos o mesmo procedimento para todos nossos blocos:
\[
\begin{array}{cccccc}
224^3& \equiv& 11239424& \equiv& 129& \pmod{391}\\
279^3& \equiv& 21717639& \equiv& 326& \pmod{391}\\
91^3&  \equiv& 753571&   \equiv& 114& \pmod{391}\\
7^3&   \equiv& 343&      \equiv& 343& \pmod{391}\\
30^3&  \equiv& 27000&    \equiv& 21&  \pmod{391}\\
222^3& \equiv& 10941048& \equiv& 86&  \pmod{391}\\
42^3&  \equiv& 74088&    \equiv& 189& \pmod{391}\\
7^3&   \equiv& 343&      \equiv& 343& \pmod{391}\\
\end{array}
\]
Portanto, ``Amor Humor'', encriptado pelo RSA com as chaves $(391 , 3)$ \'e: 

\begin{center}
	 $34$ | $129$ | $326$ | $114$ | $343$ | $21$ | $86$ | $189$ | $343$
\end{center}

\subsection{Decodificando uma mensagem}

Para podermos desencriptar uma mensagem precisamos de dois n\'umeros. O primeiro \'e a chave p\'ublica $n$. O segundo n\'umero \'e $d$, o qual é o inverso de $e \pmod{\phi{n}}$ no inverso de $\textbf{e}$ no m\'odulo $\phi(n)$, ou seja, $\textbf{e} \cdot d \equiv 1 \pmod{\phi(n)}$ (vide Teorema \ref{inversao}). Para o nosso exemplo $d= 235$.

Agora que j\'a conhecemos $d$, podemos decodificar a mensagem. Vamos fazer isso em nosso primeiro bloco codificado, o qual possui o valor $34$. Para achar a resposta precisaremos calcular $D(34) \equiv 34^{235} \pmod{391}$. Esse c\'alculo seria praticamente imposs\'ivel sem o uso dos Teoremas: Chin\^es do Resto (Teorema \ref{chines}) e de Fermat (Teorema \ref{pequeno.fermat}).

Nosso primeiro passo ser\'a o de calcular $34^{235}$ nos m\'odulos $17$ e $23$, que s\~ao os primos resultantes da fatora\c{c}\~ao de $n$. Neste caso, come\c{c}amos com:

$$34 \equiv 0 \pmod{17}$$
$$34 \equiv 11 \pmod{23}$$

Assim teremos que $34^{235} \equiv 0^{235} \equiv 0 \pmod{17}$. Aplicando o Teorema de Fermat (Teorema \ref{pequeno.fermat}) na outra congr\^encia temos:

$$11^{235} \equiv (11^{22})^{10} \cdot 11^{15} \equiv 11^{15} \pmod{23}$$

Como $ 11 \equiv -12 \equiv -4 \cdot 3 \pmod{23}$, n\'os podemos afirmar que:

$$11^{235} \equiv 11^{15} \equiv -4^{15} \cdot 3^{15}\pmod{23}$$

Com isso, teremos:

$$4^{15} \equiv 2^{30} \equiv (2^{11})^2 \cdot 2^8 \equiv 2^8 \equiv 3 \pmod{23}$$
$$3^{15} \equiv 3^{11} \cdot 3^4 \equiv 3^4 \equiv 12 \pmod{23}$$

Concluindo assim:

$$112^{35} \equiv -4^{15} \cdot 3^{15} \equiv -3 \cdot 12 \equiv 10 \pmod{23}$$

Temos assim as congru\^encias $34^{235} \equiv 0 \pmod{17}$ e $34^{235} \equiv 10 \pmod{23}$. Com isso podemos aplicar o Teorema Chin\^es do Resto (Teorema \ref{chines}) no sistema:

$$x \equiv 0 \pmod{17}$$
$$x \equiv 10 \pmod{23})$$

Dessa forma, temos: 

$$10 + 23y \equiv 0 \pmod{17})$$

Obtendo assim:

$$6y \equiv 7 \pmod{17}$$

Por\'em, $3$ \'e o inverso de $6 \pmod{17}$, e por isso teremos:

$$y \equiv 3 \cdot 7 \equiv 4 \pmod{17}$$

Com isso iremos chegar at\'e $x = 10 + 23y = 10 + 23 \cdot 4 = 102$. Caso voc\^e venha a conferir na se\c{c}\~ao sobre codifica\c{c}\~ao de mensagens poder\'a confirmar o resultado. Os demais blocos podem ser decodificados da mesma forma, apenas n\~ao ser\~ao mostrados neste trabalho por necessitar de muitos passos, o que tornaria o cap\'itulo inutilmente mais longo.

\section{Provando a funcionalidade do RSA}

Ao longo desta se\c{c}\~ao vamos mostrar como o RSA funciona no processo de decodifica\c{c}\~ao. Para podermos fazer isso teremos que  verificar que:

$$b \equiv D(C(b)) \pmod{n}$$

J\'a vimos que $C(b) \equiv b^\textbf{e} \pmod{n}$ e $D(b) \equiv b^d\pmod{n}$. Se combinarmos ambas as congru\^encia obtemos:

$$D(C(b)) \equiv {(b^\textbf{e})}^d = b^{ed}\pmod{n}$$

Logo falta mostrar que $b^{\textbf{e}d} \equiv b \pmod{n}$. Como por defini\c{c}\~ao $\textbf{e}d \equiv 1 \pmod{(p-1)(q-1)}$,temos que:

$$\textbf{e}d = 1+k(p-1)(q-1)$$

Pelo Teorema Chin\^es do Resto (Teorema \ref{chines}) e levando em conta a express\~ao para obter $\textbf{e}d$ temos que:

$$b^{\textbf{e}d} \equiv b(b^{p-1})^{k(q-1)}$$

Dado que $p$ n\~ao divide $b$, podemos usar o Teorema de Fermat (Teorema \ref{pequeno.fermat}), de modo a obter:

$$b^{p-1} \equiv 1 \pmod{p}$$

Obtendo assim:

$$b^{\textbf{e}d} \equiv b \cdot (1)^{k(q-1)}\equiv b \pmod{p}$$

Mesmo considerando que $b$ seja m\'ultiplo de $p$, temos que $b$ e $b^{\textbf{e}d}$ s\~ao congruentes a $0$, logo nesse caso tamb\'em \'e v\'alida a congru\^encia, tendo assim: 
}

$$b^{\textbf{e}d} \equiv b \pmod{p}$$

Pelo mesmo m\'etodo podemos  obter $q$, obtendo o par: 

$$b^{\textbf{e}d} \equiv b \pmod{p}$$
$$b^{\textbf{e}d} \equiv b \pmod{q}$$

Veja que $b$ \'e solu\c{c}\~ao de: 

$$x \equiv b \pmod{p}$$
$$x \equiv b \pmod{q}$$

Pelo Teorema Chin\^es do Resto (Teorema \ref{chines}) esse sistema tem solu\c{c}\~ao igual:

$$b + p \cdot q \cdot t$$

Onde $t\in \mathbb{Z}$. Logo, como provamos anteriormente, temos que:

$$b^{\textbf{e}d} \equiv b + p \cdot q \cdot k$$

Para algum inteiro $k$. Caso tomemos $k=0$ teremos $b^{\textbf{e}d} \equiv b \pmod{p}$,  Comprovando $b = D(C(b))$.

\section{Discutindo a seguran\c{c}a do RSA}

Antes de passarmos para a generaliza\c{c}\~ao do RSA, vamos falar sobre a seguran\c{c}a do RSA. Vamos supor que algu\'em, que vamos chamar por Irineu, esteja com uma escuta em nossa troca de mensagens, tendo assim acesso tanto \`a mensagem codificada quanto \`a chave p\'ublica $n$. Vamos lembrar que $n$ \'e a multiplica\c{c}\~ ao dos primos $p$ e $q$. Sabendo disso, bastaria Irineu fatorar $n$ para obter $p$ e $q$ e depois descobrir $d$ para poder decodificar a mensagem, como j\'a foi explicado neste cap\'itulo.

Isso pode parecer muito simples, mas como j\'a mostramos na se\c{c}\~ao sobre fatora\c{c}\~ao, n\~ao h\'a um algoritmo conhecido que possa fazer isso de forma eficiente. O que ocorre \'e que um algoritmo que fa\c{c}a a fatora\c{c}\~ao de forma eficiente n\~ao tem sua inexist\^encia compravada, como j\'a comentamos no cap\'itulo \ref{Num}, embora sua inexist\^encia seja praticamente consenso pela comunidade acad\^emica, posto o fato deste problema ser qualificado em NP-dif\'icil.

O que iremos fazer no pr\'oximo cap\'itulo, consiste em analisar a viabilidade de uma varia\c{c}\~ao da criptografia RSA que acreditamos n\~ao ser completamente vulner\'avel caso uma fatora\c{c}\~ao eficiente no conjunto $\mathbb{Z}$ seja descoberta: a \textit{Criptografia RSA Gaussiana}.

\pagestyle{fancy}
\fancyhead[C]{\textsl{3. Inteiros e Primos de Gauss}}
\fancyhead[R]{\thepage}
\fancyfoot[C]{}



\chapter {Primeiros passos com o conjunto de Inteiros Gaussiano}
\label{IG}
At\'e o momento apenas os n\'umeros inteiros foram abordados neste projeto, mas para podermos entender a criptografia RSA Gaussiana \'e necess\'ario conhecer o conjunto dos n\'umeros inteiros gaussianos. Ao longo deste cap\'itulo vamos conhecer os inteiros e os primos gaussianos e suas propriedades aritm\'eticas b\'asicas. 

\section{Inteiros de Gauss e suas propiedades}

Os inteiros gaussianos, conjunto que a partir de agora iremos nos referenciar por $\mathbb{Z}[i]$, s\~ao um subconjunto dos n\'umeros complexos, relembrando que os n\'umeros complexos s\~ao os n\'umeros de forma $a+b\textbf{i}$, onde $a$ e $b$ s\~ao reais e $\textbf{i}$ \'e a $\sqrt{-1}$. A diferen\c{c}a entre o conjunto $\mathbb{Z}[i]$ e o conjunto $C$ reside no fato de em $\mathbb{Z}[i]$ $a$ e $b$ serem n\'umeros inteiros. Formalmente dizemos que os inteiros gaussianos s\~ao:

$$\mathbb{Z}[i]= \left\{a+b\textbf{i} | a,b \in \mathbb{Z}  \right\}, \textrm{ onde } \textbf{i}^2 = -1$$

Por $\mathbb{Z}[i]$ estar contido em $\mathbb{C}$, as opera\c{c}\~oes deste conjunto podem ser realizadas, por exemplo, se tomarmos $z_1= a + b\textbf{i}$ e $z_2= c + d\textbf{i}$ n\'os iremos obter:

$$z_1   +   z_2 = (a + c) + (b + d)\textbf{i}$$
$$z_1 \cdot z_2 = (ac - bd) + (ad + bc)\textbf{i}$$

Outra propiedade herdada \'e a dos elementos neutros, o $0 = 0 + 0\textbf{i}$ continua sendo o elemento neutro da adi\c{c}\~ao, enquanto o $1 = 1 + 0\textbf{i}$ tamb\'em continua sendo o elemento neutro da multiplica\c{c}\~ao. As propiedades associativa da adi\c{c}\~ao e da multiplica\c{c}\~ao, comutativa da adi\c{c}\~ao e multiplica\c{c}\~ao e distributiva tamb\'em s\~ao herdadas do conjunto complexo.

Repare que se considerarmos o plano complexo, os inteiros gaussianos ter\~ao uma marca\c{c}\~ao reticulada. Outro conceito importante para os inteiros gaussianos \'e a norma do n\'umero, ela \'e importante para auxiliar na defini\c{c}\~ao de um primo gaussiano, assim como s\~ao importantes os conceitos de n\'umero conjugado e n\'umero associado. Caso venhamos a tomar um n\'umero inteiro gaussiano de forma $a+b\textbf{i}$, sua norma ser\'a $a^2 +b^2$.

\begin{Df}
A norma de um n\'umero gaussiano \'e a soma dos quadrados de seus valores absolutos como n\'umero complexo. Ela \'e o resultado de:

$$N(a+b\textbf{i}) = a^2 + b^2 = (a+b\textbf{i})(a-b\textbf{i}),$$

onde o $(a-b\textbf{i})$ \'e a conjugado de $(a+b\textbf{i})$, tamb\'em denotado por $\overline{(a+b\textbf{i})}$.
\end{Df}

Uma das propiedades da norma \'e ser multiplicativa, ou seja, a norma de $N(zw)$ \'e igual a $N(z) \cdot N(w)$.

Os inteiros gaussianos possuem como unidades b\'asicas $\pm 1$ e $\pm \textbf{i}$. Caso venhamos a multiplicar um inteiro gaussiano x, teremos que $\pm x$ e $\pm x\textbf{i}$ sendo seus elementos associados.

\begin{Df}

Os elementos associados de um n\'umero $x$, tal que $x \in \mathbb{Z}[i]$, s\~ao $\pm x$ e $\pm x\textbf{i}$.

\end{Df}

Para chegarmos aos primos Gaussianos precisaremos demonstrar para o conjunto $\mathbb{Z}[i]$ uma s\'erie de resultados que j\'a s\~ao conhecidos do conjunto dos n\'umeros inteiros, como o funcionamento da divis\~ao e o teorema da fatora\c{c}\~ao \'unica.

Podemos definir a divisibilidade gaussiana por quando dizemos que $\beta$ divide $\alpha$, representado por $\beta | \alpha$ se $\alpha = \beta \gamma$, para qualquer $\gamma \in \mathbb{Z}[i] $. Nesse caso, $\beta$ \'e um fator de $\alpha$.

\begin{Th}\label{div_gaussiana1}

Um inteiro Gaussiano $\alpha = a+b\textbf{i}$ \'e dividido por um primo inteiro $c$ se e somente se $c|a$ e $c|b$ em $\mathbb{Z}$.

\end{Th}

\begin{proof}

Dizer que $c|(a+b\textbf{i})$ em $\mathbb{Z}$ \'e o mesmo que dizer que $a+b\textbf{i} = c(m +  n\textbf{i})$, para algum $m, n \in \mathbb{Z}$, que equivale a $a=cm$ e $b=cn$.

\end{proof}

Tomemos uma divis\~ao entre inteiros gaussianos, onde $\alpha$ \'e o dividendo, $\beta$ o divisor, $\gamma$ o quociente e $\rho$ o dividendo

\begin{Th}[Teorema da divis\~ao no conjunto gaussiano  ]  \label{divgauss}

Para $ \alpha, \beta \in \mathbb{Z} $ com $\beta \neq 0$ existe um $\gamma, \rho \in \mathbb{Z}[i]$ tal qual $\alpha = \beta \gamma + \rho$ e $N(\rho) < N(\beta)$. De fato, podemos escolher $\rho$  de forma que $N(\rho) \leq (1/2)N(\beta)$

\end{Th}

Agora que j\'a entendemos a divis\~ao, vamos definir o m\'aximo divisor comum no conjunto $\mathbb{Z}[i]$.

\begin{Th}[Algoritmo Euclidiano no conjunto gaussiano ]
\label{euclideszi}

Tomemos $\alpha , \beta \in \mathbb{Z}[i]$ e diferentes de $0$. Aplicamos recursivamente o teorema da divis\~ao em $\mathbb{Z}[i]$ (\ref{divgauss}), come\c{c}ando com esse par e fazendo com o resto uma  equa\c{c}\~ao com um novo dividendo e divisor no pr\'oximo caso, enquanto o resto for diferente de zero:

\[
\begin{array}{lcll}
\alpha & = & \beta \gamma_1 + \rho_1,  & N(\rho_1) < N(\beta)  \\
\beta  & = & \rho_1 \gamma_2 + \rho_2, & N(\rho_2) < N(\rho_1) \\
\rho_1 & = & \rho_2 \gamma_3 + \rho_3, & N(\rho_3) < N(\rho_2) \\
& \vdots &  &\\
\end{array}
\]

O \'ultimo elemento que n\~ao possui resto $0$ \'e divis\'ivel por todos os divisores comuns de $\alpha$ e $\beta$, sendo esse o maior divisor comum de $\alpha$ e $\beta$.

\end{Th}

Outra possibilidade para simplificar esta mesma opera\c{c}\~ao \'e o uso do algoritmo eucliadiano estendido, para o conjunto dos $\mathbb{Z}[i]$ esse algoritmo \'e chamado de Teorema de Bezout.

\begin{Cor} \label{cor}
	Para $\alpha$ e $\beta$ diferentes de $0$ e existentes no conjunto gaussiano, tomemos $\delta$ como o maior divisor comum pelo algoritmo euclidiano no conjunto gaussiano(\ref{euclideszi}). Qualquer divisor comum de $\alpha$ e $\beta$ \'e um divisor de $\delta$.
\end{Cor}

\begin{proof}
	Tomemos $\delta'$ como o maior divisor de $\alpha$ e $\beta$. Pelo algoritmo euclidiano no conjunto gaussiano(\ref{euclideszi}), $\delta' | \delta$ (pois $\delta'$ \'e divisor comum). Tendo que $\delta = \delta' \gamma$, ent\~ao:

$$N(\delta) = N(\delta')N(\gamma) \geq N(\delta')$$

Tendo $\delta'$ como o maior divisor comum, sua norma \'e a maior entre os divisores comuns, logo a inequa\c{c}\~ao $N(\delta) \geq N(\delta')$ tem de ser uma igualdade. Isso implica que $N(\gamma) = 1$, ent\~ao $\gamma = \pm 1$ ou $\pm \textbf{i}$. Ent\~ao $\delta$ e $\delta'$ s\~ao m\'ultiplos um do outro.

\end{proof}

\begin{Th} \label{teoprimosentresi}
	Sendo $\delta$ o maior divisor comum de dois inteiros gaussianos diferentes de zero $\alpha$ e $\beta$, ent\~ao $\delta = \alpha x + \beta y$ para qualquer $x, y \in \mathbb{Z}$ .
\end{Th}

\begin{proof}
Sendo $\delta$ escrito com uma combina\c{c}\~ao em $\mathbb{Z}[i]$ de $\alpha$ e $\beta$, ele n\~ao \'e afetado por substituir $\delta$ como o m\'ultiplo por uma unidade. Por isso o Corol\'ario \ref{cor}, n\'os apenas temos que provar que $\delta$ \'e o maior divisor comum pelo algoritmo euclidiano no conjunto gaussiano Para $\delta$, uma substitui\c{c}\~ao no algoritmo euclidiano mostra que $\delta$ \'e uma combina\c{c}\~ao em $\mathbb{Z}[i]$ de $\alpha$ e $\beta$. Todos os demais detalhes s\~ao id\^eticos aos da prova para inteiros.

\end{proof}

Podemos dizer que se $\alpha$ e $\beta$ possuem apenas as unidades como fatores em comum eles s\~ao primos entre si. 

\begin{Cor} \label{primosentresi}

Os inteiros gaussianos $\alpha$ e $\beta$ s\~ao primos entre si se e somente se podemos escrever:

$$1 = \alpha x + \beta y$$

para quaisquer $x, y \in \mathbb{Z}[i]$

\end{Cor}

\begin{proof}

Se $\alpha$ e $\beta$ s\~ao primos entre si, ent\~ao $1$ \'e o maior divisor de $\alpha$ e $\beta$, ent\~ao $1 = \alpha x + \beta y$ para qualquer $\, y \in \mathbb{Z}[i]$  pelo teorema \ref{teoprimosentresi}, por outro lado se $1 = \alpha x + \beta y$ para algum $x, y \in \mathbb{Z}[i]$, ent\~ao o m\'aximo divisor comum de $\alpha$ e $\beta$ \'e divisor de $1$, logo uma unidade, provando que $\alpha$ e $\beta$ s\~ao primos entre si.

\end{proof}


Agora n\'os vamos definir o que vem a ser um inteiro gaussiano primo e composto, al\'em de falarmos sobre a fatora\c{c}\~ao \'unica.

\begin{Df}
Se tomarmos um inteiro Gaussiano $\alpha$ com $N(\alpha) > 1$. \'e denominado \textit{composto} se o n\'umero possuir um fator diferente das unidades de $\mathbb{Z}[i]$ e dos conjugados do n\'umero. Caso ele n\~ao seja composto ele \'e denominado \textit{primo}.
\end{Df}

Agora que n\'os j\'a sabemos o que \'e um n\'umero primo e composto, podemos definir como eles s\~ao fatorados de forma \'unica pelo teoremas abaixo, as provas de ambas podem ser lidas na sess\~ao 6 de ``The Gaussian Integers'' em \cite{conrad}, a partir da p\'agina 13. As provas dos outros teoremas desta sess\~ao tamb\'em foram baseadas no trabalho de \cite{conrad}.

\begin{Th}
 Todo $\alpha \in \mathbb{Z}[i]$ com $N(\alpha) > 1$ \'e um produto de primos em $\mathbb{Z}[i]$
\end{Th}

\begin{Th}[Fatora\c{c}\~ao \'Unica no conjunto gaussiano] \label{fatunicagaussiana}
 Todo $\alpha \in \mathbb{Z}[i]$ com $N(\alpha) > 1$ possui uma \'unica fatora\c{c}\~ao baseada nos primos gaussianos com o formato:

	$$\alpha = \pi_1 \pi_2 \cdots \pi_{r} = \pi'_1 \pi'_2 \cdots \pi'_{s'} $$

onde os $\pi_i$ e os $\pi'_j$ s\~ao primos em $\mathbb{Z}[i]$, ent\~ao $r=s$ e cada membro de $\pi_i$ ap\'os uma renumera\c{c}\~ao adequada \'e um m\'ultiplo por uma unidade de $\pi'_i$.

\end{Th}

Para exemplificar o que foi dito pelo Teorema da Fatora\c{c}\~ao \'Unica (\ref{fatunicagaussiana}), tomemos o n\'umero $2$. Esse n\'umero \'e fatorado como $(1 + \textbf{i})(1 - \textbf{i})$, por\'em a propiedade da fatora\c{c}\~ao \'unica se estende aos elementos associados aos fatores, logo $2$ tamb\'em pode ser fatorado na forma $(-1 - \textbf{i})(-1 + \textbf{i})$. Essa forma consiste apenas na multiplica\c{c}\~ao pela unidade $-1$ dos fatores e n\~ao altera ao resultado final. O mesmo poder\'a ser obtido por qualquer outro elemento associado em qualquer outra fatora\c{c}\~ao.

Al\'em desses resultados apresentados acima outros ainda nos s\~ao necess\'arios para a realiza\c{c}\~ao de uma criptografia RSA Gaussiana, como uma aritm\'etica modular gaussiana e teorema an\'alogos ao teorema chin\^es do resto e ao Teorema de Fermat. No pr\'oximo cap\'itulo iremos discutir sobre a viabilidade ou n\~ao do algoritmo RSA Gaussiano, al\'em de conhecer alguns resultados relacionados a \'area pelo ponto de vista de outros pesquisadores do mesmo algoritmo.
\chapter {RSA Gaussiano e Conclus�es}
\label{RSAG}

\hspace{7mm}Chegamos ao \'ultimo cap\'itlo desta obra, aqui ser\'a debatido sobre tudo o que foi alcan�ado at\'e o momento no que diz respeito ao RSA Gaussiano, veremos como sua f\'ormula \'e planejada e o que ter\'a de ser feito no futuro para que este algoritmo se torne uma op��o entre os algoritmos criptogr\'aficos.

\section{O RSA Gaussiano e seus trabalhos futuros}

\hspace{7mm}Ao longo desse se\c{c}\~ao lhe ser\'a apresentado como o RSA Gaussiano dever\'a vir a funcionar. Para iniciarmos, teremos que, assim como na criptografia RSA fazer uma pr�-encripta\c{c}\~ao vindo a transformar todas as letras em n\'umero inteiros, da mesma forma que j\'a ocorre.

A segunda parte do processo, que \'e a encripta\c{c}\~ao depende de algumas garantias as quais ainda n\~ao possu\'imos, embora j\'a saibamos que a propiedade da fatora\c{c}\~ao \'unica \'e v\'alida para o conjunto $Z[i]$. Um dos meios para que possamos seguir consiste em definir a opera\c{c}\~ao de congru\^encia modular para o conjunto gaussiano, tamb\'em se fazem necess\'arias propiedades an\'alogas ao Teorema de Fermat e ao Teorema chin\^es do resto para que possamos seguir a mesma linha de encripta\c{c}\~ao do algoritmo RSA.

Para o RSA Gaussiano \'e planejada a encripta\c{c}\~ao com a f\'ormula:

$$C'(a) \equiv a^3 \pmod{n'}$$

Nesta f\'ormula $a$ seria o n\'umero a ser encriptado, $n'$ seria a chave p\'ublica gaussiana, derivada da multiplica\c{c}\~ao de $p'$ e $q'$, que s\~ao n\'umeros primos gaussianos. Os n\'umeros $p'$ e $q'$ s\~ao as chaves privadas de encripta\c{c}\~ao.

Para a desencripta\c{c}\~ao, embora ainda n\~ao tenhamos uma prova de seu funcionamento pelos motivos j\'a citados acima, ela \'e planejada em um primeiro momento pelo n\'umero $d'$, que pode manter sua f\'ormula similar a do RSA ou n\~ao. Caso ele n\~ao precise de mudan\c{c}as por conta do conjunto gaussiano, dever\'a ser da seguinte f\'ormula:

$$3d' \equiv 1 \pmod{(p'-1)(q'-1)}$$

Supondo que a f\'ormula de $d'$ acima seja v\'alida, teremos que para podermos concluir desencripta\c{c}\~ao a f\'ormula an\'aloga a original, que seria:}

$$D'(b) \equiv b^{d'} \pmod{n'}$$

Nesta f\'ormula temos que $b$ \'e um n\'umero encriptado pelo RSA Gaussiano, $d'$ a constante cuja f\'ormula foi mostrada anteriormente, $n'$ a chave p\'ublica e $D'(b)$ o resultado da desencripta\c{c}\~ao.
Nesta f\'ormula temos que $b$ \'e um n\'umero encriptado pelo RSA Gaussiano, $d'$ a constante cuja f\'ormula foi mostrada anteriormente, $n'$ a chave p\'ublica e $D'(b)$ o resultado da desencripta\c{c}\~ao.

Com isso conclu\'imos este projeto, aguardando que muito em breve tudo o que ficou como trabalho futuro aqui venha a ser realizado, muito obrigado por sua aten\c{c}\~ao a este projeto e esperamos que ele venha a ser uma fonte de inspira\c{c}\~ao para seus projetos futuros.


%%%%%%%%%%%%%CONSIDERA��ES FINAIS%%%%%%%%%%%%%%%%%%%%%%%%%%%%%%%%%%%%%%%%%%%%%%%%%%%%%%

\pagestyle{fancy}
\fancyhead[C]{\textsl{Considera��es Finais}}
\fancyhead[R]{\thepage}
\fancyfoot[C]{}

\addcontentsline{toc}{chapter}{Considera��es Finais}

\chapter*{Considera��es Finais}

Nesta sess\~ao faremos as \'ultimas considera\c{c}\~oes sobre a viabilidade do algoritmo RSA Gaussiano. Al\'em disso vamos ver o que outros pesquisadores j\'a est\~ao concluindo em suas pesquisas.

A primeira coisa que devemos prestar aten\c{c}\~ao \'e que no decorrer deste artigo n\~ao encontramos nada que impedisse a realiza\c{c}\~ao de uma criptografia RSA Gaussiana, mas como foi visto no cap\'itulo \ref{IG}, ainda faltam a comprava\c{c}\~ao de alguns teoremas matem\'aticos importantes para a realiza\c{c}\~ao da criptografia RSA Gaussiana. 

O material publicado por \cite{koval} e \cite{elkassar} nos leva a crer na viabiliadade do algoritmo. O que ocorre \'e que ambos possuem vis\~oes bem diferentes. \cite{koval} n\~ao defende o algoritmo, pois acredita que ele n\~ao acrescenta seguran\c{c}a ao algoritmo RSA, al\'em de deix\'a-lo menos pr\'atico. Abaixo citamos o trecho onde isso \'e afirmado:

\begin{quote}
``The extension of RSA algorithm into the field of Gaussian integers [...] is viable only if real primes p congruent to 3 modulo 4 are used [...]. The extended algorithm could add security only if breaking the original RSA is not as hard as factoring. Even in this case, it is not clear whether the extended algorithm would increase security. The Gaussian integer RSA is slightly less efficient than the original, therefore the original real integer RSA is more practical.''
\end{quote}

Enquanto isso, \cite{elkassar} defende o algoritmo Gaussiano por aumentar a seguran\c{c}a comparado ao cl\'assico, como pode ser lido abaixo:

\begin{quote}

``Arithmetic needed for the RSA cryptosystem in the domains of Gaussian integers and polynomials over finite fields were modified and computational procedures were described. There are advantages for the new schemes over the classical one. First, generating the odd prime numbers in both the classical and the modified methods requires the same amount of efforts. Second, the modified method provides an extension to the range of chosen messages and the trials will be more complicated. ''

\end{quote}

Baseado nos textos de ambos podemos concluir que al\'em da realiza\c{c}\~ao de tal algoritmo, outro problema a ser investigado em um trabalho futuro consiste na an\'alise de seguran\c{c}a e complexidade do algoritmo, visto que ainda n\~ao possu\'imos uma conclus\~ao definitiva sobre isso.
        %Considera��es Finais
%
%%%%%%%%%%%%%%%%%%%%%%%%%%%AP�NDICE 1: Ipcional%%%%%%%%%%%%%%%%%%%%%%%%%%%%%%%%%%%%%%%%%%%%

\addcontentsline{toc}{chapter}{Ap�ndice 1}

\chapter*{Ap�ndice 1\\ (Opcional)}

Exemplos dos mais interessantes  manuais de latex na rede s�o os
seguintes:


P�gina sobre criptografia do IME- USP Manuais de LaTeX em
portugu�s e ingl�s, inclusive com conversoires entre LaTeX e
outros formatos:
\\
http://www.ime.eb.br/~pinho/pessoal/latex/

Manual b�sico do IFGW- UNICAMP:
\\
http://www.ifi.unicamp.br/encontro/latex-exemplo.html
         %Ap�ndice 1: Opcional

\addcontentsline{toc}{chapter}{Bibliografia} %para aparecer a  bibliografia no �ndice
%%%%%%%%%%%%%%%%%%%%%%%%%%%%%%%%%%% Comandos para gerar a bibliografia em Portugu�s %%%%%%%%%%%%%%%%%%%%%%%%

\selectbiblanguage{brazil}
%\bibliographystyle{babplain}
\bibliographystyle{babalpha}
\bibliography{Bibliografia}

%%%%%%%%%%%%%%%%%%% Caso n�o gere a bibliografia por falta de pacotes rode com os comandos abaixo e retire o pacote \usepackage{babelbib} do in�cio do documento %%%%%%%%%%%

\bibliographystyle{alpha}
\nocite{*}
%\bibliography{Bibliografia}

%%%%%%%%%%%%%%%%%%%%%%%%%%%%%%%%%%%%%%%%%%%%%%%%%%%%%%%%%%%%%%%%%%%%%%%%%%%%%%%%%%%%%%%%%%%%%%%%%%%%%%%%%%%%%
\end{document}
