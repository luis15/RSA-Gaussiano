
%%%%%%%%%%%%%%%%%%%%%%%%%%%%%%%%%% DISSERTA��O OU TESE %%%%%%%%%%%%%%%%%%%%%%%%%%%%%%%%%%%%%%%%%
\documentclass[11pt]{report}
\usepackage{graphicx}
\usepackage[brazil]{babel}
\usepackage[latin1]{inputenc}
\usepackage{babelbib} % Esse pacote gera a bibliografia em Portugu�s. Acho que deve ser instalado algum pacote, n�o me recordo.
\usepackage{latexsym}
\usepackage{amsfonts}
\usepackage{amsthm}
\usepackage{amssymb}
\usepackage{amsfonts}
\usepackage{xypic}
\usepackage{ulem}
\usepackage[all]{xy}

\usepackage{makeidx} %cria o �ndice remissivo
\makeindex

\vfuzz2pt % Don't report over-full v-boxes if over-edge is small
\hfuzz2pt % Don't report over-full h-boxes if over-edge is small

% TEOREMAS -------------------------------------------------------
\theoremstyle{alpha}
\newtheorem{Th}{Teorema}[section]
\newtheorem{Cor}[Th]{Corol�rio}
\newtheorem{Le}[Th]{Lema}
\newtheorem{Pro}[Th]{Proposi��o}
\theoremstyle{definition}
\newtheorem{Df}[Th]{Defini��o}
\newtheorem{Obs}[Th]{Observa��o}
\newtheorem{Dig}[Th]{Digress�o}
\newtheorem{Ex}[Th]{Exemplo}
\newtheorem{Esc}[Th]{Esc�lio}
\newtheorem{Ass}[Th]{Asser��o}

% MATEMATICA E LOGICA -------------------------------------------

\newcommand{\lan}     {\langle}
\newcommand{\ran}     {\rangle}
\newcommand{\IN}      {\mathbb{N}}
\newcommand{\A}       {\mathcal{A}}
\newcommand{\B}       {\mathcal{B}}
\newcommand{\C}       {\mathcal{C}}
\newcommand{\D}       {\mathcal{D}}
\newcommand{\F}       {\mathcal{F}}
\newcommand{\I}       {\mathcal{I}}
\newcommand{\T}       {\mathcal{T}}
\newcommand{\prem}    {\textsf{Prem}}
\newcommand{\con}     {\textsf{Con}}
\newcommand{\ta}      {\textsf{T}}
\newcommand{\ax}      {\textsf{Ax}}

\newcommand{\VAL}     {\textsc{Val}}
\newcommand{\SSB}     {{\bf SSbe}}
\newcommand{\SSO}     {{\bf SSat}}
\newcommand{\SSC}     {{\bf SSco}}
\newcommand{\SSU}     {{\bf SSmu}}

\newcommand{\two}     {{\bf 2}}
\newcommand{\twi}     {{\bf \Omega}}
\newcommand{\Rra}     {\Rrightarrow}

\newcommand{\jul}{\mathnormal{^{1}\hspace{-0,12cm}/\hspace{-0,04cm}_{2}}} 

\DeclareMathAlphabet{\mathpzc}{OT1}{pzc}{m}{it}

\begin{document}

\title{\textbf{Criptografia RSA gaussiana}}
\author{\textbf{Luis Antonio Co\^{e}lho}\vspace{2cm}\\
Relat\'{o}rio para Trabalho de Conclus\~{a}o de Curso - parte I apresentado \`{a}\\ Faculdade de Tecnologia da\\ Universidade Estadual de Campinas \vspace{2cm}\\
Orientador:
\textbf{Profa. Dra. Juliana Bueno}\vspace{2cm}\\}
 \maketitle



% ------------------------------------------------------------------------
%{
%
%%%%%%%%%%%%%%%%%%%%%%%%%%%%%%%%%%%%%%%%%%%%%%%%%%%%%DEDICAT�RIA%%%%%%%%%%%%%%%%%%%%%%%%%%%%%%%%%%%%%%%%%%%%%%%%%%%%%%%%%%%%%

\thispagestyle{empty}

\hspace{1cm} \vspace{8.5cm}


\hspace{2.83cm}\textit{\Large{Dedico este trabalho a minha turma e aminha fam\'ilia, por conta do apoio durante sua produ\c{c}\~ao. }}
                   % dedicat�ria
%}
% ------------------------------------------------------------------------
%{
%
%%%%%%%%%%%%%%%%%%%%%%%%%%%%%%%%%%%%%%%%%%%%%%%%%%%%%%%INVOCA��O%%%%%%%%%%%%%%%%%%%%%%%%%%%%%%%%%%%%%%%%%%%%%%%%%%%%%%%%%%%%%

\thispagestyle{empty}

%\vspace{20cm}

\noindent{AMO-TE TANTO, meu amor... n�o cante}\\
O humano cora��o com mais verdade...\\
Amo-te como amigo e como amante\\
Numa sempre deversa realidade.\\

 \vspace{0.5 cm}

\noindent{Amo-te afim, de um calmo amor prestante,}\\
E te amo al�m, presente na saudade.\\
Amo-te, enfim, com grande liberdade\\
Dentro da eternidade e a cada instante.\\

 \vspace{0.5 cm}

\noindent{Amo-te como um bicho, simplesmente,}\\
De um amor sem mist�rio e sem virtude\\
Com um desejo maci�o e permanente.\\

 \vspace{0.5 cm}

\noindent{E de te amar assim muito e ami�de,}\\
� que um dia em teu corpo de repente\\
Hei de morrer de amar mais do que pude.\\



(Vin�cius de Moraes, \textit{Soneto do amor total})
                   % invoca��o
%}
% ------------------------------------------------------------------------
%{
%\typeout{Acknowledgements}
%
%%%%%%%%%%%%%%%%%%%%%%%%%%%%%%%%%%%%%%%%%%AGRADECIMENTOS%%%%%%%%%%%%%%%%%%%%%%%%%%%%%%%%%%%%%%%%%%%%%%%%%%%%%%%%%%%%%%%%%%%%
\pagestyle{fancy}
%\pagestyle{fancy}
\fancyhead[C]{\textsl{Agradecimentos}}
\fancyhead[R]{7}
\fancyfoot[C]{}	


\chapter*{Agradecimentos}

\vspace{1.5cm}


\noindent Agrade�o neste trabalho primeiramente a Deus que permitiu que tudo isso acontecesse, ao longo de minha vida, e n�o somente nestes anos como universit�rio, mas que em todos os momentos � o maior mestre que algu�m pode conhecer.Agrade�o a todos os professores por me proporcionar o conhecimento n�o apenas racional, mas a manifesta��o do car�ter e afetividade da educa��o no processo de forma��o profissional, por tanto que dedicaram a mim, n�o somente por terem me ensinado, mas por terem me feito aprender. A palavra mestre, nunca far� justi�a aos professores dedicados aos quais sem nominar ter�o os meus eternos agradecimentos. Agrade�o a minha m�e Raquel, hero�na que me deu apoio, incentivo nas horas dif�ceis, de des�nimo e cansa�o e a todos que direta ou indiretamente fizeram parte da minha forma��o, o meu muito obrigado.\\
                   % agradecimentos
%}
% -----------------------------------------------------------------------
{ \typeout{Abstract}

%%%%%%%%%%%%%%%%%%%%%%%%%%%%%%%%%%%%%%%RESUMO%%%%%%%%%%%%%%%%%%%%%%%%%%%%%%%%%%%%%%%%%%%%%%%%%%%%%%%%%%%%%%%%%%%%%%%%%%%%%%%%

\thispagestyle{empty}

\hspace{1cm} \vspace{2.2cm}

\noindent {\Huge {\bf Resumo}}

\vspace{1.5cm}

\noindent O presente relat�rio exp�e o resultado parcial do projeto para TCC sobre o algoritmo de criptografia RSA gaussiano.
                    % resumo
}
% ------------------------------------------------------------------------

\setcounter{page}{1}
\tableofcontents                  % cria o �ndice

% ------------------------------------------------------------------------
{ %\typeout{Introducao}
%
%%%%%%%%%%%%%%%%%%%%%%%%%%%%%%%%%%%%%%%%%%%INTRODU��O%%%%%%%%%%%%%%%%%%%%%%%%%%%%%%%%%%%%%%%%%%%%%%%%%%%%%%%%%%%%%%%%%%%%%%%

\thispagestyle{empty}

\addcontentsline{toc}{chapter}{Introdu��o}

\chapter*{Introdu��o}


Introduza o que voc� pretende fazer no decorrer do seu trabalho.
					
\pagestyle{fancy}
\fancyhead[C]{\textsl{Introdu��o}}
\fancyhead[R]{\thepage}
\fancyfoot[C]{}
\fancyhead[L]{}

\addcontentsline{toc}{chapter}{Introdu��o} % insere no sum�rio a introdu��o

\chapter*{Introdu\c{c}\~{a}o}
\label{Intro}

O sigilo sempre foi uma arma explorada pelos seres humanos para vencer certas batalhas, e at\'e mesmo para a cotidiana miss\~{a}o de se comunicar. Foi a partir dessa necessidade que se criou a \textit{criptografia}, nome dado ao conjunto de t\'ecnicas usadas para se  comunicar em c\'odigos. Seu objetivo \'{e} garantir que apenas os envolvidos na comunica\c{c}\~ao possam compreender a mensagem codificada (ou criptogtafada), garantindo que terceiros n\~ao saibam o que foi conversado.

Para compreender como funciona o processo de codifica\c{c}\~ao e decodifica\c{c}\~ao faz-se necess\'ario o uso de uma s�rie de termos t\'ecnicos, e para fins pedag�gicos iremos introduzir tais conceitos apresentando um dos primeiros algoritmos criptogr\'aficos que se tem conhecimento, a criptografia de C\'esar. Para mais detalhes sobre o tema, veja Criptografia, por Coutinho\cite{coutinho}.

A chamada \textit{criptografia de C\'esar}, criada pelo imperador romano C\'esar Augusto, consistia em substituir cada letra da mensagem por outra que estivesse a tr\^es posi\c{c}\~oes a frente, como, por exemplo, a letra \textbf{A} que neste algoritmo \'e substitu\'ida pela letra \textbf{D}.  

Uma forma muito natural de se generalizar o algoritmo de C\'esar \'e fazer a troca de cada letra da mensagem por outra que venha em uma posi\c{c}\~ao qualquer fixada. A chamada \textit{criptografia de substitui\c{c}\~ao monoalfab\'etica} consiste em substituir cada letra por outra que ocupe $n$ posi\c{c}\~oes � sua frente, sendo que o n\'umero $n$ \'e conhecido apenas pelo emissor e pelo receptor da mensagem. O n\'umero $n$ \'e a \textit{chave criptogr\'afica}. Para decifrar a mensagem, precisamos substituir as letras que formam a mensagem criptografada pelas letras que est\~ao $n$ posi\c{c}\~oes antes.

O algoritmo monoalfab\'etico tem a caracter\'istica indesejada de ser de f\'acil decodifica\c{c}\~ao, pois possui apenas {26} chaves poss\'iveis, e isso faz com que no m\'aximo em {26} tentativas o c\'odigo seja decifrado. Com o intuito de dificultar a quebra do c\'odigo monoalfab\'etico foram propostas as \textit{cifras de substitui\c{c}\~ao polialfab\'eticas} em que a chave criptogr\'afica passa a ser uma \textit{palavra} ao inv\'es de um n\'umero. A ideia \'e usar as posi\c{c}\~oes ocupadas pelas letras da chave para determinar o n\'umero de posi\c{c}\~oes que devemos avan\c{c}ar para obter a posi\c{c}\~ao da letra encriptada. Vejamos, por meio de um exemplo, como funciona esse sistema criptogr\'afico.

Sejam ``SENHA'' a nossa chave criptogr\'afica e ``ABOBORA'' a mensagem a ser encriptada. Abaixo colocamos as letras do alfabeto com suas respectivas posi\c{c}\~oes. Observe que repetimos a primeira linha de letras para facilitar a localiza\c{c}\~ao da posi\c{c}\~ao da letra encriptada e usamos a barra para indicar que estamos no segundo ciclo. 

\[
\begin{array}{ccccccccccccc}
    1      & 2 & 3 & 4 & 5          & 6 & 7 & 8          & 9 & 10 &  11 & 12 & 13 \\  
\textbf{A} & B & C & D & \textbf{E} & F & G & \textbf{H} & I & J  &  K  & L  & M  \\ 
  &   &   &   &   &   &   &   &   &    &     &    &    \\ 
    14      & 15 & 16 & 17 & 18 & 19          & 20 & 21 & 22 & 23 & 24 & 25 & 26 \\
\textbf{N}  & O  & P  & Q  & R  & \textbf{S}  & T  & U  & V  & X  & Y  & W  & Z \\
&   &   &   &   &   &   &   &   &    &     &    &    \\ 
    27      & 28 & 29 & 30 & 31          & 32 & 33 & 34          & 35 & 36 &  37 & 38 & 39 \\  
\overline{A} & \overline{B} & \overline{C} & \overline{D} & \overline{E} & \overline{F} & \overline{G} & \overline{H} & \overline{I} & \overline{J}  &  \overline{K}  & \overline{L}  & \overline{M}  \\
\end{array}
\]

Vejamos como encriptar a palavra ``ABOBORA''. Iniciamos o processo escrevendo a mensagem. Ao lado de cada letra da mensagem aparece entre par\^enteses o n\'umero que indica a sua posi\c{c}\~ao. Abaixo da mensagem escrevemos as letras da chave criptogr\'afica, repetindo-as de forma c\'iclica quando necess\'ario. Analogamente, ao lado de cada letra da chave aparece entre par\^enteses o n\'umero da posi\c{c}\~ao ocupada de cada letra, e o sinal de soma indica que devemos avan\c{c}ar aquele n�mero de posi��es. Ao final do processo aparecem as letras encriptadas. Entre par�nteses est� a posi\c{c}\~ao resultante da combina\c{c}\~ao das posi\c{c}\~oes da mensagem e da chave.   

\footnotesize{
\[
\begin{array}{lllllll||l}
     A (1)  &      B (2)  &      O (15) &      B (2)  &      O (15) &     R (18)  &    A (1)	 & \textrm{Mensagem}  \\
\downarrow  & \downarrow  & \downarrow  & \downarrow  & \downarrow  & \downarrow  & \downarrow &\\ 
    S (+19) &     E (+5)  &     N (+14) &     H  (+8) &     A (+1)  &    S (+19)  &   E (+5)   &\textrm{Chave}  \\
\downarrow  & \downarrow  & \downarrow  & \downarrow  & \downarrow  & \downarrow  & \downarrow & \\
		 T (20) &      G (7)  &      C (29) &      J (10) &     P (16)  &    K (37)   &    F (6)   & \textrm{Mensagem encriptada}  \\
\end{array}
\]
}
  
Observe que a encripta\c{c}\~ao polialfab\'etica \'e mais dif\'icil de ser quebrada que a monoalfab\'etica uma vez que letras iguais n\~ao t\^em, necessariamente, a mesma encripta\c{c}\~ao. Neste tipo de criptografia o emissor precisa passar a chave para o receptor da mensagem de forma segura para que o receptor possa decifrar a mensagem, isto \'e, a chave usada para encriptar a mensagem \'e a mesma que deve ser usada para decifrar a mensagem. Veremos que esse \'e justamente o ponto fraco neste tipo de encripta\c{c}\~ao pois usa a chamada \textit{chave sim\'etrica}, ou seja, a chave usada pelo emissor para codificar a mensagem \'e a mesma usada pelo receptor para decodificar a mensagem. Nesse processo, a chave deve ser mantida em segredo e bem guardada para garantir que o c\'odigo n\~ao seja quebrado, e isso requer algum tipo de contato f\'isico entre emissor e receptor da mensagem.

Durante a  Primeira Guerra Mundial o contato f\'isico para a troca de chaves era complicado, e isso estimulou a cria\c{c}\~ao de m\'aquinas autom\'aticas de criptografia. O \textit{Enigma} foi uma dessas m\'aquinas e era utilizada pelos alem\~aes tanto para criptografar como para descriptografar c\'odigos de guerra. Semelhante a uma m\'aquina de escrever, os primeiros modelos foram patenteados por Arthur Scherbius em 1918. Essas m\'aquinas ganharam popularidade entre as for\c{c}as militares alem\~as devido � facilidade de uso e sua suposta indecifrabilidade do c\'odigo. 

O matem\'atico Alan Turing foi o respons\'avel por quebrar o c\'odigo dos alem\~aes durante a Segunda Guerra Mundial. A descoberta de Turing mostrou a fragilidade da criptografia baseada em chave sim\'etrica e colocou novos desafios \`a criptografia. O grande problema passou a ser a quest\~ao dos protocolos, isto \'e, como transmitir a chave para o receptor de forma segura sem que seja necess\'ario o contato f\'isico entre as partes? 

Em 1949, com a publica\c{c}\~ao do artigo \textit{Communication Theory of Secrecy Systems} \cite{shannon} de Shannon, temos a inaugura\c{c}\~ao da criptografia moderna. Neste artigo ele escreve matematicamente que cifras teoricamente inquebr\'aveis s\~ao semelhantes �s cifras polialfab\'eticas. Com isso ele transformou a criptografia que at\'e ent\~ao era uma arte, em uma ci\^encia.

Em 1976 Diffie e Hellman publicaram \textit{New Directions in Cryptography} \cite{newdirections}. Neste artigo h\'a a introdu\c{c}\~ao ao conceito de \textit{chave assim\'etrica}, onde h\'a chaves diferentes entre o emissor da mensagem e seu receptor. Com a assimetria de chaves n\~ao era mais necess\'ario um contato t\~ao pr\'oximo entre emissor e receptor. Neste mesmo artigo \'e apresentado o primeiro algoritmo de criptografia de chave assim\'etrica ou como \'e mais conhecido nos dias atuais \textit{Algoritmo de Criptografia de Chave P\'ublica}, o protocolo de Diffie-Hellman.

Um dos algoritmos mais famosos da criptografia de chave p\'ublica \'e o \textit{RSA} \cite{rivest}, algoritmo desenvolvido por Rivest, Shamir e Adleman. Este algoritmo se tornou popular por estar presente em muitas aplica\c{c}\~oes de alta seguran\c{c}a, como bancos, sistemas militares e servidores de internet.

Para que se possa compreender por completo o algoritmo faz-se necess�rio possuir alguns conhecimentos em teoria de n\'umeros como fatora\c{c}\~ao e aritm\'etica modular. Estes conhecimentos ser\~ao apresentados mais adiante neste trabalho.

No algoritmo RSA existe uma chave p\'ublica $n$, que \'e a multiplica\c{c}\~ao dos primos $p$ e $q$. O emissor E codifica a mensagem usando um n\'umero primo $p$. Em seguida E envia publicamente a mensagem codificada junto com a chave $n$ para o receptor R. R possui o n\'umero $q$, que juntamente ao n\'umero $n$ servem para decodificar a mensagem. 

Embora a quebra do RSA seja aparentemente simples, bastando fatorar $n$ para descobrir seus fatores, o grande problema \'e na realidade computacional, pois usa-se como $p$ e $q$ n\'umeros primos muito altos, pr\'oximos a $2^{512}$. Com um n\'umero t\~ao alto um computador comum levaria bem mais que uma vida humana para decifrar a mensagem.

Com base nestes conhecimentos sobre criptografia, temos que o objetivo deste trabalho \'e analisar a viabilidade de uma criptografia inspirada pelo algoritmo RSA cl\'assico, a qual substitui os n\'umeros primos pelo conjunto denominado de \textit{primos de Gauss} \cite{intGauss}, resultando, assim, no que chamamos por \textit{criptografia RSA gaussiana}. Para que tal algoritmo seja vi\'avel \'e necess\'ario adaptar uma s\'erie de resultados relativas aos n\'umeros primos aos n\'umero primos de Gauss. Dessa forma, nossa tarefa ser\'a adaptar tanto quanto o poss\'ivel os primos de Gauss \`as demosnta\c{c}\~oes desses teoremas.

Como se trata de uma proposta inovadora, deixamos para trabalhos futuros uma an\'alise comparativa entre as criptografias RSA cl\'assica e a RSA gaussiana.        
\chapter {Aritm\'{e}tica Modular}
\label{Mod}

\section{Ciclos e Restos}	
\subparagraph{
Para podermos compreender a aritm\'etica modular, precisamos come\c{c}ar entendendo o conceito de ciclicidade. Um bom exemplo deste conceito \'e o nascer do sol, que \'e um evento que ocorre sempre ap\'os um ciclo de {24} horas, assim como o dia de seu anivers\'ario ocorre uma vez a cada ciclo de um ano.
}

\section{N\'{u}meros Primos Naturais}	

\section{Per\'{i}odo e Fatora\c{c}\~{a}o}

\section{Inverso Multiplicativo}

      
%%%%%%%%%%%%%CONSIDERA��ES FINAIS%%%%%%%%%%%%%%%%%%%%%%%%%%%%%%%%%%%%%%%%%%%%%%%%%%%%%%

\pagestyle{fancy}
\fancyhead[C]{\textsl{Considera��es Finais}}
\fancyhead[R]{\thepage}
\fancyfoot[C]{}

\addcontentsline{toc}{chapter}{Considera��es Finais}

\chapter*{Considera��es Finais}

Nesta sess\~ao faremos as \'ultimas considera\c{c}\~oes sobre a viabilidade do algoritmo RSA Gaussiano. Al\'em disso vamos ver o que outros pesquisadores j\'a est\~ao concluindo em suas pesquisas.

A primeira coisa que devemos prestar aten\c{c}\~ao \'e que no decorrer desta monografia fomos capazes de descrever o funcionamento do algoritmo de criptografia RSA.

Al\'em disso demos os primeiros passos rumo a uma criptografia RSA Gaussiana, e n\~ao encontramos nada que impedisse a sua realiza\c{c}\~ao, mas como foi visto no cap\'itulo \ref{IG}, ainda precisamos da comprava\c{c}\~ao de alguns teoremas matem\'aticos importantes para a realiza\c{c}\~ao deste algoritmo. 

O material publicado por \cite{koval} e \cite{elkassar} nos leva a crer na viabiliadade do algoritmo. O que ocorre \'e que ambos possuem vis\~oes bem diferentes com rela\c{c}\~ao ao RSA Gaussiano. \cite{koval} n\~ao defende o algoritmo, pois acredita que ele n\~ao acrescenta seguran\c{c}a ao algoritmo RSA, al\'em de deix\'a-lo menos pr\'atico. Abaixo citamos o trecho onde isso \'e afirmado:

\begin{quote}
``The extension of RSA algorithm into the field of Gaussian integers [...] is viable only if real primes p congruent to 3 modulo 4 are used [...]. The extended algorithm could add security only if breaking the original RSA is not as hard as factoring. Even in this case, it is not clear whether the extended algorithm would increase security. The Gaussian integer RSA is slightly less efficient than the original, therefore the original real integer RSA is more practical.''
\end{quote}

Enquanto isso, \cite{elkassar} defende o algoritmo Gaussiano por aumentar a seguran\c{c}a comparado ao cl\'assico, como pode ser lido abaixo:

\begin{quote}

``Arithmetic needed for the RSA cryptosystem in the domains of Gaussian integers and polynomials over finite fields were modified and computational procedures were described. There are advantages for the new schemes over the classical one. First, generating the odd prime numbers in both the classical and the modified methods requires the same amount of efforts. Second, the modified method provides an extension to the range of chosen messages and the trials will be more complicated. ''

\end{quote}

Baseado nos textos de ambos podemos concluir que al\'em da realiza\c{c}\~ao de tal algoritmo, outro problema a ser investigado em um trabalho futuro consiste na an\'alise de seguran\c{c}a e complexidade do algoritmo, visto que ainda n\~ao possu\'imos uma conclus\~ao definitiva sobre isso.
        %Considera��es Finais
%
%%%%%%%%%%%%%%%%%%%%%%%%%%%AP�NDICE 1: Ipcional%%%%%%%%%%%%%%%%%%%%%%%%%%%%%%%%%%%%%%%%%%%%

\addcontentsline{toc}{chapter}{Ap�ndice 1}

\chapter*{Ap�ndice 1\\ (Opcional)}

Exemplos dos mais interessantes  manuais de latex na rede s�o os
seguintes:


P�gina sobre criptografia do IME- USP Manuais de LaTeX em
portugu�s e ingl�s, inclusive com conversoires entre LaTeX e
outros formatos:
\\
http://www.ime.eb.br/~pinho/pessoal/latex/

Manual b�sico do IFGW- UNICAMP:
\\
http://www.ifi.unicamp.br/encontro/latex-exemplo.html
         %Ap�ndice 1: Opcional

\addcontentsline{toc}{chapter}{Bibliografia} %para aparecer a  bibliografia no �ndice
%%%%%%%%%%%%%%%%%%%%%%%%%%%%%%%%%%% Comandos para gerar a bibliografia em Portugu�s %%%%%%%%%%%%%%%%%%%%%%%%

\selectbiblanguage{brazil}
%\bibliographystyle{babplain}
\bibliographystyle{babalpha}
\bibliography{Bibliografia}

%%%%%%%%%%%%%%%%%%% Caso n�o gere a bibliografia por falta de pacotes rode com os comandos abaixo e retire o pacote \usepackage{babelbib} do in�cio do documento %%%%%%%%%%%

\bibliographystyle{alpha}
\nocite{*}
%\bibliography{Bibliografia}

%%%%%%%%%%%%%%%%%%%%%%%%%%%%%%%%%%%%%%%%%%%%%%%%%%%%%%%%%%%%%%%%%%%%%%%%%%%%%%%%%%%%%%%%%%%%%%%%%%%%%%%%%%%%%
\end{document}
