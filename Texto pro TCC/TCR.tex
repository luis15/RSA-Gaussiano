\chapter {T\'ecnicas para facilitar opera\c{c}\~oes}
\label{TCR}

\section{Teorema chin\^es do resto}

\subparagraph{
Para sermos iniciados nesta t\'ecnica, vamos analisar o seguinte problema: Qual o menor inteiro que possui resto $1$ na divis\~ao por $3$ e resto $2$ na divis\~ao por $5$. Podemos vir a tranformar esse problema nas seguintes equa\c{c}\~oes:
}
\[	
	\begin{array}{c}
		\textit{$n = 3q_1 + 1$ e $n = 5q_2 + 2$}
	\end{array}
\]
\paragraph{
Essas equa\c{c}\~oes tamb\'em podem ser denotadas em forma modular como:
}
\[	
	\begin{array}{c}
		\textit{$n \equiv 1 (mod 3)$ e $n \equiv 2 (mod 5)$}
	\end{array}
\]
\paragraph{
Essa sa\'ida modular nos deixou com apenas uma vari\'avel, mas ainda n\~ao resolveu ao nosso problema. Para fazermos isso vamos substituir $n$ por $5q_2 + 2$, montando a seguine equa\c{c}\~ao modular:
}
\[	
	\begin{array}{c}
		\textit{$5q_2 + 2 \equiv 1 (mod 3)$}
	\end{array}
\]
\paragraph{
Como $5 \equiv 2(mod 3)$, substitu\'imos:
}
\[	
	\begin{array}{c}
		\textit{$ 2q_2 + 2 \equiv 1 (mod 3)$}
	\end{array}
\]
\paragraph{
Feito isso, passamos $2$ para o outro lado da equa\c{c}\~ao
}
\[	
	\begin{array}{c}
		\textit{$ 2q_2  \equiv -1 (mod 3)$}
	\end{array}
\]
\paragraph{
Como $-1 \equiv 2 (mod 3)$, n\'os substit\'imos novamente, e depois dividimos a equa\c{c}\~ao por $2$, e obtemos
}
\[	
	\begin{array}{c}
		\textit{$ q_2  \equiv 1 (mod 3)$}
	\end{array}
\]
\paragraph{
Com isso, conclu\'imos que
}
\[	
	\begin{array}{c}
		\textit{$ q_2  \equiv q_3 + 1 (mod 3)$}
	\end{array}
\]
