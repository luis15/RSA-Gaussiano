\chapter {Teorema chin\^es do resto}
\label{TCR}

\section{Introdu\c{c}\~ao a t\'ecnica}

\subparagraph{
Para sermos iniciados nesta t\'ecnica, vamos analisar o seguinte problema: Qual o menor inteiro que possui resto $1$ na divis\~ao por $3$ e resto $2$ na divis\~ao por $5$. Podemos vir a tranformar esse problema nas seguintes equa\c{c}\~oes:
}
\[	
	\begin{array}{c}
		\textit{$n = 3q_1 + 1$ e $n = 5q_2 + 2$}
	\end{array}
\]
\paragraph{
Essas equa\c{c}\~oes tamb\'em podem ser denotadas em forma modular como:
}
\[	
	\begin{array}{c}
		\textit{$n \equiv 1 (mod 3)$ e $n \equiv 2 (mod 5)$}
	\end{array}
\]
\paragraph{
Essa sa\'ida modular nos deixou com apenas uma vari\'avel, mas ainda n\~ao resolveu ao nosso problema. Para fazermos isso vamos substituir $n$ por $5q_2 + 2$, montando a seguine equa\c{c}\~ao modular:
}
\[	
	\begin{array}{c}
		\textit{$5q_2 + 2 \equiv 1 (mod 3)$}
	\end{array}
\]
\paragraph{
Como $5 \equiv 2(mod 3)$, substitu\'imos:
}
\[	
	\begin{array}{c}
		\textit{$ 2q_2 + 2 \equiv 1 (mod 3)$}
	\end{array}
\]
\paragraph{
Feito isso, passamos $2$ para o outro lado da equa\c{c}\~ao
}
\[	
	\begin{array}{c}
		\textit{$ 2q_2  \equiv -1 (mod 3)$}
	\end{array}
\]
\paragraph{
Como $-1 \equiv 2 (mod 3)$, n\'os substit\'imos novamente, e depois dividimos a equa\c{c}\~ao por $2$, e obtemos
}
\[	
	\begin{array}{c}
		\textit{$ q_2  \equiv 1 (mod 3)$}
	\end{array}
\]
\paragraph{
Com isso, conclu\'imos que
}
\[	
	\begin{array}{c}
		\textit{$ q_2  \equiv q_3 + 1 (mod 3)$}
	\end{array}
\]
\paragraph{
Sei que parece que mais uma equa\c{c}\~ao s\'o serve para tornar a resolu\c{c}\~ao mais complexa, mas vamos a reorganizar como
}
\[	
	\begin{array}{c}
		\textit{$ q_2 = 3q_3 + 1 $}
	\end{array}
\]
\paragraph{
Agora substitu\'imos
}
\[	
	\begin{array}{c}
		\textit{$n = 5(3q_3 + 1) + 2 = 15q_3 +7$}
	\end{array}
\]
\paragraph{
Feito isso, vamos por o $3$ em evid\^encia em todos os lugares, obtendo:
}
\[	
	\begin{array}{c}
		\textit{$n = 3(5q_3) +3(2) +1 = 3(5q_3 +2)+1$}
	\end{array}
\]
\paragraph{
Este procedimento foi feito apenas para provar que a equa\c{c}\~ao deixa resto 1 se dividida por 3, de forma an\'aloga, abaixo \'e mostrado como ela deixa resto $2$ quando dividida por $5$.
}
\[	
	\begin{array}{c}
		\textit{$n = 5(3q_3) +5(1) +2 = 5(3q_3 +1)+2$}
	\end{array}
\]
\subparagraph{
Ap\'os tudo isso feito ainda n\~ao possu\'imos a solu\c{c}\~ao final, mas j\'a sabemos que \'e um n\'umero da forma $15q_3 + 7$, substituindo $q_3$ or $0$, iremos obter $7$, que \'e o resultado procurado.
}

\section{O teorema}

\subparagraph{
O teorema chin\^es do resto \'e um procedimento tomado ara resolver sistema de congru\^encias, como o descrito acima. Ele foi descrito pela primeira vez pelo Manual de aritm\'etica do mestre Sun, por volta do s\'eculo III d.C. 
}
\subparagraph{
Para ver a defini\c{c}\~ao formal desse teorema, vamos considerar o sistema
}
\[	
	\begin{array}{c}
		\textit{$x \equiv a (mod n)$}\\
		\textit{$x \equiv b (mod m)$}\\
	\end{array}
\]
\paragraph{
nele, $n$ e $m$ s\~ao inteiros diferentes entre si. Tomemos $x_0$ como um n\'umero cappaz de satisfazer ambas as congru\^encia de forma simult\^anea e teremos:
}
\[	
	\begin{array}{c}
		\textit{$x_0 \equiv a (mod m)$}\\
		\textit{$x_0 \equiv b (mod n)$}\\
	\end{array}
\]
\paragraph{
Para podermos juntar ambas as equa\c{c}\~oes converteremos uma em equa\c{c}\~ao, nesse caso teremos 
}
\[	
	\begin{array}{c}
		\textit{$x_0 = a + m\cdot k$, com $k$ sendo um inteiro qualquer}\\
	\end{array}
\]
\paragraph{
Feito isso, chegaremos em
}
\[	
	\begin{array}{c}
		\textit{$a + m\cdot k \equiv b (mod n)$}\\
	\end{array}
\]
\paragraph{
que pode ser substitu\'ida por
}
\[	
	\begin{array}{c}
		\textit{$ m\cdot k \equiv (b-a) (mod n)$}\\
	\end{array}
\]
\subparagraph{
Agora vamos supor que $m$ e $n$ s\~ao primos entre si. Pelo teorema apresentado no cap\'ituo sobre inversos multiplicativos n\'os j\'a sabemos que eles possuem inverso multiplicativo um para o outro. Tomemos $m'$ como o inverso de $m$ no m\'odulo $n$. Multipplicando toda a congru\^encia por $m'$ obtemos
}
\[	
	\begin{array}{c}
		\textit{$ k \equiv m'\cdot(b-a) (mod n)$}\\
	\end{array}
\]