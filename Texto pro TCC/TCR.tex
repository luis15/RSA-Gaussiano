\chapter {T\'ecnicas para facilitar opera\c{c}\~oes}
\label{TCR}

\section{Teorema chin\^es do resto}

\subparagraph{
Para sermos iniciados nesta t\'ecnica, vamos analisar o seguinte problema: Qual o menor inteiro que possui resto $1$ na divis\~ao por $3$ e resto $3$ na divis\~ao por $5$. Podemos vir a tranformar esse problema nas seguintes equa\c{c}\~oes:
}
\[	
	\begin{array}{c}
		$n = 3q_1 + 1$ e $n = 5q_2 + 3$
	\end{array}
\]
\paragraph{
Essas equa\c{c}\~oes tamb\'em podem ser denotadas em forma modular como:
}
\[	
	\begin{array}{c}
		$n \equiv 1 (mod 3)$ e $n \equiv 3 (mod 5)$
	\end{array}
\]