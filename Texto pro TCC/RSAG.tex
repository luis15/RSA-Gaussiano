\chapter {RSA Gaussiano e Conclus�es}
\label{RSAG}

\subparagraph{
Chegamos ao \'ultimo cap\'itlo desta obra, aqui ser\'a debatido sobre tudo o que foi alcan�ado at\'e o momento no que diz respeito ao RSA Gaussiano, veremos como sua forma \'e planejada e o que ter\'a de ser feito no futuro para que este algoritmo se torne uma op��o entre os algoritmos criptogr\'aficos.
}

\section{O RSA Gaussiano e o que falta para ele ser utiliz�vel}
\subparagraph{
Ao longo desse se\c{c}\~ao lhe ser\'a apresentado como o RSA Gaussiano dever\'a vir a funcionar. Para iniciarmos, teremos que, assim como na criptografia RSA fazer uma pr�-encripta\c{c}\~ao vindo a transformar todas as letras em n\'umero inteiros, da mesma forma que j\'a ocorre.
}
\subparagraph{
A segunda parte do processo, que \'e a encripta\c{c}\~ao depende de algumas garantias as quais ainda n\~ao possu\'imos. Embora saibamos que a propiedade da fatora\c{c}\~ao \'unica \'e v\'alida para o conjunto $Z[i]$. Um dos meios para que possamos seguir consiste em definir a opera\c{c}\~ao de congru\^encia modular para o conjunto gaussiano, tamb\'em se fazem necess\'arias propiedades an\'alogas ao Teorema de Fermat e ao Teorema chin\^es do resto para que possamos seguir a mesma linha de encripta\c{c}\~ao do algoritmo RSA.
}
\subparagraph{
Para o RSA Gaussiano \'e planejada a encripta\c{c}\~ao com a f\'ormula:
}
\[
\begin{array}{c}
\textit{$C'(a) \equiv a^3 (mod n')$}
\end{array}
\]
\paragraph{
onde $a$ seria o n\'umero a ser encriptado, $n'$ seria a chave p\'ublica gaussiana, derivada de $p'$ e $q'$, que s\~ao n\'umeros primos gaussianos. Os n\'umeros $p'$ e $q'$ s\~ao as chaves privadas de encripta\c{c}\~ao.
}
\subparagraph{
Para a desencripta\c{c}\~ao, embora ainda n\~ao tenhamos uma prova de seu funcionamento pelos motivos j\'a citados acima, ela \'e planejada em um primeiro momento pelo n\'umero $d'$, que pode manter sua f\'ormula similar a do RSA ou n\~ao. Caso ele n\~ao precise de mudan\c{c}as por conta do conjunto gaussiano, dever\'a ser da seguinte f\'ormula:
}
\[
\begin{array}{c}
\textit{$3d' \equiv 1(mod((p'-1)(q'-1)))$}
\end{array}
\]
\subparagraph{
Supondo que a f\'ormula de $d'$ acima seja v\'alida, teremos que para podermos concluir desencripta\c{c}\~ao a f\'ormula an\'aloga a original, que seria:}
\[
\begin{array}{c}
\textit{$D'(b) \equiv b^d' (mod n')$}
\end{array}
\]
\subparagraph{
Nesta f\'ormula temos que $b$ \'e um n\'umero encriptado pelo RSA Gaussiano, $d'$ a constante cuja f\'ormula foi mostrada anteriormente, $n'$ a chave p\'ublica e $D'(b)$ o resultado da desencripta\c{c}\~ao.
}
\subparagraph{
Com isso conclu\'imos este projeto, aguardando que muito em breve tudo o que ficou como trabalho futuro aqui venha a ser realizado, muito obrigado por sua aten\c{c}\~ao a este projeto e esperamos que ele venha a ser uma fonte de inspira\c{c}\~ao para seus projetos futuros.
}