\chapter {Potencia\c{c}\~ao}
\label{Pot}

\section{Restos na potencia\c{c}\~ao}	
\subparagraph{
Ao longo deste cap\'itulo vamos estudar como tornar as opera\c{c}\~oes de potencia\c{c}\~ao e a obten\c{c}ao de seus restos calcal\'aveis de forma simpls e r\'apida. Para isso vamos dispor de algumas artimanhas matem\'aticas, al\'em do famoso \textit{Teorema de Fermat}.
}
\subparagraph{
Vamos come�ar tentando uma coisa que aparentemente \'e complexa, mas se converter\'a em uma opera\c{c}\~ao bem simples: Calcular o resto da divis\~ao de $10^{135}$ por $7$. Podemos fazer da forma tradicional, mas dividir um n\'umero t\~ao alto n\~ao seria nada pr\'atico.
}
\subparagraph{
O que faremos \'e tomar uma propiedade da multiplica\c{c}\~ao e da potencia\c{c}\~ao emprestadas, a do elemento neutro, nesse caso o $1$. O que faremos \'e calcular em qual pot\^encia $10$ \'e congruente a $1$ no m\'odulo $7$. Logo teremos a tabela:
}
\[
\begin{array}{c}
  \textit{$10^1 \equiv 3(mod7)$} \\  
	\textit{$10^2 \equiv 2(mod7)$}\\
	\textit{$10^3 \equiv 6(mod7)$}\\ 
	\textit{$10^4 \equiv 4(mod7)$}\\ 
	\textit{$10^5 \equiv 5(mod7)$}\\ 
	\textit{$10^6 \equiv 1(mod7)$}\\ 
\end{array}
\]
\subparagraph{
Como sabemos agora que $10^6$ \'e o n\'umero que quer\'iamos, vamos decompor o $135$ em raz\~ao de $6$ e teremos que $135 = (6 \cdot 22)+3$, essa express\~ao nos levar\'a a seguinte congru\^encia:
}
\[
\begin{array}{c}
  \textit{$10^{135} \equiv (10^6)^{22} \cdot 10^3 \equiv (1)^{22} \cdot 10^3 \equiv 10^3 \equiv 6 (mod7)$} \\  
\end{array}
\]
\subparagraph{
Agora n\'os vamos deixar essa opera\c{c}\~ao um pouco mais complexa, ao passo que vamos calcular o resto por $31$ de $2^{124512}$. Vamos pelo mesmos caminho que anteriormente, buscando a pot\^encia de $2$ que \'e congruente a $1$ no m\'odulo $31$. Obtemos:
}
\[
\begin{array}{c}
  \textit{$2^1 \equiv 2(mod31)$} \\  
	\textit{$2^2 \equiv 4(mod31)$}\\
	\textit{$2^3 \equiv 8(mod31)$}\\ 
	\textit{$2^4 \equiv 16(mod31)$}\\ 
	\textit{$2^5 \equiv 1(mod31)$}\\ 
\end{array}
\]
\subparagraph{
Vamos dividir $124512$ por $5$ e obteremos $4016$ com resto $2$, obtendo assim que $2^{124512} \equiv 2^2 \equiv 4 (mod31)$.
}
\subparagraph{
Tornando um pouco mais dif\'icil, podemos calcular o resto de $2^{13}^{98765}$, \'e descobrir o resto de ${13}^{98765}$ por $5$, podemos dizer que ${13}^{98765} \equiv {3}^{98765}(mod 5)$ como se sabe que $3^4 = 81 \equiv 1(mod 5)$, podemos usar isso em nosso favor, pois teremos ${3}^{98765}\equiv {3}^{4\cdot24691 + 1}\equiv 3 (mod 5)$, logo o resultado de ${13}^{98765}$ \'e um n\'umero da forma $5q'+3$.
}
\subparagraph{
Como isso, n\'os podemos dizer que $2^{13}^{98765} \equiv 2^{5q'+3} \equiv 2^{5q'}\cdot{2^3}\equiv {1}^{q'}\cdot{2^3} \equiv 8 (mod 31)$.
}
\section{O teorema de Fermat}
\subparagraph{
	\textit{Teorema de Fermat} - Se $p$ \'e um n\'umero primo e $a$ \'e um inteiro n\~ao divis\'ivel por $p$, ent\~ao:
}
\[
\begin{array}{c}
  \textit{$a^{p-1}\equiv 1(mod p)$} \\  
\end{array}
\]
\subparagraph{
Embora esse seja denominado como o pequeno teorema de Fermat, ele possui suma import\^ancia para o algoritmo RSA cl\'assico. Vamos apresentar abaixo uma de suas demonstra\c{c}\~oes.
}
\subparagraph{
Sabemos que os poss\'iveis res\'iduos no m\'odulo $p$ s\~ao todos os inteiros entre $1$ e $p-1$. Vamos multiplic\'a-los por $a$, obtendo assim: 
}
\[
\begin{array}{c}
  \textit{$a \cdot 1,a \cdot 2,a \cdot 3,...,a \cdot (p-1)$} \\  
\end{array}
\]
\subparagraph{
Vamos levar em conta que $r_1 \equiv a\cdot 1(mod p)$,$r_2 = a\cdot 2(mod p)$, e assim por diante at\'e $r_{p-1} = a\cdot (p-1)(mod p)$. Tomemos par $r_k$ e $r_l$ um par de inteiros $k$ e $l$ que est\'a entre $1$ e $p-1$. Com isso teremos:
}
\[
\begin{array}{c}
  \textit{$a \cdot k \equiv k \equiv l \equiv a\cdot l (mod p)$} \\  
\end{array}
\]
\paragraph{
que equivale \`a:
}
\[
\begin{array}{c}
  \textit{$a \cdot k \equiv a\cdot l (mod p)$} \\  
\end{array}
\]

