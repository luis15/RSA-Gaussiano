
%%%%%%%%%%%%%%%%%%%%%%%%%%%%%%%%%%%%%%%RESUMO%%%%%%%%%%%%%%%%%%%%%%%%%%%%%%%%%%%%%%%%%%%%%%%%%%%%%%%%%%%%%%%%%%%%%%%%%%%%%%%%

\thispagestyle{empty}

\hspace{1cm} \vspace{2.2cm}

\noindent {\Huge {\bf Pref\'acio}}

\vspace{1.5cm}

\noindent O presente artigo exp�e o resultado da pesquisa para TCC em raz\~ao da viabilidade algoritmo de criptografia RSA gaussiano. Para isso veremos ao longo dos cap\'itulos desta obra o algoritmo de criptografia RSA e a teoria de n\'umeros envolvida por ele, passando pelo Teorema de Fermat e pelo Teorema Chin\^es do Resto. Veremos tamb\'em o que j\'a conhecido no campo dos n\'umeros inteiros e primos gaussianos e como isso ir\'a influenciar em nosso resultado final.