\chapter {O essencial em teoria dos n\'umeros}
\label{Num}

\section{Ciclos e Restos}	
\subparagraph{
Para podermos compreender a aritm\'etica modular, precisamos come\c{c}ar entendendo o conceito de ciclicidade, que s\~ao os fatos que ocorrem sempre ap\'os um determinado per\'iodo constante. Um bom exemplo deste conceito \'e o nascer do sol, que \'e um evento que ocorre sempre ap\'os um ciclo de {24} horas, assim como o dia de seu anivers\'ario ocorre uma vez a cada ciclo de um ano.
}
\subparagraph{
O mesmo tipo de evento \'e observado com o resto dos n\'umeros inteiros. Tomemos por exemplo os restos de divis\~ao pelo n\'umero inteiro {4}:
}

\[
\begin{array}{ccccccccccccc}
  {Inteiro} & 1 & 2 & 3 & 4 & 5 & 6 & 7 & 8 & 9 & 10 &  11 & 12 \\  
	{Resto} & 1 & 2 & 3 & 0 & 1 & 2 & 3 & 0 & 1 & 2  &  3 & 0 \\ 
\end{array}
\]

\subparagraph{
\'E vis\'ivel que ap\'os {4} n\'umeros o resto tende a se repetir. O mesmo feito ocorre a qualquer n\'umero inteiro $n$, onde o ciclo se repetir\'a sempre a cada $n$ itera\c{c}\~oes. Os n\'umeros que apresentam o resto {0} s\~ao conhecidos como m\'ultiplos de $n$.
}

\section{N\'{u}meros Primos e Fatora\c{c}\~{a}o}

\subparagraph{
Existe um tipo especial de n\'umero que s\'o \'e m\'ultiplo, ou seja, possui resto {0}, em duas condi\c{c}\~oes, quando $n$ \'e igual a {1} ou quando ele \'e igual a $n$. A esse conjunto de n\'umeros atribui-se o nome de \textit{n\'umeros primos}.
}
\subparagraph{
Todo o n\'umero que n\~ao \'e primo \'e chamado de \textit{N\'umero Composto}, sendo que este n\'umero composto pode ser escrito em uma combina\c{c}\~ao \'unica de fatores primos. O processo de se descobrir estes faotres \'e chamado de \textit{fatora\c{c}\~ao}.
}

\section{Inverso Multiplicativo}
