\chapter {Primos e Fatora\c{c}\~ao}
\label{Num}

\section{Ciclos e Restos}	
\subparagraph{
Para podermos compreender a aritm\'etica modular, precisamos come\c{c}ar entendendo o conceito de ciclicidade, que s\~ao os fatos que ocorrem sempre ap\'os um determinado per\'iodo constante. Um bom exemplo deste conceito \'e o nascer do sol, que \'e um evento que ocorre sempre ap\'os um ciclo de {24} horas, assim como o dia de seu anivers\'ario ocorre uma vez a cada ciclo de um ano.
}
\subparagraph{
O mesmo tipo de evento \'e observado com o resto dos n\'umeros inteiros. Tomemos por exemplo os restos de divis\~ao dos n\'umeros inteiros, abaixo mostrados de 1 \`a 12, pelo n\'umero inteiro {4}:
}

\[
\begin{array}{ccccccccccccc}
  {Inteiro} & 1 & 2 & 3 & 4 & 5 & 6 & 7 & 8 & 9 & 10 &  11 & 12 \\  
	{Resto} & 1 & 2 & 3 & 0 & 1 & 2 & 3 & 0 & 1 & 2  &  3 & 0 \\ 
\end{array}
\]

\subparagraph{
\'E vis\'ivel que ap\'os {4} n\'umeros o resto tende a se repetir. O mesmo feito ocorre a qualquer n\'umero inteiro $n$, onde o ciclo se repetir\'a sempre a cada $n$ itera\c{c}\~oes. Os n\'umeros que apresentam o resto {0} s\~ao conhecidos como m\'ultiplos de $n$.
}

\section{N\'{u}meros Primos e Compostos}

\subparagraph{
Existe um tipo especial de n\'umero que s\'o \'e m\'ultiplo, ou seja, possui resto {0}, em duas condi\c{c}\~oes, quando $n$ \'e igual a {1} ou quando ele \'e igual a $n$. A esse conjunto de n\'umeros atribui-se o nome de \textit{n\'umeros primos}.
}
\subparagraph{
\textit{Existem infinitos n\'umeros primos}, caso n\~ao acredite vamos supor que o conjunto finito de primos seja composto por $p_{1},  p_{2}, ..., p_{r} $. Considerando que o n\'umero inteiro $n=(p_{1})(p_{2})...(p_{r}) + 1$. $n$ deve possuir um fator $p$, que est\'a contido em $p_{1},  p_{2}, ..., p_{r} $, mas isso significa q $p$ divide $1$, o que \'e absurdo e prova que o conjunto n\~ao tem fim.
}
\subparagraph{
Todo o n\'umero que n\~ao \'e primo \'e chamado de \textit{N\'umero Composto}, sendo que este n\'umero composto pode ser escrito em \textit {uma combina\c{c}\~ao \'unica de fatores primos}. O processo de se descobrir estes fatores \'e chamado de \textit{fatora\c{c}\~ao} e \'e detalhado na pr\'oxima seção.
}

\section{Fatora\c{c}\~{a}o}

\subparagraph{
Anteriormente falamos que todo o n\'umero pode ser escrito por uma combina\c{c}\~ao de fatores primos, neste cap\'itulo vamos abordar como se pode obter estes fatores.
}
\subparagraph{
Come\c{c}amos por escolher o n\'umero inteiro $n$ ao qual iremos fatorar, em seguida testamos a sua divisibilidade por $2$, se for tente divid\'i-lo novamente por $2$, sen\~ao passa-se para o pr\'oximo n\'umero primo, o $3$. Repete-se esse procedimento at\'e chegarmos a $\sqrt{n}$, caso n\~ao achemos nenhum fator primo at\'e $\sqrt{n}$, $n$ \'e primo.
}
\subparagraph{
Quando acabamos de realizar a fatora\c{c}\~ao, chegamos a um n\'umero fatorado da forma $n = (2^{a_{1}})(3^{a_{2}}) ... (p^{a_{p}})$, todo o n\'umero inteiro pode ser escrito nessa forma, chamada forma fatorada, veja, por exemplo o $12 = (2^2)(3^1)$ e o $19 = (19^1)$.
}
\subparagraph{
Essa forma fatorada nos \'e formalmente apresentada pelo \textit{Teorema da Fatora\c{c}\~ao \'Unica}. Ele nos diz que dado um n\'umero inteiro $n\geq2$ pode-se escrev\^e-lo de forma \'unica como:
}
\[	
	\begin{array}{c}
		\textit{$n = (p^{e_{1}}_{1}) ... (p^{e_{k}}_{k}) $}
	\end{array}
\]
\paragraph{
onde $1 < p_1 < ... < p_k $ s\~ao primos e $e_1, ..., e_k$ s\~ao inteiros.
}
\subparagraph{
Mesmo algoritmo da fatora\c{c}\~ao sendo t\~ao simples de se compreender, ele \'e demorado at\'e para os mais modernos computadores. Para se ter uma ideia disto, um computador comum executa cerca de {50} divis\~oes por segundo, para se calcular com certeza que um n\'umero pr\'oximo a $10^{100}$ \'e primo ele levaria cerca de {317} decilh\~oes de anos. Essa demora computacional que torna os primos t\~ao atraentes a criptografia, pois sua multipli\c{c}\~ao \'e f\'acil para se obter o resultado, mas muito complexa para que se descubram quais os n\'umeros envolvidos nela apenas com o resultado final.
}
