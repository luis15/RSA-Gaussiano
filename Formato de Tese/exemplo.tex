\documentclass[11pt,a4paper]{article}
\usepackage{latexsym}
\usepackage{amssymb}


% OS SEGUINTES DOIS COMANDOS SAO PARA COLOCAR UM QUADRADO PRETO NO FINAL DAS PROVAS
% TROCANDO $\blacksquare$ POR OUTRO SIMBOLO TROCA O QUADRADO POR ESSE OUTRO SIMBOLO
\def\pushright#1{{\parfillskip=0pt\widowpenalty=10000
  \displaywidowpenalty=10000\finalhyphendemerits=0
  \leavevmode\unskip\nobreak\hfil\penalty50\hskip.2em\null
  \hfill{#1}\par}}

\newcommand{\curtains}     {\pushright{$\blacksquare$}\par \vspace{1ex}}


% COMANDOS PARA DEFINICAO, TEOREMA ETC.
\newtheorem{defi}{Defini\c c\~ao}[section]
\newtheorem{coro}[defi]{Corol\'ario}
\newtheorem{fato}[defi]{Fato}
\newtheorem{lema}[defi]{Lema}
\newtheorem{teor}[defi]{Teorema}
\newtheorem{prop}[defi]{Proposi\c c\~ao}
\newtheorem{exem}[defi]{Exemplo}
\newtheorem{exems}[defi]{Exemplos}
\newtheorem{obs}[defi]{Observa\c c\~ao}
\newtheorem{obss}[defi]{Observa\c c\~oes}

% COMANDO 1 PARA COMECAR E ACABAR AS DEMONSTRACOES
\newenvironment{prova1}{\begin{trivlist}
\item{\bf Demonstra\c c\~ao:}}{\curtains\end{trivlist}}

% COMANDO 2 PARA COMECAR E ACABAR AS DEMONSTRACOES
\newcommand{\prova}{{\bf Demonstra\c c\~ao: }}

\begin{document}

\title{Exemplo \LaTeX}

\author{Marcelo E. Coniglio\\
GTAL - Department of Philosophy and CLE\\
State University of Campinas \\
{\tt coniglio@cle.unicamp.br}}
\date{}
\maketitle
\begin{abstract}
Este texto \'e apenas para ver alguns comandos \LaTeX.
\end{abstract}

\section*{Introdu\c c\~ao}
S\'o para mostrar como colocar uma se\c c\~ao sem n\'umero de se\c
c\~ao.

\section{Esta \'e a primeira se\c c\~ao} \label{secao1}
Come\c camos. Como escrever teoremas, proposi\c c\~oes, etc.

\begin{teor} \label{primeiro}
Enunciado do teorema (observe que fica em it\'alico). Colocamos
uma label nele, para depois poder mencion\'a-lo.
\end{teor}

\begin{prop} \label{segundo} \em
Enunciado da proposi\c c\~ao (observe que agora n\~ao fica em
it\'alico. Por que?). Tamb\'em tem label (mas isto nao \'e
obrigat\'orio).
\end{prop}
\begin{prova1}
Assim usa-se o comando de in\'\i cio e finaliza\c c\~ao de provas.
\end{prova1}


\begin{lema}
Agora mostraremos como usar o outro comando de in\'\i cio e
finaliza\c c\~ao de provas.
\end{lema}
\prova Observe que o in\'\i cio \'e manual. O final tamb\'em,
usando ``curtains" (olhe, de passagem, o uso das aspas). \curtains

\section{Esta \'e a segunda se\c c\~ao}

Podemos passar para uma subse\c c\~ao

\subsection{Outras coisas} \label{subsecao}

Agora vamos ver outras coisas. Podemos mencionar resultados
anteriores (note que a primeira linha n\~ao aparece indentada).

Por exemplo, o Teorema \ref{primeiro} aparece em it\'alico. J\'a a
Proposi\c c\~ao \ref{segundo} n\~ao. Isso foi na Se\c c\~ao
\ref{secao1}. Agora estamos na subse\c c\~ao \ref{subsecao}.

\subsection{Mais coisas}

Faremos demonstra\c c\~oes:

$\begin{array}{lll} 1.& a \rightarrow (b \rightarrow a) &
\textrm{[Ax.]}\\
2.& a & \textrm{[Hip\'otese]}\\
3. & b\rightarrow a & \textrm{[MP 1, 2]}
\end{array}
$

\

\noindent E agora, tabelas de verdade:
$$
\begin{array}{|c|c|c|c|}\hline
  \wedge & {\bf 1} & ^{1}/_{2} & {\bf 0} \\ \hline
  {\bf 1} & 1 & ^{1}/_{2} & 0 \\ \hline
  ^{1}/_{2} & ^{1}/_{2} & ^{1}/_{2} & 0 \\ \hline
  {\bf 0} & 0 & 0 & 0 \\ \hline
\end{array}
\hspace{0.5 cm}
\begin{array}{|c|c|c|c|}\hline
  \vee & {\bf 1} & ^{1}/_{2} & {\bf 0} \\ \hline
  {\bf 1} & 1 & 1 & 1 \\ \hline
  ^{1}/_{2} & 1 & ^{1}/_{2} & ^{1}/_{2} \\ \hline
  {\bf 0} & 1 & ^{1}/_{2} & 0 \\ \hline
\end{array}
\hspace{0.5 cm}
\begin{array}{|c|c|c|c|}\hline
  \rightarrow & {\bf 1} & ^{1}/_{2} & {\bf 0} \\ \hline
  {\bf 1} & 1 & ^{1}/_{2} & 0 \\ \hline
  ^{1}/_{2} & 1 & ^{1}/_{2} & 0 \\ \hline
  {\bf 0} & 1 & 1 & 1 \\ \hline
\end{array}
\hspace{0.5 cm}
\begin{array}{|c|c|}\hline
   & \neg \\ \hline
  {\bf 1} & 0 \\ \hline
  ^{1}/_{2} & ^{1}/_{2} \\ \hline
  {\bf 0} & 1 \\ \hline
\end{array}
$$

\

\begin{obs} \em As fra\c c\~oes podem ser escritas como $\frac{n}{m}$, mas
na tabela n\~ao ficam muito bem.
\end{obs}

Finalmente, as equa\c c\~oes podem ser enumeradas para depois ser
mencionadas:

\begin{equation}
2 + 2= 3,14  \ \textrm{(Tudo \'e relativo!)}\label{muito boba}
\end{equation}

\

A equa\c c\~ao (\ref{muito boba}) faz jus ao seu nome.

\begin{obss} \em \ \\
(1) Notar que ``$\backslash$" \'e usado dentro de uma linha para
deixar um espa\c co entre as palavras. Pode usar v\'arias vezes,
mas separados por um espa\c  co, ou seja: ``$\backslash$
$\backslash$", e n\~ao ``$\backslash\backslash$". Se colocado
numa linha s\'o, ``$\backslash$" deixa uma linha vazia. Isto
serve para separar tabelas do texto (como eu acabei de fazer).\\
(2) Note que, para se referir a uma equa\c c\~ao como eu fiz
antes, tem que colocar a refer\^encia entre par\^enteses,
enquanto que, quando nos referimos a uma Se\c c\~ao, Teorema etc,
n\~ao precisamos de botar par\^enteses. \curtains
\end{obss}




\end{document}
