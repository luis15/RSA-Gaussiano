\chapter {O in�cio do que n�o existe}
\label{Cap1}

Neste cap�tulo iremos escrever alguns comandos b�sicos tais como:
\begin{enumerate}
    \item Enumerar uma lista de �tens.
    \item Inserir marcador, isto �, uma bolinha preta na frente do
          par�grafo.
    \item Outros
\end{enumerate}

\section{Enumerar uma lista de axiomas}

  A seguinte enumera��o �  obviamente auto-referente,  porque o  foi
  produzido usando-se a si mesmo...

  Utilizamos  o comando
  {\bf{description}}, mas poder�amos utilizar o {\bf{enumerate}}, como
  feito acima.

\begin{description}
    \item[1.] $\vdash_{PC}\phi \rightarrow (\psi \rightarrow \phi)$
    \item[2.] $\vdash_{PC}(\phi\rightarrow(\psi\rightarrow\lambda))\rightarrow((\phi\rightarrow\psi)\rightarrow(\phi\rightarrow\lambda))$
    \item[3.] $\vdash_{PC}((\neg\phi\rightarrow\neg\psi)\rightarrow((\neg\phi\rightarrow\psi)\rightarrow\phi))$
\end{description}


\section{Como referenciar e enumerar automaticamente teoremas, lemas,  corol�rios, etc.}

Os procedimentos  abaixo s�o auto-explicativos.

 \begin{Le}\label{lema1}
  O {\bf{label}} � utilizado para dar um c�digo ao lema, teorema ou
  defini��o, para facilitar a refer�ncia no corpo do trabalho.
\end{Le}

\begin{Cor}\label{corolario2}
  Para citar um lema, teorema ou defini��o, basta digitar
  o c�digo deste, dentro do comando {\bf{ref}} .
\end{Cor}
\begin{proof}
	   � imediato a partir do lema \ref{lema1}.
\end{proof}

\begin{Le}\label{lema3}
$\vdash_{PC}(\neg\varphi\rightarrow\varphi)\rightarrow\varphi$)
\end{Le}

\begin{proof} Vejamos:\\

$
\begin{array}{lll}
1.&\vdash_{PC}((\neg\varphi\rightarrow\neg\varphi)\rightarrow((\neg\varphi\rightarrow\varphi)\rightarrow\varphi))& [\textrm{ Ax3 }]\\
2.&\vdash_{PC}(\neg\varphi \rightarrow \neg\varphi)                                                              & [\textrm{Lm \ref{lema1}}]\\
3.&\vdash_{PC}((\neg\varphi\rightarrow\varphi)\rightarrow\varphi))                                               & [\textrm{ MP em 1 e 2 }]\\

\end{array}
$
\end{proof}

Estes comandos para enumera��o  n�o precisam ser digitados, basta
voc� procurar dentro do {\bf{insert}}, na barra de ferramentas do
Winedit (se voc� estiver  trabalhando com ele) .
