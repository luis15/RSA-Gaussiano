%%%%%%%%%%%%CONSIDERA��ES FINAIS%%%%%%%%%%%%%%%%%%%%%%%%%%%%%%%%%%%%%%%%%%%%%%%%%%%%%%

\addcontentsline{toc}{chapter}{Considera��es Finais}

\chapter*{Considera��es Finais}

Como j� sabemos, o objetivo desta nota � apenas mostrar  um pouco
dos comandos que eu julgo serem os mais utilizados no decorrer do
texto. Para maiores informa��es consulte um guia de Latex ou pe�a
auxilio aos colegas,  e obviamente ao orientador.


Incluo a seguir um exemplo de arquivo de bibliografia. Prefiro
fazer os r�tulos bastante mnem�nicos, para que eu lembre o que
estou citnadp. Por exemplo,
\cite{Sette_1973_On_the_Propositional_Calculus_P1} mas voc� pode
escreve qualquer coisa ao inv�s de
\verb"Sette_1973_On_the_Propositional_Calculus_P1" . Note que
estou usando o comando chamado  \emph{Verbatim} para escrever a
frase anterior sem que o programa entenda isso como matem�tica.
Este comando � produzido por \verb"\verb". Veja que exemplo
curioso da diferen�a de \textsl{uso} e \textsl{men��o}!

Outro exemplos de cita��o seriam
\cite{Popper_1959_The_Logic_of_Scientific_Discovey},
\cite{Dottaviano_1987_Definability_and_quantifier_elimination_for_J3_theories},etc.

Voc�  pode fazer com que apare�a  trudo o que voc� tem no banco
de dados  da  Bibliografia  com o comando  dddddd.
