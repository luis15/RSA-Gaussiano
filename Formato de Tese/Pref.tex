
%%%%%%%%%%%%%%%%%%%%%%%%%%%%%%%%%%%%%%%RESUMO%%%%%%%%%%%%%%%%%%%%%%%%%%%%%%%%%%%%%%%%%%%%%%%%%%%%%%%%%%%%%%%%%%%%%%%%%%%%%%%%

\thispagestyle{empty}

\hspace{1cm} \vspace{2.2cm}

\noindent {\Huge {\bf Resumo}}

\vspace{1.5cm}

\noindent O presente trabalho  tem por objetivo apresentar um
esquema de como construir rapidamente e sem dor  uma tese em
Latex.

Todos os dados s�o fict�cios e tudo n�o passa de uma brincadeira
com as mais s�rias inten��es.

No decarrer dos cap�tulos exibiremos como se faz automaticamente a
enumera��o de teoremas, lemas, color�rios,defini��es, notas, etc.

Deixaremos alguns exemplos a respeito de como construir tabelas,
fazer refer�ncias, e outros, sem a menor preocupa��o com o
contexto.

A id�ia � que com este formato a pessoa que est� interessada em
elaborar um trabalho em Latex consiga dar o passo inicial.

H�  dispon�veis na rede uma enorme variedade de manuais de Latex.
Para mais  informa��es veja o ap�ndice.

A maneira  mais r�pida e  simpels de editar um arquivo Latex �
atrav�s do ambiente  (programa) Winedit, mas n�o �  obrigat�rio
utiliz�-lo.
