\chapter{No caminho das nuvens}
\label{Cap2}

 Vamos neste cap�tulo nos deter a alguns detalhes tal como deixar
 o texto centralizado e como construir uma tabela.

\section{Centralizando}

Para dar destaque, por exemplo a uma equa��o no meio do texto,
deixando a mesma centralizada, basta:

$$a \wedge b \leq a $$

\noindent{digitar a f�rmula ou a equa��o entre o comando
{\bf{dolar dolar}}, como no exemplo.}

Note que para continuar o texto, foi digitado o comando
{\bf{noindent}}. Este comando serve para o programa entender que
o texto deve dar seq�encia ao texto acima, isto �, para n�o
iniciar um novo par�grafo.

\section{Tabelas}

Considere as seguintes matrizes  trivalentes onde $T$ e $T^-$ s�o
valores distinguidos:


$$
\begin{array}{|c|c|c|c|}\hline
  \wedge    & T     & T^{-}     & F \\ \hline
  T         & T^-   & T^-       & F \\ \hline
  T^-       & T^-   & T^-       & F \\ \hline
  F         & F     & F         & F \\ \hline
\end{array}
\hspace{1.5 cm}
\begin{array}{|c|c|c|c|}\hline
  \vee     & T      & T^-   & F     \\ \hline
  T        & T^-    & T^-   & T^-   \\ \hline
  T^-      & T^-    & T^-   & T^-   \\ \hline
  F        & T^-    & T^-   & F     \\ \hline
\end{array}
$$

$$
\begin{array}{|c|c|c|c|}\hline
  \rightarrow   & T     & T^-   & F     \\ \hline
  T             & T^-   & T^-   & F     \\ \hline
  T^-           & T^-   & T^-   & F     \\ \hline
  F             & T^-   & T^-   & T^-   \\ \hline
\end{array}
\hspace{1.5 cm}
\begin{array}{|c|c|c|c|c|}\hline
        & \neg_w    & \neg_s    & {\circ}_w     & {\circ}_s     \\ \hline
  T     & F         & F         & T             & T             \\ \hline
  T^-   & T^-       & F         & T             & F             \\ \hline
  F     & T         & T         & T             & T             \\ \hline
\end{array}
$$

\

As barras entre os {\bf{c}} servem para colocar as linhas
horizontais da tabela e o comando {\bf{hline}} � colocado em cada
linha que se deseja colocar a linha horizontal da tabela.

A quantidade de {\bf{c}} indicam o n�mero de colunas da tabela e
a letra {\bf{c}} indica que os dados ficar�o centralizados. Para
alinhar a direita utiliza-se a letra {\bf{r}} e para alinhar a
esquerda a letra {\bf{l}}.

Os �tens das colunas s�o separados pelo \&. Para mais detalhes
observe os exemplos.
