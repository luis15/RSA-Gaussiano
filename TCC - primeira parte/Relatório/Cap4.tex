\chapter{Organiza��o para conclus�o}
\label{Cap4}

A proposta para o desenvolvimento do trabalho consiste em estudar detalhadamente os teoremas matem�ticos envolvidos na criptografia RSA com o intuito de adapt�-los ao que chamamos de criptografia RSA gaussiana.

O trabalho ser� estruturado da seguinte forma:
\begin{enumerate}
\item{Introdu��o contendo um hist�rico sobre o desenvolvimento da criptografia focando no papel dos protocolos envolvidos nesse processo.}
\item{O primeiro cap�tulo ser� reservado para descrever a criptografia RSA, focando nos resultados matem�ticos necess�rios para a constru��o dos protocolos e seus respectivos problemas, tais como o problema P=NP.}
\item{No segundo cap�tulo iremos descrever a hist�ria e a defini��o dos n�meros primos de gauss, objetivando adaptar os teoremas vistos na Cap�tulo 1. Ainda neste cap�tulo levantaremos os principais problemas envolvidos na criptografia RSA gaussiana.}
\item{Discuss�o sobre a criptografia RSA gaussiana enfocando nos principais desafios a serem atacados.}
\end{enumerate}
