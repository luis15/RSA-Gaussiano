\pagestyle{fancy}
\fancyhead[C]{\textsl{1. Um passeio pela Teoria de N\'umeros}}
\fancyhead[R]{\thepage}
\fancyfoot[C]{}


\chapter{Um passeio pela Teoria de N\'umeros}
\label{Num}

A teoria de n\'umeros \'e umas das mais antigas \'areas da matem\'atica e dedica-se ao estudo relacionado \`as propriedades relativas aos n\'umeros inteiros tais como a quest\~ao da fatora\c{c}\~ao, m\'aximo divisor comum, entre outras. Ao longo deste cap\'itulo mostraremos os principais resultados de teoria de n\'umeros essenciais para a compreens\~ao do m\'etodo de criptografia de chave p\'ublica conhecido como RSA.

\section{N\'umeros Primos e Fatora\c{c}\~ao \'Unica}

Os n\'umeros primos ocupam lugar importante tanto na teoria de n\'umeros quanto na criptografia RSA: na primeira por serem capazes de gerar todos os elementos do conjunto dos n\'umeros inteiros e suas consequentes propriedades; na segunda pelo fato de formarem um conjunto infinito e isso permite que se tome um primo de dimens\~ao estrondosa para codificar uma mensagem, consequentemente dificultando que o c\'odigo seja decifrado por terceiros em tempo razo\'avel, como mostraremos ao longo deste trabalho.

Os primos atuam como \'atomos dentro do conjunto dos n\'umeros inteiros no sentido em que todo n\'umero pode ser escrito como um produto de primos. Esse fato \'e consequ\^encia do chamado \textit{Teorema da Fatora\c{c}\~ao \'Unica} tamb\'em conhecido por \textit{Teorema Fundamental da Aritm\'etica}. Al\'em de ser um resultado fundamental para a teoria de n\'umeros, ele tamb\'em \'e um dos pilares da criptografia RSA, pois a decodifica\c{c}\~ao de uma mensagem vai passar pela fatora\c{c}\~ao de um n\'umero, e para demonstrar esse teorema \'e preciso ter a disposi\c{c}\~ao o chamado \textit{Teorema de Divis\~ao}. 


\begin{Th}
[Teorema de divis\~ao]\label{teo.div}
Sejam $a$ e $b$ inteiros positivos. Existem n\'umeros inteiros $q$ (quociente) e $r$ (resto) tais que:	
	\begin{center}
		$a=bq+r$ e $0\leq r <b$
	\end{center}
Al\'em disso, os valores de $q$ e $r$ satisfazendo as rela\c{c}\~oes acima s\~ao \'unicos.
\end{Th} 

\begin{proof}
Confira em \cite{cou:2014}, se\c{c}\~ao 3 do cap\'itulo 1, p. 22.
\end{proof}

O teorema acima faz duas afirma\c{c}\~oes: a primeira que o quociente e o resto da divis\~ao sempre existem; a segunda, que o quociente e o resto s\~ao \'unicos. A garantia da unicidade \'e o ponto crucial na aplica\c{c}\~ao \`a criptografia RSA, pois assim temos a garantia de que uma mensagem possa ser decodificada de maneira \'unica. Um outro resultado 
igualmente importante \'e o \textit{Algoritmo de Euclides} mais conhecido como m\'etodo para se calcular o m\'aximo divisor comum entre dois n\'umeros. Esse resultado \'e importante para definir o que entendemos por n\'umeros primos e consequentemente para o teorema da fatora\c{c}\~ao \'unica que mostra como expressar um n\'umero em fatores primos de forma \'unica. Para este trabalho vamos precisar da vers\~ao estendida desse m\'etodo.

\begin{Th}[Algor\'itmo Euclidiano Estendido]\label{alg.eucl.est.}
Sejam $a$ e $b$ inteiros positivos e seja $d$ o m\'aximo divisor comum entre $a$ e $b$. Existem
inteiros $\alpha$ e $\beta$ tais que:
	$$\alpha\cdot a+\beta\cdot b=d$$
\end{Th}

Diferente do Teorema de Divis\~ao, o Teorema do M\'aximo Divisor Comum ou  Algor\'itmo Euclidiano Estendido n\~ao garante a unicidade com rela\c{c}\~ao aos inteiros $\alpha$ e $\beta$. Isso acaba sendo um complicador para a criptografia RSA, mas veremos que esse problema acaba sendo contornado por temos \`{a} disposi\c{c}\~ao um m\'etodo eficiente para calcular esses n\'umeros.

Com os resultados acima podemos, agora, definir o que entendemos por n\'umeros primos para ent\~ao atingir nossa meta com este 
cap\'itulo: o Teorema da Fatora\c{c}\~ao \'Unica.

\begin{Df}
Um n\'umero inteiro $p$ \'e \textit{primo} se $p\neq \pm 1$ e os \'unicos divisores de $p$ s\~ao $\pm 1$ e $\pm p$. 
\end{Df} 

S\~ao exemplos de n\'umeros primos: $\pm 2$, $\pm 3$, $\pm 5$, $\pm 7$, $\pm 11$, $\pm 13$, etc.

Um n\'umero inteiro, diferente de $\pm 1$, que n\~ao \'e primo \'e chamado \textit{composto}. Observe que os n\'umeros $1$ e $-1$ n\~ao s\~ao nem primos e nem compostos. A exclus\~ao desses n\'umeros do conjunto dos primos tem a finalidade de garantir a unicidade da fatora\c{c}\~ao no teorema a seguir. Um outro aspecto a se destacar acerca desse par de n\'umeros \'e que eles s\~ao os \'unicos em $\mathbb{Z}$ que admitem inverso multiplicativo. Falaremos mais sobre esse assundo mais adiante.

\begin{Th}
[Teorema da Fatora\c{c}\~ao \'Unica]\label{fat.unica} 
Dado um inteiro positivo $n\geq 2$ podemos sempre escrev\^e-lo, de modo \'unico, na forma
$$n=p_{1}^{e_1}\dots p_{k}^{e_k}$$
onde $1<p_1<p_2<p_3<\cdots<p_k$ s\~ao n\'umeros primos e $e_1, \cdots, e_k$ s\~ao inteiros positivos.
\end{Th}
\begin{proof}
	Detalhes do argumento pode ser encontrado no capítulo 2 em \cite{cou:2014}.
\end{proof}

No teorema acima, os expoentes $e_i$, para $1\leq i\leq k$ s\~ao chamados de \textit{multiplicidades}, pois indicam a quantidade de vezes que um mesmo n\'umero primo ocorre na fatora\c{c}\~ao. A prova de que \'e sempre poss\'ivel encontrar os fatores usados decompor para o n\'umero em fatores primos consiste no procedimento para fatorar um n\'umero, esse procedimento \'e chamado \textit{Algoritmo de Euclides}: trata-se do m\'etodo que se aprende na escola para fatorar
um n\'umero e que n\~ao iremos detalhar aqui. Como bem sabemos, esse m\'etodo \'e bastante ineficaz quando pensamos em n\'umeros muito grande, pois depende de realizar uma sequ\^encia bem grande de divis\~oes, dependendo do n\'umero. A prova garante que o procedimento termina, mas o que se nota \'e que tal procedimento \'e muito ineficiente no sentido em que demanda muito tempo para se chegar a uma resposta dependendo do n\'umero de desejamos fatorar. Na literatura existem v\'arios algoritmos de fatora\c{c}\~ao que tornam o m\'etodo mais eficiente, no entanto nenhum desses m\'etodos funciona bem para todos os n\'umeros inteiros. A criptografia RSA aproveita a inefici\^encia dos m\'etodos para fatorar um n\'umero para garantir a seguran\c{c}a do seu sistema. \'E um problema em aberto saber se existe ou n\~ao um m\'etodo r\'apido para fatorar n\'umeros inteiros.  

A demonstra\c{c}\~ao do teorema acima requer uma s\'erie de resultados acerca de n\'umeros primos os quais detalharemos abaixo. O interesse em apresentar as demonstra\c{c}\~oes de tais resultados \'e devido ao fato de estarmos interessados em comparar tais provas para o primos de Gauss se quisermos implementar um m\'etodo de criptografia RSA baseado nesses primos. 

\begin{Th}\label{propriedade_de_primos}
Sejam $a$ e $b$ inteiros positivos e suponhamos que $a$ e $b$ s\~ao primos entre si.
\begin{enumerate}
\item Se $b$ divide o produto $a\cdot c$ ent\~ao $b$ divide $c$.
\item Se $a$ e $b$ dividem $c$ ent\~ao o produto $a\cdot b$ divide $c$.
\end{enumerate}
\end{Th}

\begin{proof}

\begin{enumerate}
\item Se $a$ e $b$ s\~ao primos entre si, ent\~ao o m\'aximo divisor comum entre $a$ e $b$ \'e 1, isto \'e,
$mdc(a,b)=1$. Pelo Algoritmo Euclidiano Estendido (Teorema~\ref{alg.eucl.est.}), temos que existem inteiros 
$\alpha$ e $\beta$ tais que 

$$\alpha\cdot a+\beta\cdot b=1$$

Ent\~ao, multiplicando toda a express\~ao por $c$ temos que: 
$$\alpha\cdot a\cdot c+\beta\cdot b \cdot c=c\eqno(1.1)$$ 

Dado que $b$ divide $a \cdot c$ e $b$ divide $c \cdot b$, ent\~ao $b$ divide $\alpha\cdot ac+\beta\cdot bc$. Portanto, 
a partir da igualdade (1.1) temos que $b$ divide $c$.

\item Se $a$ divide $c$, ent\~ao existe $t\in \mathbb{Z}$ tal que $c=a\cdot t$. Como, por hip\'otese, $b$ divide $c$ e $a$ e $b$ s\~ao primos entre si, ent\~ao $b$ tem que dividir $t$. Logo, para algum $t$ vale que $t=b\cdot k$. Portanto, $c=a\cdot t=a(b\cdot k)=(a\cdot b)k$ \'e divis\'ivel por $a\cdot b$.  	
\end{enumerate}	
\end{proof}

\begin{Th}[Propriedade Fundamental dos Primos]\label{fundprimos}
Seja $p$ um n\'umero primo e sejam $a$ e $b$ inteiros positivos. 
Se $p$ divide o produto $a\cdot b$ ent\~ao $p$ divide $a$ ou $p$ divide $b$. 
\end{Th}

\begin{proof}
Se $a$ e $p$ n\~ao forem primos entre si ent\~ao o m\'aximo divisor comum entre eles \'e $p$, logo $p$ divide $a$. Suponhamos que $a$ e $p$ s\~ao primos entre si, isto \'e, $mdc(p,a)=1$. Como, por hip\'otese, $p$ divide $a\cdot b$, ent\~ao pelo Teorema \ref{propriedade_de_primos} segue que $p$ divide $b$.
\end{proof}

Estamos interessados, agora, em mostrar que a lista de primos \'e infinita, para tal vejamos o seguinte resultado intermedi\'ario.

\begin{Th}[Exist\^encia de Divisor Primo]\label{divisor_primo}
Se $n$ \'e um n\'umero inteiro positivo composto, ent\~ao $n$ tem um divisor primo
$p$ tal que $p\leq\sqrt{n}$.
\end{Th}

\begin{proof}
Se $n$ \'e um n\'umero composto e positivo, podemos supor que $n=a\cdot b$, com $1<a\leq b$.
\begin{enumerate}
\item De $1<a$, temos que existe um primo $p$ que divide $a$ (Teorema~\ref{fat.unica}), com $p\leq a$, da\'i $p^2\leq a^2$.
\item De $a\leq b$ temos que $a^2\leq a\cdot b=n$
\end{enumerate}
De (1) e (2) segue que $p^2\leq n$, logo $p\leq\sqrt{n}$.
\end{proof}

\begin{Th}[Infinidade de Primos]\label{inf.primos}
Existe uma quantidade infinita de n\'umeros primos.
\end{Th}

\begin{proof}
Suponha, por redu\c{c}\~ao ao absurdo, que existe apenas uma quantidade finita de n\'umeros primos: $p_1, p_2,\cdots p_n$. Tome $a=1+p_1\times p_2\times \cdots \times p_n$ um n\'umero inteiro. Claramente $a>p_i$ para cada $1\leq i\leq n$, logo $a$ deve ser um n\'umero composto, caso contr\'ario a lista acima estaria incompleta. Como $a$ \'{e} composto, pelo Teorema~\ref{divisor_primo} existe um primo $p_i$ tal que 
divide $a$. Dado que $p_i$ divide $p_1\times p_2\times \cdots \times p_n$ e divide $a$, ent\~ao $p_1$ divide $1$. Absurdo, pois o \'unico divisor de $1$ \'e ele mesmo. Portanto, \'e falso supor que a lista de primos seja finita, logo ela deve ser infinita.
\end{proof}

Existe, ainda, um outro debate acerca dos n\'umeros primos: como gerar os n\'umeros primos. Existem diversos m\'etodos para gerar os primos, como por exemplo, o Crivo de Erat\'ostenes, o mais antigo deles, mas n\~ao envolve nenhuma f\'ormula espec\'ifica. No entanto, todos esses 
m\'etodos mostram-se ineficazes, como mostra Coutinho em \cite{cou:2014}.

Como mostramos, temos o Teorema~\ref{fat.unica} que garante que um n\'umero possa ser decomposto
em fatores primos de forma \'unica e o Teorema~\ref{inf.primos} que garante uma infinidade de n\'umeros primos, no entanto, como dissemos, os procedimentos atrelados a esses resultados s\~ao todos muito ineficientes em termos computacionais. Para implementar a criptografia RSA vamos precisar de procedimentos mais eficazes e por essa raz\~ao ser\'a conveniente trabalhar com o conjunto de n\'umeros inteiros. Para isso vamos separar os n\'umeros inteiros em classes de equival\^encias, pois dessa forma ser\'a poss\'ivel operar com essas classes de forma semelhante como fazemos com os inteiros. Esse \'e justamente o tema da pr\'oxima se\c{c}\~ao.  


\section{Aritm\'{e}tica Modular}

Para compreender a intui\c{c}\~{a}o por tr\'{a}s da aritm\'{e}tica modular \'{e} interessante pensar na ideia de \textit{ciclicidade}, 
isto \'{e}, em
fatos que ocorrem ap\'{o}s um determinado per\'{i}odo constante ou ciclo. Por exemplo, o nascer do sol \'{e} um evento que marca 
que permite marcar um ciclo, pois ocorre sempre ap\'{o}s um ciclo de 24 horas; a data de seu anivers\'{a}rio \'{e} outro evento que permite marcar um ciclo, pois ocorre a cada 12 meses; e assim por diante. Trabalhar com objetos que tem um corportamento c\'{i}clico requer que tenhamos uma nova forma de operar, pois quando somamos 13 com 15 o resultado pode ser 4 se estivermos pensando em termos de horas. Quando termina um ciclo de 24 horas, nesse mesmo instante inicia-se um novo ciclo (do dia seguinte). A repeti\c{c}\~{a}o de uma mesma hora indica o completamento de um ciclo (24 horas) a partir do ponto estabelecido como marco inicial. 

Quando mostramos o processo de codifica\c{c}\~{a}o e de decodifica\c{c}\~{a}o de um c\'{o}digo em 
\textit{cifras de substitui\c{c}\~{a}o polialfab\'{e}ticas} voc\^{e} deve ter notado que precisamos repetir o alfabeto 
a fim de podermos operar com as posi\c{c}\~{o}es ocupadas por uma determinada letra do alfabeto. A repeti\c{c}\~{a}o do alfabeto 
foi usada para mostrar as diferentes posi\c{c}\~{o}es ocupadas por uma mesma letra. Observe que h\'{a} nesse processo o 
estabelecimento de um ciclo determinado pela quantidade de letras do alfabeto.

A partir dos ciclos podemos construir classes de n\'{u}meros representando as poss\'{i}veis marca\c{c}\~{o}es (inteiras). Por exemplo,  
no caso das horas temos as seguintes classes:  

\begin{itemize}
	\item $0h=\{x\in\mathbb{N}: x=0+24\}=\{0, 24, 48, 72, \cdots\}$ 
	\item $1h=\{x\in\mathbb{N}: x=1+24\}=\{1, 25, 49, 73, \cdots\}$
	\item $2h=\{x\in\mathbb{N}: x=2+24\}=\{2, 26, 50, 74, \cdots\}$
	\item $\vdots$
	\item $23h=\{x\in\mathbb{N}: x=23+24\}=\{23, 47, 71, 95, \cdots\}$
\end{itemize}

No exemplo das posi\c{o}\~{e}es ocupadas pelas letras do alfabeto temos as seguintes classes: 

\begin{itemize}
	\item $A=\{x\in\mathbb{N^*}: x=1+26\}=\{1, 27, 53, 79, \cdots\}$ 
	\item $B=\{x\in\mathbb{N^*}: x=2+26\}=\{2, 28, 54, 80, \cdots\}$
	\item $C=\{x\in\mathbb{N^*}: x=3+26\}=\{3, 29, 55, 81, \cdots\}$
	\item $\vdots$
	\item $Z=\{x\in\mathbb{N^*}: x=25+26\}=\{25, 51, 77, 103, \cdots\}$
\end{itemize}

Observe, pelos exemplos acima, que cada elemento do conjunto indica o marco inicial da contagem, isto \'{e}, cada elemento do conjunto indica a mesma hora em dias distintos, ou a mesma letra do alfabeto mas em ciclos distintos.

Ser\'{a} interessante nomearmos tais classes apenas por n\'{u}meros ao inv\'{e}s de sua denota\c{c}\~{a}o intuitiva, pois nosso interesse ser\'{a} operar com essas classes. Dessa forma, vamos estipular que o nome da classe seja dado pela sua 
primeira marca\c{c}\~{a}o, por exemplo: em $A=\{x\in\mathbb{N^*}: x=1+26\}$ o n\'{u}mero 1 indica a primeira posi\c{c}\~{a}o ocupada pela letra $A$, por isso a classe da letra $A$ ser\'{a} representada pelo n\'{u}mero $\overline{1}$. A barra \'{e} 
para diferenciar o nome da classe de seu elemento, o n\'{u}mero 1. 

Vejamos como podemos generalizar esse processo de modo a deixar claro o nome dessas classes, o tamanho do ciclo que est\'a sendo adotado 
e o contexto em que desejamos trabalhar para ent\~{a}o podermos definir as opera\c{c}\~{o}es entre esses elementos.
 

\begin{Df}
	Seja $\sim$ uma rela\c{c}\~{a}o e $X$ um conjunto, sendo $x$, $y$ e $z$ elementos de $X$. 
	Uma classe em $X$ \'{e} chamada de \textsl{equival\^{e}ncia} se 
	para todo elemento de $X$ as seguintes propriedades forem satisfeitas:
	\begin{itemize}
		\item Reflexividade: $x\sim x$; 		
		\item Simetria: se $x\sim y$, ent\~{a}o $y\sim x$; 
		\item Transitividade: se $x\sim y$ e $y\sim z$, ent\~{a}o $x\sim z$.
	\end{itemize}
\end{Df}   
	
	
	Dizemos, nesse caso, que a rela\c{c}\~ao $\sim$ \'e uma rela\c{c}\~{a}o de equival\^encia. 
		
	Seja $X$ um conjunto e $\sim$ uma rela\c{c}\~{a}o de equival\^{e}ncia definida em $X$. Denotamos por $\overline{x}$ 
	a classe de equival\^{e}ncia de $x$ e escrevemos, em s\'{i}mbolos, da seguinte forma:
	
	                     $$\overline{x}=\{y\in X: y\sim x\}$$
	
	Como cada classe \'{e} composta por elementos distintos, para garantir a unicidade da soma precimos que o 
	seguinte princ\'{i}pio seja satisfeito:
	
	\begin{Th}
		Seja $X$ um conjunto e $\sim$ uma rela\c{c}\~{a}o de equival\^{e}ncia definida em $X$, ent\~{a}o:
		$$\textrm{Se } x\in X \textrm{ e } y\in\overline{x} \textrm{ ent\~{a}o } \overline{x}=\overline{y}$$
	\end{Th}
	
	\begin{proof}
		Para mostrar a igualdade entre dois conjuntos devemos mostrar que $\overline{x}\subseteq\overline{y}$
		e $\overline{y}\subseteq\overline{x}$ s\~ao verificadas. Vamos demonstrar o primeiro caso, o outro \'{e} an\'{a}logo.
			
		Tome $z\in\overline{x}$, ent\~{a}o $z\sim x$. Dado que $y\in\overline{x}$
			ent\~{a}o $y\sim x$, da\'{i} pela propriedade sim\'{e}trica temos que $x\sim y$. Logo, j\'{a} que 
			$z\sim x$ e $x\sim y$ ent\~{a}o, pela transitividade, temos que $z\sim y$, isto \'{e}, $z\in\overline{y}$.
			Portanto, $\overline{x}\subseteq\overline{y}$. 
	\end{proof}
	
	O resultado acima mostra que duas classes de equival\^{e}ncia n\~{a}o compartilham elementos, isto \'{e}, as classes de equival\^{e}ncia
	n\~{a}o t\^{e}m elementos em comum, vide exemplos acima. Como consequ\^{e}ncia deste fato temos que o conjunto $X$ \'{e} a 
	uni\~{a}o de todas as classes de equival\^{e}ncia. 
	
	O conjunto das classes de equival\^{e}ncia em $X$ \'{e} chamado \textit{conjunto quociente de $X$ por $\sim$} e 
	seus elementos s\~{a}o formados por subconjuntos de $X$ separados pela rela\c{c}\~{a}o de equival\^{e}ncia.
	
	Nosso interesse ser\'{a} separar em classes de equival\^{e}ncia o conjunto dos n\'{u}meros inteiros, dessa forma $X$ representa
	o conjunto $\mathbb{Z}$, e $x$ um n\'{u}mero inteiro enquanto que
	$\sim$ representa alguma rela\c{c}\~{a}o estabelecida entre os n\'{u}meros inteiros. 
	
	Considere o exemplo em que as classes s\~{a}o compostas aqueles n\'{u}meros que partilham do mesmo resto quando s\~{a}o divididos por 
	5. Neste caso podemos formar cinco classes distintas em $\mathbb{Z}$, a saber:
	\begin{itemize}
		\item Classe do resto zero: $\overline{0}=\{0, 5, 10, 15, 20, \cdots\}$
		\item Classe do resto um: $\overline{1}=\{1, 6, 11, 16, 21, \cdots\}$
		\item Classe do resto dois: $\overline{2}=\{2, 7, 12, 17, 22, \cdots\}$
		\item Classe do resto tr\^{e}s: $\overline{3}=\{3, 8, 13, 18, 23, \cdots\}$
		\item Classe do resto quatro: $\overline{4}=\{4, 9, 14, 19, 24, \cdots\}$
	\end{itemize}

	O conjunto das classes de equival\^{e}ncia em $X$ \'{e} chamado \textit{conjunto quociente de $X$ por $\sim$}.
	No nosso exemplo $\{\bar{0}, \bar{1}, \bar{2}, \bar{3}, \bar{4}\}$ representa conjunto quociente 
	de $\mathbb{Z}$ pela divis\~{a}o por 5, o qual ser\'{a} denotado por $\mathbb{Z}_{5}$. Em termos gerais, 
	o conjunto quociente de $\mathbb{Z}$ pela divis\~{a}o por $n\neq 0$ \'{e} denotado por:

	$$\mathbb{Z}_{n}=\{\overline{0}, \overline{1}, \overline{2}, \cdots, \overline{n-1}\}$$ 

Pelo Teorema de Divis\~{a}o (Teorema~\ref{teo.div}) sabemos que a divis\~{a}o de $a$ pode $n$ \'{e} expressa como:
$a=kn+r$, com $0\leq r< k$. Dessa forma, podemos expressar esse fato como $a-r=kn$, para algum $k\in\mathbb{Z}$. Isso 
quer dizer que a diferen\c{c}a $a-r$ \'{e} um m\'{u}ltiplo de $n$. Ser\'{a} conveniente nos referirmos aos elementos 
de uma mesma classe em $\mathbb{Z}_{n}$ dessa maneira olhando para a diferen\c{c}a entre eles, como mostramos a seguir. 
   
Dizemos que dois inteiros $a$ e $b$ s\~{a}o \textit{congruentes m\'{o}dulo $n$} se a diferen\c{c}a $a-b$ \'{e} um 
m\'{u}ltiplo de $n$. Nesse caso, denotamos da seguinte forma:

$$a \equiv b \pmod{n}$$ 

Observe que em nosso exemplo, temos que $12 \equiv 22 \pmod{5}$, pois $12-22=-10$ e $-10$ \'{e} um m\'{u}ltiplo de 5. 

A defini\c{c}\~{a}o acima formaliza a ideia de ``pular de $n$ em $n$'' como \'{e} poss\'{i}vel notar 
pelos exemplos dados. Vejamos mais alguns exemplos de elementos equivalentes segundo um determinado pulo:
\begin{itemize}
	\item $8 \equiv 0 \pmod{4}$, pois $8-0=8$ e 8 \'{e} um m\'{u}ltiplo de 4;
	\item $14 \equiv 24  \pmod{5}$, pois $14-24=-10$ e $-10$ \'{e} um m\'{u}ltiplo de 5;
	\item $25 \equiv 5 \pmod{5}$, pois $25-5=20$ e 20 \'{e} um m\'{u}ltiplo de 5;
\end{itemize}

Observe que a congru\^{e}ncia entre dois n\'{u}meros depende do m\'{o}dulo esculhido, isto \'{e}, do tamanho 
do pulo que a ser dado. Pode-se mostrar que a congu\^{e}ncia m\'{o}dulo $n$ forma uma rela\c{c}\~{a}o de 
equival\^{e}ncia. N\~{a}o daremos essa demonstra\c{c}\~{a}o aqui e o leitor interessado pode encontr\'{a}-la no 
cap\'{i}tulo 4 em \cite{cou:2014}.

Nosso principal objetivo, agora, \'{e} mostrar como podemos operar com essas classes. Veremos que as 
opera\c{c}\~{o}es em $\mathbb{Z}_{n}$ mant\'{e}m todas as mesmas propriedade satisfeitas em $\mathbb{Z}$.

\begin{Df}
Sejam $\overline{a},\overline{b}\in\mathbb{Z}_{n}$. As opera\c{c}\~{o}es de \textit{soma}, \textit{diferen\c{c}a} 
e \textit{produto} em $\mathbb{Z}_{n}$ s\~{a}o dadas por:
	\begin{itemize}
			\item \textbf{Soma:} $\overline{a}+\overline{b}=\overline{a+b}$;
			\item \textbf{Diferen\c{c}a:} $\overline{a}-\overline{b}=\overline{a-b}$;
			\item \textbf{Produto:} $\overline{a}\times\overline{b}=\overline{a\times b}$;
	\end{itemize}
\end{Df}

Como estamos operando com classes, qualquer elemento da classe pode ser tomado como representando 
a classe a qual ele pertence, dessa forma, precisamos nos certificar de que o resultado dessas opera\c{c}\~{o}es 
independem do representante da classe. Vejamos a validade desta propriedade por meio de exemplo da soma.

Sejam $\overline{4},\overline{3}\in\mathbb{Z}_{5}$. Queremos mostrar que a soma desses elementos independe do 
elemento tomado na classe como sendo seu representante.Tome 4 e 9 como representantes da classe $\overline{4}$ e
tome 3 e 8 como representantes da classe $\overline{3}$. Ent\~{a}o:
				\begin{itemize}
					\item $\overline{4}+\overline{3}=\overline{4+3}=\overline{7}$
					\item $\overline{4}+\overline{3}=\overline{9+8}=\overline{17}$ 
				\end{itemize}
Como $7\in\overline{2}$ e $17\in\overline{2}$, ent\~{a}o $\overline{7}=\overline{17}=\overline{2}$.

Sejam $\overline{a}, \overline{b}$ e $\overline{c}$ elementos de $\mathbb{Z}_{n}$. 
As seguintes propriedades s\~{a}o v\'{a}lidas para a adi\c{c}\~{a}o e multiplica\c{c}\~{a}o:

\[
\begin{array}{rclr}
(\overline{a}+\overline{b})+\overline{c} &=& \overline{a}+(\overline{b}+\overline{c}) & \hfill\textrm{ Associatividade da soma}\\
\overline{a}+\textbf{b}									 &=& \overline{b}+\overline{c} 								& \hfill\textrm{ Comutatividade da soma}\\
\overline{a}+\overline{0} 							 &=& \overline{a} 														& \hfill\textrm{ Elemento neutro da soma}\\
\overline{a}+\overline{-a}							 &=& \overline{0} 														& \hfill\textrm{ Elemento oposto }\\
(\overline{a}\times\overline{b})\times\overline{c} &=& \overline{a}\times(\overline{b}\times\overline{c}) & \hfill\textrm{ Associatividade do produto}\\
\overline{a}\times\textbf{b}								 &=& \overline{b}\times\overline{c} 								& \hfill\textrm{ Comutatividade do produto}\\
\overline{a}\times\overline{1} 							 &=& \overline{a} 														& \hfill\textrm{ Elemento neutro do produto}\\
\overline{a}\times(\overline{b}+\overline{c})&=& \overline{a}\times\overline{b}+\overline{a}+\overline{c} 														& \hfill\textrm{ Distributividade}\\
\end{array}
\]

A demonstra\c{c}\~{a}o dessas propriedades seguem imediatamente das propriedades correspondentes em $\mathbb{Z}$.
Vejamos onde a aritm\'{e}tica modular se difere da aritim\'{e}tica em $\mathbb{Z}$. Considere as classes 
$\overline{2}$ e $\overline{6}$ em $\mathbb{Z}_{6}$. Claramente essas classes s\~{a}o diferentes
da classe $\overline{0}$, no entanto vale que:

$$\overline{2}\times\overline{3}=\overline{6}=\overline{0}$$
 
Isso indica que a aritm\'{e}tica modular se difere da aritm\'{e}tica em $\mathbb{Z}$. Fatos como esses 
permitem que certas propriedades que n\~{a}o s\~{a}o v\'{a}lidas em $\mathbb{Z}$ passem a valer em $\mathbb{Z}_{n}$. 
Por exemplo, em $\mathbb{Z}_{n}$ algumas classes v\~{a}o tem elemento inverso, como veremos mais adiante. 
Isso ser\'{a} importante para a criptografia RSA quando quisermos decodificar uma mensagem, pois teremos uma maneira de 
fazer o caminho inverso. Para isso ser\'{a} interessante coletar todos os elementos invers\'{i}veis num mesmo conjunto, 
como veremos a seguir.
  
Para obter esse 


%=======================================================================


Seguindo esse conceito vamos trabalhar construindo rela\c{c}\~oes de equival\^encia entre inteiros. Se dissermos que os n\'umeros inteiros separados de um m\'ultiplo de $n$ s\~ao equivalentes, ter\'iamos que formalmente dizer que eles s\~ao \textit{congruentes no m\'odulo $n$}. Isso sempre ocorre em casos onde $a-b$ \'e um m\'ultiplo de $n$. Para representar esse caso n\'os escrevemos:



Uma das aplica\c{c}\~oes que n\'os iremos dar as congru\^encias ser\'a invers\~ao modular. 

\begin{Th}[Invers\~ao Modular]\label{inversao}
A classe $\overline{a}$ tem inverso em $Z_n$ se, e somente se, $a$ e $n$ n\~ao s\~ao primos entre si.
\end{Th}

\begin{proof}
Suponha que $\overline{a}$ tenha inverso em $\mathbb{Z}_n$. Logo existe um $\overline{b}$ tal que:

$$\overline{a} \cdot \overline{b} \equiv 1 \pmod{n}$$

Logo:

$$a \cdot b - k \cdot n = 1 $$

e portanto $mdc(a,n) = 1$.Supondo que $mdc(a,n) = 1$, logo existem $\alpha$ e $\beta$ tais que:

$$\alpha \cdot a + \beta \cdot n = 1$$

Ou seja:

$$\alpha \cdot a \equiv 1 \pmod{n}$$
\end{proof}

O conjunto dos elementos invers\'iveis \'e muito importante. Iremos denot\'a-lo por $U(n)$. Para sintetizar podemos dizer que:

$$U(n) = \left\{\overline{a} \in Z(n) | mdc(a,n) = 1\right\}$$

Outra aplica\c{c}\~ao importante s\~ao as potencia\c{c}\~oes. Vamos ver como elas funcionam por meio de um exemplo. Supondo que iremos calcular

$$10^{135} \pmod{7}$$

Iniciaremos procurando por um elemento neutro, ou seja, por uma pot\^encia de $10$ congruente \`a $1$ no m\'odulo $7$. Procurando encontraremos que:

  $$10^1 \equiv 3 \pmod{7}$$  
	$$10^2 \equiv 2 \pmod{7}$$
	$$10^3 \equiv 6 \pmod{7}$$ 
	$$10^4 \equiv 4 \pmod{7}$$ 
	$$10^5 \equiv 5 \pmod{7}$$ 
	$$10^6 \equiv 1 \pmod{7}$$ 
	
Feito isso sabemos que $10^6$ \'e a pot\^encia a qual quer\'iamos, agora n\'os podemos decompor $135$ por $6$, obtendo que $135 = (6 \cdot 22) + 3$, o que nos leva \`{a}:

$$10^{135} \equiv (10^6)^{22} \cdot 10^3 \equiv (1)^{22} \cdot 10^3 \equiv 10^3 \equiv 6 \pmod{7}$$

Esse m\'etodos de simplifica\c{c}\~ao nos ajudar\~ao depois para as opera\c{c}\~oes aritm\'eticas relacionadas ao RSA, nas pr\'oximas sess\~oes teremos alguns teoremas que tamb\'em ir\~ao nos ajudar na simplifica\c{c}\~ao de processos.

\section{Teorema de Fermat}

\begin{Th}[Pequeno Teorema de Fermat]\label{fermat}
Se $p$ \'e um n\'umero primo e $a$ \'e um inteiro n\~ao divis\'ivel por $p$, ent\~ao:
$$a^{p-1}\equiv 1 \pmod{p}$$
\end{Th}

\begin{proof}
Como $mdc(a,p)=1$ existe um $a'$ tal que

$$a' \cdot a \equiv 1 \pmod{p}$$

Multiplicando ambos os membros de $a^p \equiv a \pmod{p}$ por $a'$ obtemos:

$$a' \cdot a \cdot a^{p+1} \equiv a' \equiv a \pmod{p}$$

Logo,

$$a^{p-1} \equiv 1 \pmod{p}$$
\end{proof}

O Teorema de Fermat \'e de grande valia para a obten\c{c}\~{a}o de congru\^encia, motivo pelo qual ele \'e usado no RSA, vejamos no exemplo como ele pode vir a ser \'util. Vamos tentar descobrir qual o valor da congru\^encia ${3^{1034}}^{2}$ no m\'odulo $1033$. Sabe-se que $1033$ \'e primo o que nos permite usar o teorema de Fermat. Aplicando-o teremos que:

$$3^{1032} \equiv 1 \pmod{1033}$$

O que faremos agora consiste em ``dividir'' $1034$ por $1032$, de forma a obter o resto da divis\~ao. Com isso iremos verificar que:

$$1034^2 \equiv 2^2 \equiv 4 \pmod{1033}$$

Chegando \`a:

$$3^{1034} \equiv 3^{1032}\cdot q + 4 \equiv (3^{1032})^{q} + 3^4 \pmod{1033}$$

Agora com a simples aplica\c{c}\~ao do Teorema de Fermat, podemos chegar a conclus\~ao que: 

$${3^{1034}}^2 \equiv 1 \cdot 81 \pmod{1033}$$

Obtemos assim que ${3^{1034}}^{2} \equiv 81 \pmod{1033}$.

\section{Teorema Chin\^es do resto}

Para sermos iniciados nesta t\'ecnica, vamos analisar o seguinte problema: Qual o menor inteiro que possui resto $1$ na divis\~ao por $3$ e resto $2$ na divis\~ao por $5$. Podemos vir a tranformar esse problema nas seguintes equa\c{c}\~oes:

$$n = 3q_1 + 1$$ $$n = 5q_2 + 2$$

Essas equa\c{c}\~oes tamb\'em podem ser denotadas em forma modular como:

$$n \equiv 1\pmod{3}$$ $$n \equiv 2 \pmod{5}$$

Essa sa\'ida modular nos deixou com apenas uma vari\'avel, mas ainda n\~ao resolveu o nosso problema. Para fazermos isso vamos substituir $n$ por $5q_2 + 2$, montando a seguine equa\c{c}\~ao modular:

$$5q_2 + 2 \equiv 1 \pmod{3}$$

Como $5 \equiv 2\pmod{3}$, substitu\'imos:

$$ 2q_2 + 2 \equiv 1 \pmod{3}$$

Feito isso, passamos $2$ para o outro lado da equa\c{c}\~ao

$$ 2q_2  \equiv -1 \pmod{3}$$

Como $-1 \equiv 2 \pmod{3}$, n\'os substitu\'imos novamente, e depois dividimos a equa\c{c}\~ao por $2$, e obtemos

$$ q_2  \equiv 1 \pmod{3}$$

Com isso, conclu\'imos que

$$ q_2  \equiv q_3 + 1 \pmod{3}$$

Embora pare\c{c}a mais uma equa\c{c}\~ao s\'o serve para tornar a resolu\c{c}\~ao mais complexa, vamos a reorganizar como

$$ q_2 = 3q_3 + 1 $$

Agora substitu\'imos

$$n = 5(3q_3 + 1) + 2 = 15q_3 +7$$

Feito isso, vamos por o $3$ em evid\^encia em todos os lugares, obtendo:

$$n = 3(5q_3) +3(2) +1 = 3(5q_3 +2)+1$$

Este procedimento foi feito apenas para provar que a equa\c{c}\~ao deixa resto 1 se dividida por 3, de forma an\'aloga, abaixo \'e mostrado como ela deixa resto $2$ quando dividida por $5$.

$$n = 5(3q_3) +5(1) +2 = 5(3q_3 +1)+2$$

Ap\'os tudo isso feito ainda n\~ao possu\'imos a solu\c{c}\~ao final, mas j\'a sabemos que \'e um n\'umero qualquer da forma $15q_3 + 7$, substituindo $q_3$ por $0$, obtemos $7$, que funciona como o resultado procurado.


Para termos a defini\c{c}\~ao formal desse teorema, vamos considerar o sistema:

$$x \equiv a \pmod{m}$$
$$x \equiv b \pmod{n}$$

Observe que, $n$ e $m$ s\~ao inteiros diferentes entre si. Tomemos $x_0$ como um n\'umero capaz de satisfazer ambas as congru\^encia de forma simult\^anea e teremos:

$$x_0 \equiv a \pmod{m}$$
$$x_0 \equiv b \pmod{n}$$

Para podermos juntar ambas as equa\c{c}\~oes modulares converteremos uma em equa\c{c}\~ao alg\'ebrica, nesse caso teremos: 

$$x_0 = a + m\cdot k$$

$k$ \'e um inteiro qualquer Feito isso, chegaremos em:

$$a + m\cdot k \equiv b \pmod{n}$$

que pode ser substitu\'ida por:

$$ m\cdot k \equiv (b-a) \pmod{n}$$

Agora vamos supor que $m$ e $n$ s\~ao primos entre si. Pelo teorema de Invers\~ao multiplicativa (\ref{inversao}), n\'os j\'a sabemos que eles possuem inverso multiplicativo um para o outro. Tomemos $m'$ como o inverso de $m$ no m\'odulo $n$. Multiplicando toda a congru\^encia por $m'$ obtemos:

$$ k \equiv m'\cdot(b-a) \pmod{n}$$

que pode ser escrita como:

$$ k \equiv m'\cdot(b-a)+n \cdot t$$


Com $t$ um inteiro qualquer. Substituindo a parte de $k$, n\'os obtemos

$$ x_0 \equiv a + m (m'\cdot(b-a)+n \cdot t) $$


Podemos ver agora que para qualquer $t$, $a + m (m'\cdot(b-a)+n \cdot t$ \'e parte da solu\c{c}\~ao da congru\^encia, sabendo disso, agora vamos a descri\c{c}\~ao do teorema.


\begin{Th}[Teorema Chin\^es do Resto]\label{chines}

Sejam $m$ e $n$ inteiros positivos primos entre si. Se $a$ e $b$ s\~ao inteiros quaisquer, ent\~ao o sistema

$$ x \equiv a (mod m) $$
$$ x \equiv b (mod n) $$

Sempre tem solu\c{c}\~ao e qualquer uma de suas solu\c{c}\~oes pode ser escrita na forma

$$ a + m \cdot(m' \cdot (b-a) + n \cdot t) $$

Onde $t$ \'e um inteiro qualquer e $m'$ \'e o inverso de $m$ no m\'odulo $n$.

\end{Th}

\section{Aplica\c{c}\~ao conjunta de teoremas}

Para encerrarmos este cap\'itulo, vejamos como usar o Teorema de Fermat(\ref{fermat}) e o Teorema Chin\^es do Resto(\ref{chines}) para resolvermos equa\c{c}\~oes modulares com n\'umeros compostos. A aplica\c{c}\~ao conjunta dos teoremas ter\'a grande valor no momento da desencripta\c{c}\~ao de uma mensagem. Tomemos por exemplo que queremos calcular:

$$ 2^{6754} \pmod{1155}$$

Nosso primeiro passo \'e fatorar o $1155$. Ao fim da fatora\c{c}\~ao vamos obter que $1155 = 3 \cdot 5 \cdot 7 \cdot 11$. Em seguida vamos aplicar o teorema de Fermat a cada um dos primos, obtendo assim:

$$ 2^2 \equiv 1 \pmod{3}$$
$$ 2^4 \equiv 1 \pmod{5}$$ 
$$ 2^6 \equiv 1 \pmod{7}$$  
$$ 2^{10} \equiv 1 \pmod{11}$$

Agora dividimos $6754$ por $p-1$ para cada um dos m\'ultiplos:

$$6754 = 2 \cdot 3377 $$
$$6754 = 4 \cdot 1688 + 2$$
$$6754 = 6 \cdot 1125 + 4$$
$$6754 = 10 \cdot 675 + 4$$ 

Em seguida substitu\'imos nas congru\^encias e as reduzimos

$$2^{6754} \equiv {2^{3377}}^{2} \equiv 1 \pmod{3} $$  
$$2^{6754} \equiv {2^{1688}}^{4} \cdot 2^2 \equiv 1 \cdot 4 \equiv  4 \pmod{5}$$  
$$2^{6754} \equiv {2^{1125}}^{6} \cdot 2^4 \equiv 1 \cdot 16\equiv  2 \pmod{7}$$ 
$$2^{6754} \equiv {2^{675}}^{10} \cdot 2^4 \equiv 1 \cdot 16\equiv  5 \pmod{11}$$  

Logo, nossa tarefa consiste em resolver o sistema

$$x \equiv 1 \pmod{3}$$
$$x \equiv 4 \pmod{5}$$
$$x \equiv 2 \pmod{7}$$
$$x \equiv 5 \pmod{11}$$  

Podemos resolver esse sistema usando o algoritmo chin\^es, vamos come\c{c}ar substituindo na primeira congru\^encia, onde $x = 3y + 1$, em seguida substitu\'imos $x$ por $y$ na segunda congru\^encia, tornando-a $3y + 1 \equiv 4 \pmod{5}$, que equivale a $y \equiv 1 \pmod{5}$, como $3$ \'e invers\'ivel no m\'odulo $5$ ele pode ser anulado na equa\c{c}\~ao. Com isso temos $x = 4+15z$ que se substituindo na terceira equa\c{c}\~ao e resolvendo obtemos $z \equiv 5 \pmod{7}$, que significa que $x = 79 + 105t$. Finalmente substituindo na \'ultima equa\c{c}\~ao, teremos que $t \equiv 6 \pmod{11}$, o que resulta em $x = 709+1155u$. Conclu\'imos com isso que $26754 \equiv 709 \pmod{1155}$ resolvendo nosso problema. 