
%%%%%%%%%%%%%%%%%%%%%%%%%%%%%%%%%%%%%%%RESUMO%%%%%%%%%%%%%%%%%%%%%%%%%%%%%%%%%%%%%%%%%%%%%%%%%%%%%%%%%%%%%%%%%%%%%%%%%%%%%%%%

\thispagestyle{empty}
%\chead{\textsl{Resumo}}
%\lhead{}
%\rhead{\thepage}
	
\chapter*{Resumo}
	
\vspace{1.5cm}

\noindent 
A motiva��o principal para esta monografia � conhecer a criptografia RSA, de forma secund�ria come�aremos a analisar a viabilidade de uma criptografia RSA baseada em n�meros primos de Gauss, isto �, n�meros primos definidos dentro de um subconjunto do corpo dos complexos, os chamados \textit{inteiros de Gauss}. 
Para iniciar esse estudo fazemos um levantamento dos principais resultados de teoria de n�meros necess�rios para a compreens�o do algoritmo RSA cl�ssico. Como o leitor ir� notar a aritm�tica modular � central nessa constru��o na medida em que resultados como o \textit{Teorema de Fermat} e o \textit{Teorema Chin\^es do Resto} s�o elementos centrais da criptografia RSA. 

Ap�s uma exposi��o detalhada do m�todo criptogr�fico RSA cl�ssico iniciamos uma discuss�o sobre a criptografia RSA Gaussiana, tema secund�rio desta monografia. Para concluir faremos uma discuss�o sobre os desafios dentro deste campo para a computa��o.





