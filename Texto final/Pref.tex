
%%%%%%%%%%%%%%%%%%%%%%%%%%%%%%%%%%%%%%%RESUMO%%%%%%%%%%%%%%%%%%%%%%%%%%%%%%%%%%%%%%%%%%%%%%%%%%%%%%%%%%%%%%%%%%%%%%%%%%%%%%%%

\thispagestyle{empty}
%\chead{\textsl{Resumo}}
%\lhead{}
%\rhead{\thepage}
	
\chapter*{Resumo}
	
\vspace{1.5cm}

\noindent 
A motiva��o para esta monografia � analisar a viabilidade de uma criptografia RSA baseada em n�meros primos de Gauss, isto �, 
n�meros primos definidos dentro de um subconjunto do corpo dos complexos, os chamados \textit{inteiros de Gauss}. 
Para iniciar esse estudo fazemos um levantamento dos principais resultados de teoria de n�meros necess�rios para 
a compreens�o do algoritmo RSA cl�ssico. Como o leitor ir� notar a aritm�tica modular � central nessa constru��o na 
medida em que resultados como o \textit{Teorema de Fermat} e o \textit{Teorema Chin\^es do Resto} s�o elementos centrais da 
criptografia RSA. 

Ap�s uma exposi��o detalhada do m�todo criptogr�fico RSA cl�ssico iniciamos uma discuss�o sobre a criptografia RSA Gaussiana, 
tema desta monografia. O percurso escolhido para essa exposi��o ser� o mesmo apresentado no caso cl�ssico tentando identificar 
os pontos fr�geis para essa contru��o. Finalmente fazemos uma discuss�o sobre os desafios dentro deste novo campo 
para a computa��o.





